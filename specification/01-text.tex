\thispagestyle{empty}
\begin{center}
    Министерство образования и науки Российской Федерации \\
    Федеральное государственное бюджетное образовательное учреждение высшего образования\\
    <<Волгоградский государственный технический университет>>\\
    Факультет электроники и вычислительной техники\\
    Кафедра <<Системы автоматизированного проектирования и поискового конструирования>>
    \vspace{1em}
\end{center}
\begin{flushright}
    \begin{center}
        \hspace*{10.5em}Утверждаю
    \end{center}
    И.о.зав. кафедрой <<САПР~и~ПК>>\\
    \UNDER{\LINE{3cm}}{\TINY{(подпись)}}\quad\UNDER{Щербаков~М.~В.}{\TINY{(инициалы, фамилия)}}\\
    <<\underline{\hspace{2em}}>> \underline{\hspace{7.5em}} \the\year\ г.
\end{flushright}
\begin{center}
    Программа формирования маршрутов общественного транспорта на основании обработки данных\\
    ТЕХНИЧЕСКОЕ ЗАДАНИЕ\\
    \vspace{2em}
    ВРБ.40461806.10.27-\SPECIFICATION.11-91\\
    ЛИСТОВ \SPAGES
\end{center}
\vspace{5em}
\begin{minipage}[t]{0.6\textwidth}
    \vspace{4em}
    \begin{flushleft}
        Нормоконтролер\\
        Садовникова~Н.~П.\\
        <<\LINE{1.5em}>>\ \LINE{7em} \the\year\ г.
    \end{flushleft}
\end{minipage}
\begin{minipage}[t]{0.39\textwidth}
    \begin{flushleft}
        Научный руководитель\\
        \underline{\smash{М.~В.~Щербаков\hspace{6em}}}\\
        <<\LINE{1.5em}>>\ \LINE{7em} \the\year\ г.\\
        Исполнитель\\
        Студент группы\\
        \underline{\smash{А.~В.~Голубев\hspace{7em}}}\\
        <<\LINE{1.5em}>>\ \LINE{7em} \the\year\ г.\\
    \end{flushleft}
\end{minipage}
\vspace{\fill}
\begin{center}
    Волгоград \the\year\ г.
\end{center}
\newpage

\tocless\part{Аннотация}
Документ содержит задание на разработку системы <<Программа формирования маршрутов общественного транспорта 
на основании обработки данных>> для автоматизации построения маршрутов общественного транспорта. В документе 
дано общее описание системы, её название, цель создания и назначение. Приведено описание основных требований 
предъявляемых к системе в целом и функциям системы, на основе которых выделены и описаны подсистемы. 
Разработана предварительная структура разрабатываемой системы. Описаны основные алгоритмы, которые будут 
использованы.
\newpage

\startcontents[sections]
\begin{changemargin}{-0.5cm}{0pt}
    \printcontents[sections]{ }{2}{\contentsname}
\end{changemargin}

% 
% :::       ::: ::::::::::: :::::::::  
% :+:       :+:     :+:     :+:    :+: 
% +:+       +:+     +:+     +:+    +:+ 
% +#+  +:+  +#+     +#+     +#++:++#+  
% +#+ +#+#+ +#+     +#+     +#+        
%  #+#+# #+#+#      #+#     #+#        
%   ###   ###   ########### ###       
%
% info:
% * http://tdocs.su/12215
% * http://www.prj-exp.ru/gost/gost_19-201-78.php
% docs: found in docs directory

% --->
\chapter{Введение}
\section{Наименование программы}
Полное наименование программы -- <<Программа формирования маршрутов 
общественного транспорта на основании обработки данных>>.

\section{Краткая характеристика области применения}
Программа предназначена для формирования начальной маршрутной сети общественного транспорта на основе 
кластеризованных данных о перемещении.

% --->
\chapter{Основания для разработки}
\section{Основания для проведения разработки}
Основание для разработки является магистерская диссертация \ldots

\section{Наименование и условное обозначение темы разработки}
Наименования темы разработки -- <<Разработка эволюционного алгоритма формирования маршрутов общественного 
транспорта на основании обработки данных>>.

Условное обозначение темы разработки (шифр темы) -- <<\ldots>>.

% --->
\chapter{Назначение разработки}
\section{Функциональное назначение}
Функциональным назначением программы является предоставление пользователю возможность формировать сеть 
маршрутов общественного транспорта на основе данных о предпочтении жителей.

% ????
\section{Эксплуатационное обозначение}
Программа должна эксплуатироваться в профильных подразделениях на объектах Заказчика.

Конечными пользователями программы должны являться сотрудники профильных подразделений объектов Заказчика.

% --->
\chapter{Требования к программе или программному изделию}
\section{Требования к функциональным характеристикам}
% 
% NEED MORE INFO: more items and sub-items
%
\subsection{Требования к составу выполняемых функций}
Программа должна обеспечивать возможность выполнения следующих ниже функций:
\begin{enumerate}
    \item загрузка и сохранение данных
    \begin{enumerate}
        \item загрузки кластеризованных данных о перемещении;
        \item преобразованию загруженных данных;
        \item сохранения расчётных данных для последующей обработки;
    \end{enumerate}
    \item построение матрицы корреспонденций
    \begin{enumerate}
        \item преобразования данных о перемещении в матрицу корреспондений
    \end{enumerate}
    \item выбор терминальных кластеров
    \begin{enumerate}
        \item анализ графа и выбор узлов отправления-назначения;
    \end{enumerate}
    \item построение маршрутной сети
    \begin{enumerate}
        \item инициализации маршрутной сети;
        \item построения маршрутной сети по заданной метрике;
    \end{enumerate}
    \item оценка построенной маршрутной сети
    \begin{enumerate}
        \item оценка маршрута по одном из критериев (длина, \ldots);
        \item общая оценка маршрутной сети;
        \item многокритериальная оценка.
    \end{enumerate}
\end{enumerate}

% другое название и разбить на несколько подпунктов
\subsubsection{Подсистемы программы}
% Подсистема построения и расчёта длины маршрута предназначена для:
% \begin{enumerate}
%     \item взаимодействия с программных обеспечение OSRM;
%     \item построения маршрута проходящего через контрольные точки;
%     \item расчёта длины построенного маршрута.
% \end{enumerate}

% Подсистема генерации маршрутной сети предназначена для:
% \begin{enumerate}
%     \item инициализации маршрутной сети;
%     \item нахождения терминальных кластеров;
%     \item построения маршрутной сети по заданной метрике.
% \end{enumerate}

% Подсистема сохранения, загрузки и преобразования данных предназначена для:
% \begin{enumerate}
%     \item загрузки кластеризованных данных о перемещении;
%     \item преобразованию загруженных данных;
%     \item сохранения расчётных данных для последующей обработки.
% \end{enumerate}

% ????
\subsection{Требования к организации входных данных}
% Входные данные программы должны быть организованы в виде отдельных файлов формата json, соответствующие 
% спецификации.

% Файлы указанного формата должны размещаться (храниться) на локальных или съёмных носителях, отформатированных 
% согласно требованиям операционной системы.

% НАПИСАТЬ ЭТО: в виде таблицы (см. доки)
\subsubsection{Кластеры предпочтений}
\ldots
\subsubsection{Точки отправления-назначения}
\ldots

% ????
\subsection{Требования к организации выходных данных}
См. Требования к организации входных данных.

% НАПИСАТЬ ЭТО: в виде таблицы (см. доки)
\subsubsection{Маршрутная сеть}
\ldots

% ????
\subsection{Требования к временным характеристикам}
Требования к временным характеристикам программы не предъявляются.

% ВОЗМОЖНО СТОИТ ИСПОЛЬЗОВАТЬ
% \section{Требования к надёжности}
% \subsection{Требования к обеспечению надёжного (устойчивого) функционирования программы}
% Надёжное (устойчивое) функционирование Программы должно быть обеспечено выполнением Заказчиком совокупности 
% организационно-технических мероприятий, а именно:
% \begin{enumerate}
%     \item организация бесперебойного питания \ldots оборудования;
%     \item использование лицензионного программного обеспечения;
%     \item регулярным выполнением рекомендаций Министерства труда и социального развития РФ, изложенных в 
%         Постановлении от 23 июля 1998 г. <<Об утверждении межотраслевых типовых норм времени на работы по
%         сервисному обслуживанию ПЭВМ и оргтехники и сопровождению программных средств>>;
%     \item регулярным выполнением требований ГОСТ 51188-98. <<Защита информации. Испытание программных средств 
%         на наличие компьютерных вирусов>>.
% \end{enumerate}

% \subsection{Время восстановления после отказа}
% Время восстановления после отказа, вызванного сбоем электропитания технических средств (иными внешними 
% факторами), не фатальным сбоем (не крахом) операционной или файловой системы, не должно превышать \ldots 
% % выбери сам ;)
% минут/часов/дней при соблюдении условий эксплуатации технических и программных средств и правильной настройки 
% операционной системы.

% \subsection{Отказы из-за некорректных действий оператора}
% Возможными считаются отказы Программы (нарушение штатного режима обслуживания) в следствие некорректных 
% действий персонала обслуживающего операционную систему, под управлением которой работает Программа. Защита 
% от подобных действий настоящим Техническим заданием не предусматривается. Меры безопасности по недопущению 
% некорректных действий персонала должны определяться соответствующими должностными инструкциями.

% \section{Условия эксплуатации}
% \subsection{Климатические условия эксплуатации}
% \subsection{Требования к видам обслуживания}
% \subsection{Требования к численности и квалификации персонала}
\section{Требования к составу и параметрам технических средств}
% ВЫБРАТЬ ЛУЧШЕЕ или скомпоновать

% В состав технических средств должен входить IBM-совместимый персональный компьютер (ПЭВМ), включающий в себя:
% \begin{enumerate}
%     \item процессор Intel Core 2 Duo с тактовой частотой, ГГц -- 2, не менее;
%     \item оперативную память объёмом, Гб -- 2, не менее;
%     \item жесткий диск объёмом, Гб -- 40, не менее;
%     \item \ldots
% \end{enumerate}

% Минимальные аппаратные требования: процессор Intel-совместимый, тактовая частота не ниже 2 ГГц, оперативная 
% память не менее 2 Гб, не менее 500 Мб свободного дискового пространства.

\section{Требования к информационной и программной совместимости}
\subsection{Требования к информационным структурам и методам решения}
Информационная структура файла должна содержать разметку, предусмотренную спецификацией формата json и 
включать в себя координаты кластеров, координаты точек отправления-назначения, а также принадлежность точек 
отправления-назначения к заданным кластерам.

Требования к информационным структурам (файлов) на входе и выходе, а также к методам решения не предъявляются.

\subsection{Требования к исходным кодам и языкам программирования}
Исходные коды должны быть реализованы на языке Python 3. К интегрированной среде разработки особых требований 
не предъявляется.

\subsection{Требования к программным средствам, используемым программой}
Программные средства, используемые программой, должны иметь свободную лицензию.

\subsection{Требования к защите информации и программ}
Требования к защите информации и программ не предъявляются.

% \section{Требования к маркировке и упаковке}
% \subsection{Требования к маркировке}
% \subsection{Требования к упаковке}
% \subsubsection{Условия упаковывания}
% \subsubsection{Порядок упаковки}
% \section{Требования к транспортированию и хранению}
% \subsection{Условия транспортирования и хранения}
% \section{Специальные требования}

% --->
\chapter{Требования к программной документации}
\section{Предварительный состав программной документации}
Состав программной документации должен включать в себя:
\begin{enumerate}
    \item техническое задание;
    \item программу и методики испытаний.
\end{enumerate}

% --->
% \chapter{Технико-экономические показатели}
% \section{Экономические преимущества разработки}

% --->
\chapter{Стадии и этапы разработки}
\section{Стадии разработки}
Разработка должна быть проведена в три стадии:
\begin{enumerate}
    \item разработка технического задания;
    \item рабочее проектирование;
    \item внедрение.
\end{enumerate}

\section{Этапы разработки}
На стадии разработки технического задания должен быть выполнен этап разработки, согласования и утверждения 
настоящего технического задания.

На стадии рабочего проектирования должны быть выполнены перечисленные ниже этапы работ:
\begin{enumerate}
    \item разработка программы;
    \item разработка программной документации;
    \item испытания программы.
\end{enumerate}

На этапе внедрения должен быть выполнен этап разработки -- подготовка и передача программы.

\section{Содержание работ по этапам}
На этапе разработки технического задания должны быть выполнены перечисленные ниже работы:
\begin{enumerate}
    \item постановка задачи;
    \item определение и уточнение требований к техническим средствам;
    \item определение требований к программе;
    \item определение стадий, этапов и сроков разработки программы и документации на неё;
    \item выбор языков программирования;
    \item согласование и утверждение технического задания.
\end{enumerate}

На этапе разработки программы должна быть выполнена работа по программированию (кодированию) и отладке 
программы.

На этапе разработки программной документации должна быть выполнена разработка программных документов в 
соответствии с требованием ГОСТ 19.101-77. Предварительный состав программной документации настоящего 
технического задания.

На этапе испытаний программы должны быть выполнены перечисленные ниже виды работ:
\begin{enumerate}
    \item разработка, согласование и утверждение программы и методики испытаний;
    \item проведение приёмо-сдаточных работ;
    \item корректировка программы и программной документации по результатам испытаний.
\end{enumerate}

На этапе подготовки и передачи программы должна быть выполнена работа по подготовке и передаче программы и 
программной документации в эксплуатацию на объектах Заказчика.

% --->
\chapter{Порядок контроля и приёмки}
\section{Виды испытаний}
Приёмо-сдаточные испытания программы должны проводиться согласно разработанной Исполнителем и согласованной 
Заказчиком Программы и методик испытаний.

Ход проведения приёмо-сдаточных испытаний Заказчик и Исполнитель документирует в Протоколе проведения 
испытаний.

\section{Общие требования к приёмке работы}
На основании Протокола проведения испытаний Исполнитель совместно с заказчиком подписывают Акт приёмки-сдачи 
программы в эксплуатацию.