\section{Общая характеристика работы}
\textbf{Актуальность темы исследования.} Быстрые изменения происходящие в городской среде, как следствие 
технического прогресса требуют формирования новых методов в планирования инфраструктуры города для 
организации комфортной жизни людей. Это относится, в том числе, и к организации транспортной 
инфраструктуры, в частности к построению маршрутов общественного транспорта. Несмотря на кажущуюся 
хаотичность все перемещения пассажиров, подчиняются определенным закономерностям, связанным с масштабом и 
планировкой городской среды. Для принятия обоснованных решений по планированию или изменению маршрутной 
сети города необходимо выявить закономерности поведения населения и сформировать обобщенную модель, на 
основе которой можно строить и оценивать варианты транспортной системы. Для получения эффективных 
результатов, следует осуществлять принятие решений на основе актуальных данных, отражающих предпочтения 
жителей. В рамках магистерской диссертации следует разработать метод построения маршрутов общественного 
транспорта на основе предпочтений жителей.

% В связи с этим, можно сформулировать научную проблему, связанную с совершенствованием маршрутной сети 
% пассажирского транспорта на основе методов обработки больших данных о предпочтениях жителей по перемещению. 

\textbf{Цель и задачи работы.} данной работы являлось разработка метода генерации маршрутов общественного 
транспорта на основе предпочтений жителей для минимизации дискомфорта перемещения в городе. Для достижения 
поставленной цели решались следующие задачи:
\begin{itemize}
    \item генерация псевдореалистичных данных кластеров предпочтений;
    \item разработка методов маршрутизации между кластерами предпочтений;
    \item модификация и использование существующих алгоритмов для задачи маршрутизации;
    \item разработка критериев оценки качества построенных маршрутов;
    \item представление построенных маршрутов на карте;
\end{itemize}

\textbf{Объектом исследования} является построения маршрутов общественного транспорта на основе актуальных 
данных о предпочтениях жителей по перемещениям в современной городской среде. 

\textbf{Предметом исследования} является разработка и применение методов методы построения маршрутов 
общественного транспорта учитывающие актуальные данные о предпочтениях жителей по перемещению и 
интенсивности пассажиропотоков в городе.

% TODO
\textbf{Гипотеза исследования} является \ldots

% TODO
\textbf{Научная новизна} диссертационной работы заключается в:
\begin{itemize}
    \item Автоматизация процесса построения начальной маршрутной сети без участия транспортного инженера.
    \item Метод построения сети маршрутов основанный на использовании данных о предпочтении жителей.
    \item \ldots
\end{itemize}

% TODO
\textbf{Практическая ценность.} \ldots

% TODO
\textbf{Публикации.} По материалам диссертации автором опубликовано 3 работы, \ldots из которых представлены 
в рецензируемом научном журнале, входящем в перечень Высшей аттестационной комиссии. 

% TODO
\textbf{Структура и объём работы.} Диссертационная работа состоит из введения, четырех глав, заключения, 
списка использованных источников из \ldots наименований и насчитывает \ldots страниц, в том числе \ldots 
страниц основного текста, \ldots рисунков, \ldots таблиц и \ldots приложения.

% Первая задача заключается в создании псевдореалистичных данных для замены отсутствующий реальных на данный 
% момент. Они нужны для работы над последующими задачами как некий приближенный аналог.

% Вторая задача заключается в разработке метода обхода кластеров предпочтений, который используя информацию о 
% пассажиропотоках генерирует оптимальный список обхода кластеров. В данной работе используется разработанный 
% метод минимального увеличения длины маршрута, основанный на идее, что идеальный маршрут должен минимально 
% отличаться по длине от проложенного для автомобиля.

% Третья задача заключается в анализе и модификации существующий алгоритмов поиска маршрутов из точки \( A \) 
% в точку \( B \), но для применения на графе дорог с учётом рельефа и других специфичных городских 
% препятствий. Используемый алгоритм должен быть оптимален по времени работы и требуемой памяти для частого 
% построения маршрутов.

% Четвёртая задача заключается в разработке критериев по которым можно будет оценить качество построенного 
% маршрута. Также предоставить данную информацию пользователю и модуля построения маршрутов для последующей 
% оптимизации.

% Пятая задача заключается в разработке web-приложения для визуализации построенных маршрутов по предыдущим 
% пунктам.

% TODO: more text
\section{Основное содержание работы}
\textbf{Во введении} обосновываются выбор темы диссертационного исследования и ее актуальность, определяются 
цели и задачи работы, объект, предмет и гипотеза исследования, формулируется научная новизна.

\textbf{В первой главе} приводятся результаты исследования предметной области. Произведён анализ общего 
состояние существующей проблемы в транспортной инфраструктуре. Описаны существующие программные продукты 
частично решающие данную проблему в полуавтоматическом режиме, но требующие вмешательства транспортного 
инженера для проектирования транспортной сети. Также рассмотрена литература по современным исследованиям в 
данной области и методам предлагаемых в них. В результате сформирован подход органично вписывающийся в 
существующую систему построения для решения поставленной задачи.

Следует отметить, что развитие транспортной инфраструктуры основывается на устаревших нормативах, не 
учитывающих стремительное увеличение личного транспорта и изменений функционального назначения городских 
пространств. Это ведет к ухудшению транспортной ситуации, снижению качества транспортного обслуживания, 
увеличению пробок, и как следствие к усилению неудовлетворенности жителей. Критичным фактором является 
игнорирование фактических предпочтений жителей по перемещению по городу при проектировании маршрутов 
городского транспорта.

Для оптимизации маршрутной сети общественного транспорта необходимо проанализировать большой объем данных, 
характеризующих численность и мобильность населения, среднее время перемещения, расположение мест приложения 
труда и жилых массивов. Источниками этих данных выступают статистические сборники, выписки о численности 
сотрудников крупных предприятий, собираемые муниципальными предприятиями общественного транспорта, 
информация о количестве проданных билетах на маршрутах общественного транспорта. Для сбора данных о 
перемещениях жителей организуется целый комплекс мероприятий по натурному подсчету пассажиропотока в 
подвижном составе общественного транспорта и на остановочных пунктах существующих маршрутов, а также 
анкетированию жителей. Такие традиционные методы являются достаточно трудоемкими, а полученные данные не в 
полной мере отражают динамично меняющуюся ситуацию. В связи с этим, необходимо использовать современные 
технологии и новые ресурсы для получения актуальных данных о предпочтениях жителей по перемещениям в городе 
и интенсивности пассажиропотоков. Основываясь на современных подходах к анализу данных можно получить 
ценную информацию для поддержки принятия решений в процессе планирования развития транспортной системы 
города.

\textbf{Во второй главе} рассмотрены и проанализированы существующие методы применимые к задаче формирования 
маршрутов, а также разработаны методы на их основе для формирования маршрутных сетей используя 
кластеризованные данные о предпочтения по перемещению в городе.

Сформулируем постановку задачи формирования маршрутов. Пусть заданы \( N_s \) – число узлов (остановочных 
пунктов), \( N_r \) – число маршрутов. Требуется построить \( N_r \) последовательностей узлов, при которых 
целевая функция качества маршрутной сети будет принимать максимальное значение. 

Задача тесно связана с задачей маршрутизацией <<из пункта А в пункт Б>>, но имеет некоторые отличия. 
Во-первых, в типовой задаче маршрутизации целевая функция – это время поездки от начала до конца маршрута, 
которое необходимо минимизировать. В случае с построением сети маршрутов общественного транспорта целевая 
функция -- интегральная, учитывающая средние показатели времени пешего хода до остановки, длины пути, 
количества пересадок, и пр. Во-вторых, в типовой задаче маршрутизации для движения выбираются любые 
промежуточные точки, сокращающие маршрут, тогда как в рассматриваемой задаче промежуточные точки 
расположены там же, где и центры кластеров.

Предлагается две стратегии генерации исходной сети маршрутов общественного транспорта. Первая предполагает 
генерацию одного <<длинного>> маршрута, обходящего все исходные узлы. Далее осуществляется разрез маршрута 
на \( N_r \) маршрутов и осуществляется модификация сети в соответствии с эволюционным алгоритмом. Вторая 
стратегия: формирование начальной сети, число маршрутов в которой соответствует заданному значению \( N_r \) 
и применение эволюционных алгоритмов для поиска сети маршрутов с минимальной длиной.

\textbf{В третье главе} описана методология проектирования ПО, разработана методика проведения эксперимента, 
произведено испытание разработанных алгоритмов, а также обсуждены полученные результаты в ходе эксперимента 
и сделан вывод на их основе.

Предложенные алгоритмы были реализованы с использованием языка программирования Python и сервиса построения 
маршрутов Open Source Routing Machine (OSRM) для расчёта расстояния между узлами графа по городским дорогам.

Для оценки эффективности алгоритма и изучения его специфики, были проведены эксперименты в ходе которых 
менялось количество узлов в дорожном графе и количество создаваемых маршрутов в городской сети, а также 
несколько вариантов реализации данного метода. Наиболее интересным случаем является, когда обрабатывается 
большое число узлов в графе или большое число геопространственных данных.

\textbf{В четвёртой главе} \ldots

\textbf{В заключении} приведены основные результаты работы и определены наиболее перспективные направления 
их развития.

\textbf{В приложении} приведены материалы справочного, иллюстративного характера и техническое задание на 
создание программы формирования маршрутов общественного транспорта на основании обработки данных.

% TODO
\section{Основные результаты работы}
\begin{itemize}
    \item \ldots
\end{itemize}

\section{Перспективные направления развития работы}
Предложенный алгоритм может быть использован для построения начальной маршрутной сети городского транспорта.
Разработанный модуль может быть использован как компонент системы \ldots

\renewcommand{\bibname}{Публикации по теме диссертации}
\begin{thebibliography}{10}
    \bibitem{first} Golubev A, Chechetkin I, Solnushkin K.S., Sadovnikova N., Parygin D., Shcherbakov M., 
        Brebels A., Strategway: web solutions for building public transportation routes using big geodata 
        analysis // Proceedings of The 17th International Conference on Information Integration and 
        Web-based Applications \& Services (iiWAS2015) (December 11 - 13, 2015 Brussels, Belgium) 
        ACM New York, New York pp. 665 - 668
    \bibitem{second} Садовникова Н.П., Щербаков М.В., Парыгин Д.С., Солнушкин К.С., Голубев А.В., 
        Чечёткин И.А. Комплекс инструментов интеллектуального анализа данных strategway для поддержки 
        принятия решений по управлению развитием инфраструктуры города / В сборнике: Развитие средних 
        городов: замысел, модели, практика Материалы III Международной научно-практической конференции. 
        Волгоград, 2015. С. 147-150
    \bibitem{third} М.В. Щербаков, Н.П. Садовникова, Д.С. Парыгин, А.В. Голубев, И.А. Чечеткин 
        Автоматизация поддержки принятия решений по разработке маршрутов общественного транспорта на 
        основе анализа данных о корреспонденциях жителей // Вестник компьютерных и информационных 
        технологий (сдана в редакцию)
\end{thebibliography}