\chapter{Теоретический анализ модели}

\section{Функционал свободной энергии}

Сверхпроводимость 1,5-го рода изучается с помощью следующего двухкомпонентного 
функционала свободной энергии Гинзбурга-Ландау:
\begin{gather}
    F = \frac{1}{2}(D\psi_1)(D\psi_1)^* + \frac{1}{2}(D\psi_2)(D\psi_2)^* - 
        \nu Re\left( (D\psi_1)(D\psi_2)^* \right) + \nonumber \\
        + \frac{1}{2}\left(\nabla\times\vec{A}\right)^2 + F_p
    \label{eq:1}
\end{gather}

Здесь \( D = \nabla + ie\vec{A} \) и \( \psi_a = |\psi_a|e^{i\theta_a} \), 
\( a = 1,2 \), представляет собой две сверхтекучих компоненты, которые в 
двухщелевом сверхпроводнике соответствуют двум сверхтекучим плотностям в 
в различных диапазонах. Слагаемое \( F_p \) содержит в текущем анализе 
произвольный набор не градиентных членов.

Особая форма двухкомпонентной модели ГЛ точно выведенная в
\cite{bib:8,bib:9,bib:10} для двухщелевых сверхпроводников представлена в виде:
\begin{gather}
    F = \frac{1}{2}(D\psi_1)(D\psi_1)^* + \frac{1}{2}(D\psi_2)(D\psi_2)^* - 
        \nu Re\left\{ (D\psi_1)(D\psi_2)^* \right\} + \nonumber \\
        + \frac{1}{2}\left(\nabla\times\vec{A}\right)^2 + 
        \alpha_1|\psi_1|^2 + \frac{1}{2}\beta_1|\psi_1|^4 + 
        \alpha_2|\psi_2|^2 + \frac{1}{2}\beta_2|\psi_2|^4 + \nonumber \\
        + \eta_1|\psi_1||\psi_2| \cos(\theta_1-\theta_2) + 
        \eta_2|\psi_1|^4|\psi_2|^2
    \label{eq:2}
\end{gather}

Первые два слагаемых представляют стандартный градиентный член 
Гинзбурга-Ландау, второе слагаемое представляет смешанные градиентные 
взаимодействия, которые появляются в двухщелевых сверхпроводниках с примесным 
рассеянием \cite{bib:8,bib:9}. Следующее член является плотностью энергии 
магнитного поля, а остальные слагаемые представляют эффективный потенциал. 
Здесь же отметим, что \( \alpha_1 \) и \( \alpha_2 \) могут менять знак 
при различных температурах. Режим, где значение \( \alpha_1 \) положительно, 
в то время \( \alpha_2 \) является отрицательным, соответствует ситуации, 
когда одина из группы не имеет собственной сверхпроводимости, но, тем не менее 
имеет некоторые сверхтекучие плотности из-за межзонного туннелирования 
Джозефсона, которая представлена 
\( \eta|\psi_1||\psi_2|\cos(\theta_1-\theta_2) \) слагаемым. Поведения 
сверхпроводника 1,5-го рода в этом режиме был исследован в \cite{bib:2}. В 
данной работе сосредоточимся в основном на ситуации когда обе зоны являются 
активными, то есть при \( \alpha_{1,2} < 0 \). Для общности добавим слагаемое 
более высокого порядка связи \( \eta_2|\psi_1|^2|\psi_2|^2 \). Также 
рассмотрим случай независимо сохраняющихся конденсатов, где третий и девятый 
члены в \eqref{eq:2} запрещены на основании симметрии, то есть 
\( \nu = \eta_1 = 0 \). Переход между текущими единицами и общепринятыми 
представлен в Приложении Б.

Точно выведенная модель ГЛ \eqref{eq:2} требует малость полей \( |\psi_a| \). 
Однако это не требует в принципе от \( \alpha_a \) менять знак при той же 
температуре. Кроме того, как и в случае однокомпонентной теории ГЛ модель 
\eqref{eq:2} даёт во многих случаях приемлемую картину в низкотемпературном 
режиме. Фактически, анализ может в некоторых случаях дать качественную картину 
для случая, когда одно из полей не обладает эффективным потенциалом ГЛ-типа, 
так как режим, где одна из зон в лондоновском приближении (т.е. она не 
обладает эффективным потенциалом ГЛ, но небольшое ядро вихря моделируется 
резкой границей отсечки) может быть восстановлена из анализа, как предельный 
случай. Из анализа представленного ниже будет понятна видна поддержка 
сверхпроводимости 1,5-го рода.

\section{Вихревая асимптотика}

Ключом к пониманию взаимодействия хорошо разделенных вихрей является анализ 
при больших \( r \) асимптотического вихревого решения. Проанализируем эту 
проблему в контексте общей двухкомпонентной модели ГЛ \eqref{eq:1}, чью 
свободную энергия можно представить в виде
\begin{equation}
    F = \frac{1}{2}\left( D_i \psi_1 \right)^{*} D_i \psi_1 + 
        \frac{1}{2}\left( D_i \psi_2 \right)^{*} D_i \psi_2 + 
        \frac{1}{2}\left( \partial_1 A_2 - \partial_2 A_1 \right)^2 + F_p
    \label{eq:3}
\end{equation}
где \( F_p \) содержит все не градиентные члены. Эта свободная энергия 
соответствует \eqref{eq:2} в случае \( \nu = 0 \). Точная форма \( F_p \) в 
данном случае не является решающим фактором для анализа. При калибровочной 
инвариантности, это может зависеть только через \( |\psi_1|, |\psi_2| \) и 
(если конденсаты не являются независимо сохраняющимися) на 
\( \theta_1 - \theta_2 \). Будем считать, что \( F_p \) принимает минимальное 
значение (которое должно быть приведено к 0), когда 
\( |\psi_1| = u_1 > 0, |\psi_2| = u_2 > 0 \) и \( \theta_1 - \theta_2 = 0 \).
Таким образом, либо нет связи фаз (\( F_p \) не зависит от 
\( \theta_1 - \theta_2 \)) и выбор \( \theta_1 - \theta_2 = 0 \) произволен, 
или связь фаз является таковой, что стимулирует её синхронизацию. 

Уравнения поля получаются из \( F \), при условии, что общая свободная энергия 
\( E = \int F dx_1 dx_2 \) является стационарной по отношению ко всем 
изменениям \( \psi_1, \psi_2 \) и \( A_i \). Обычный расчёт даёт 
\begin{gather}
    D_i D_i \psi_a = 2\pder{F_p}{\psi_a^{*}}
    \label{eq:4} \\
    \partial_i \left( \partial_i A_j - \partial_j A_i \right) = 
        e\sum\limits_{a=1}^{2}\mathrm{Im} \left( \psi_a^* D_j \psi_a \right)
    \label{eq:5}
\end{gather}
Решение этой пары связанных нелинейных дифференциальных уравнений в частных 
производных можно представить в виде
\begin{gather}
    \psi_a = f_a(r)e^{i\theta} \nonumber \\
    (A_1, A_2) = \frac{a(r)}{r}(-\sin\theta, \cos\theta)
    \label{eq:6}
\end{gather}
где \( f_1, f_2, a \) вещественные функции профиля. Примем во внимание, что 
в некоторых случаях смешанные градиентные слагаемые выступают не за 
осесимметричное решение. Рассмотрим только аксиально-симметричные вихри. 
Поля, в пределах указанного выше подхода, удовлетворяют уравнениям поля, если и 
только если функция профиля \( f_1(r), f_2(r), a(r) \) удовлетворяют взаимной 
системе дифференциальных уравнений
\begin{gather}
    f''_a + \frac{1}{r} f'_a - \frac{1}{r^2}(1+ea)^2 f_a = 
        \left. \pder{F_p}{|\psi_a|} \right|_{(u_1, u_2, 0)}
    \label{eq:7} \\
    a'' - \frac{1}{r} a' - e(1+ea)(f_1^2+f_2^2) = 0
    \label{eq:8}
\end{gather}
Потребуем чтобы вихревое решение имело поведение на границе вида 
\( f_a(r) \rightarrow u_a \), \( a(r) \rightarrow -1/e \) при
\( r \rightarrow \infty \). Так, для больших значениях \( r \) величины
\begin{equation}
    \epsilon_a(r) = f_a(r) - u_a, \quad
    \alpha(r) = a(r) + \frac{1}{e}
    \label{eq:9}
\end{equation}
малы и удовлетворяющие линеаризации \eqref{eq:7},\eqref{eq:8} относительно 
\( (u_1, u_2, -1/e) \). То есть, при больших \( r \),
\begin{gather}
    \epsilon''_a + \frac{1}{r} \epsilon'_a = \sum\limits_{b=1}^{2}
        \mathcal{H}_{ab} \epsilon_b
    \label{eq:10} \\
    \alpha'' - \frac{1}{r} \alpha' - e^2(u_1^2 + u_2^2 )\alpha = 0
    \label{eq:11}
\end{gather}
где \( \mathcal{H} \) является матрицей Гессе \( F_p(|\psi_1|, |\psi_2|, 0) \) 
и его минимум
\begin{equation}
    \mathcal{H}_{ab} = \left. \pcder{F_p}{|\psi_a|}{|\psi_b|} 
        \right|_{(u_1, u_2, 0)}
    \label{eq:12}
\end{equation}
Так \( \alpha \) асимптотически отделяется от \( \epsilon_1, \epsilon_2\) и 
сразу видно, что
\begin{equation}
    \alpha(r) = q_0 r K_1(\mu_A r), \quad
    \mu_a = e\sqrt{u_1^2 + u_2^2}
    \label{eq:13}
\end{equation}
где \( K_n \) обозначает \( n \)-ую модифицированную функцию Бесселя второго 
рода, и \( q_0 \) неизвестная действительная постоянная. Таким образом
\begin{equation}
    \vec{A} \sim \left( -\frac{1}{er} + q_0 K_1(\mu_A r) \right)
        (-\sin\theta, \cos\theta)
    \label{eq:14}
\end{equation}
Так, для всех \( n \), 
\begin{equation}
    K_n(s) \sim \sqrt{\frac{\pi}{2s}}e^{-s} \text{ as } s \rightarrow \infty
    \label{eq:15}
\end{equation}
отсюда вытекает, что магнитное поле затухает экспоненциально в зависимости от 
\( r \), с масштабной величиной (глубиной проникновения)
\begin{equation}
    \lambda \equiv \frac{1}{\mu_A} = \frac{1}{e\sqrt{u_1^2 + u_2^2}}
    \label{eq:16}
\end{equation}

С другой стороны, \eqref{eq:10} представляет, в общем, пару связанных 
обыкновенных дифференциальных уравнений для \( \epsilon_1, \epsilon_2 \). Так 
как \( (u_1, u_2, 0 ) \) является минимумом от 
\( F_p(|\psi_1|, |\psi_2|, \theta_1 - \theta_2) \), где матрица Гессе является  
симметричной и положительно определенной действительной матрицей размера 
\( 2\times2 \). Следовательно, её собственные числа \( \mu_1^2, \mu_2^2\), 
допустим вещественны и положительны, и тогда её собственные векторы 
\( v_1, v_2 \) формируют ортонормированный базис на \( \mathbb{R} \). Расширяя 
базис \( v_1, v_2 \)
\begin{equation}
    \epsilon(r) = \chi_1(r) v_1 + \chi_2(r) v_2
    \label{eq:17}
\end{equation}
видим, что \( \chi_1, \chi_2 \) удовлетворяет несвязанной паре обычных 
дифференциальных уравнений
\begin{equation}
    \chi''_a + \frac{1}{r}\chi'_a = \mu_a^2 \chi_a
    \label{eq:18}
\end{equation}
откуда
\begin{equation}
    \chi_a(r) = q_a K_0(\mu_a r)
    \label{eq:19}
\end{equation}
где \( q_1, q_2 \) некоторые неизвестные константы. Так как \( v_1, v_2 \) 
являются ортонормированными, то существует угол \( \Theta \), называемый углом 
смешивания, такой, что собственные векторы \( \mathcal{H} \) являются
\begin{equation}
    v_1 = \left( \begin{array}{c}
        \cos\Theta \\
        \sin\Theta
    \end{array} \right), \quad
    v_2 = \left( \begin{array}{c}
        -\sin\Theta \\
        \cos\Theta
    \end{array} \right)
    \label{eq:20}
\end{equation}
Так, при больших \( r \) поля плотности ведут себя как 
\begin{gather}
    \psi_1 \sim \left[ u_1 + q_1\cos\Theta K_0(\mu_1 r) - 
        q_2\sin\Theta K_0(\mu_2 r) \right]e^{i\theta} \nonumber \\
    \psi_2 \sim \left[ u_2 + q_1\sin\Theta K_0(\mu_1 r) - 
        q_2\cos\Theta K_0(\mu_2 r) \right]e^{i\theta}
    \label{eq:21}
\end{gather}
где, ещё раз, \( K_0 \) -- функции Бесселя.

Из этого анализа следует, что:
\begin{enumerate}
    \item В целом есть три фундаментальные масштабные длины в задаче (в 
        отличие от двух масштабных длин в однокомпонентной теории 
        Гинзбурга-Ландау), которые проявляются в вихревых асимптотиках, а 
        именно \( 1/\mu_A, 1/\mu_1 \) и \( 1/\mu_2 \).
    \item Они получаются из вакуумного среднего \( u_a \) в \( |\psi_a| \) 
        (в случае с \( 1/\mu_A \)) и из собственных значений 
        \( \mathcal{H} \), матрица Гессе \( F_p \) (т.е. основного состояния).
    \item \( 1/\mu_{A} \) может быть интерпретирована как лондоновская глубина 
        проникновения магнитного поля.
    \item Однако если угол смешивания \( \Theta \) не является кратным  
        \( \pi/2, 1/\mu_1 \) и тогда \( 1/\mu_2 \) не может быть истолкована 
        как длина когерентности \( \psi_1, \psi_2 \) в обычном смысле. Это 
        потому, что нормальные режимы теории поля, близкие к вакууму не 
        \( |\psi_a| - u_a \), а скорее
        \[ 
            \chi_1 = (|\psi_1| - u_1)\cos\Theta - (|\psi_2| - u_2)\sin\Theta 
        \]
        \[ 
            \chi_2 = (|\psi_1| - u_1)\sin\Theta - (|\psi_2| - u_2)\cos\Theta 
        \]
        получаемые поворотом на угол смешивания \( \Theta \), который также 
        определяется из \( \mathcal{H} \). Поэтому в целом (например, в 
        присутствии межкомпонентной джозефсоновской связи) для однопоточного
        квантово-осесимметричного вихря, восстановление обоих полей
        \( \psi_a \) в очень большом диапазоне будет происходить по тому же 
        экспоненциальному закону, который устанавливает наименьшее из масс 
        \( \mu_1, \mu_2 \); Следует использовать представление в терминах 
        \( \chi_{1,2} \), которые будут должным образом представлять два 
        пространственных масштаба, связанных с восстановлением плотности.
    \item Этот анализ говорит нам только о вихревой структуре при больших
        \( r \). Это не дает прямую информацию о ядре вихря, чтобы 
        количественно понять природу вихревых взаимодействий на переходных 
        и коротких расстояниях.
\end{enumerate}

Поскольку калибровочное поле является посредником силы отталкивания между 
вихрями, а поля конденсатов посредником силы притяжения, ясно, что можно 
считать из приведенного выше анализа условие, при котором межвихревая сила 
является притягивающей на большом расстоянии: потребуем, чтобы \( 1/\mu_A \) 
являлось не самым большим из трёх пространственных масштабных длин, или, 
более явно, чтобы (по крайней мере) одно из собственных значений 
\( \mathcal{H} \) должно быть меньше, чем \( \mu_A^2 = e^2(u_1^2 + u_2^2) \). 
Можно предсказать явную формулу для дальнодействующего двухвихревого 
потенциала взаимодействия, с помощью формализма точечного вихря \cite{bib:19} 
(краткий обзор метода приведён в Приложении Б). Это основывается на 
наблюдении, что недалеко от его ядра, поле вихря совпадают с гипотетической 
точечной частицей в линейной теории с двумя полями Клейна-Гордона 
(\( \chi_1 \) и \( \chi_2 \) указанных выше) массы и векторного поля (\( A \)) 
массы \( \mu_A \). Точечная частица несёт монопольные скалярные заряды 
\( 2\pi q_1 \) и \( 2\pi q_2 \), и магнитный дипольный момент \( 2\pi q_0 \). 
Две таких гипотетических частицы, удерживаемые на расстоянии \( r \) 
испытывают потенциал взаимодействия
\begin{equation}
    V(r) = 2\pi\left[ q_0^2 K_0(\mu_A r) - q_1^2 K_0(\mu_1 r) - 
        q_2^2 K_0(\mu_2 r) \right]
    \label{eq:22}
\end{equation}

Эта формула воспроизводит предсказанное выше объяснение: взаимодействие на 
больших расстояниях будет притягивающим, если (по крайней мере) один из 
\( \mu_1, \mu_2 \) меньше чем \( \mu_A \). 

Можно задаться вопросом: является приближённая линеаризация при малых 
параметрах \( \alpha(r), \chi_1(r), \chi_2(r) \) оправданной? Строгий анализ 
однокомпонентной модели \cite{bib:20} показывает, что если любая из масс 
скалярного режима, скажем превышает \( 2\mu_A \), то квадратичные члены в 
\( \alpha \) становятся сопоставимыми при больших \( r \) с линейными членами 
\( \chi_2 \), так что уравнение для \( \chi_2 \) должно включать в себя 
дополнительные условия. В этом случае \( \chi_2 \) убывает как 
\( K_0(\mu_A r)^2 \), а не как \( K_0(\mu_2 r) \). Следует отметить, что если 
\( \mu_1 > 2\mu_A \), то главное слагаемое в \eqref{eq:21}, спадает как 
\( K_0(\mu_1 r) \), и решение по-прежнему является верным, и это только 
главное слагаемое, которое определяет характер (притяжение или отталкивание) в 
межвихревом взаимодействии на больших расстояниях. Особый интерес представляет 
случай, когда дальнодействующая сила является притягивающей, то есть, когда 
хотя бы один из \( \mu_1, \mu_2 \) является меньше чем \( \mu_A \), так 
анализ линеаризованного уравнения, представленный выше является достаточным 
для текущих целей. \cite{bib:main}

\newpage