\chapter{Введение}

% место для общего текста

\section{Актуальность}
\textbf{Актуальность.} Изменения в городской среде требуют формирования новых механизмов 
планирования инфраструктуры города. Для получения эффективных результатов, следует осуществлять 
принятие решений на основе актуальных данных, отражающих предпочтения жителей. В рамках 
магистерской диссертации следует разработать метод построения маршрутов общественного транспорта 
на основе предпочтений жителей.

\section{Цели и задачи}

\textbf{Цель работы} -- разработка метода генерации маршрутов общественного транспорта на основе 
предпочтений жителей для минимизации дискомфорта перемещения в городе.

\textbf{Теоретический этап} -- рассмотрение информации по существующим алгоритмам, свободных систем 
маршрутизации; составление теоретической базы проекта и систематизация полученных знаний; 
составление технического задания. 

\textbf{Практический этап} -- реализация системы на основе теоретической базы, составленной ранее с 
использование технического задания.

\textbf{Финальный этап} -- внедрение готового продукта, получение обратной информации и исправление 
ошибок.

\textbf{Теоретические задачи:}
\vspace*{-1em}
\begin{itemize}\itemsep-5pt
    \item разработка алгоритма формирования маршрутов на основе имеющихся данных;
    \item выбор критериев качества для оценки построенных маршрутов;
    \item методы построение маршрута по заданным критериям;
    \begin{itemize}\itemsep-5pt
        \item предпочтения жителей;
        \item длина маршрута;
        \item дискомфорт перемещения;
        \item и др.
    \end{itemize}
    \item разработка критериев для оценки качества построенного маршрута;
\end{itemize}
\textbf{Практические задачи:}
\vspace*{-1em}
\begin{itemize}\itemsep-5pt
    \item разработка механизма генерация исходных данных;
    \item реализация разработанных алгоритмов и методов;
    \item отображение результатов генерации маршрутов на карте;
    \item оценка качества построенных маршрутов.
\end{itemize}

\textbf{Объект исследования} -- построения маршрутов общественного транспорта на основе актуальных 
данных о предпочтениях жителей по перемещениям в современной городской среде.

\textbf{Предмет исследования} -- методы построения маршрутов общественного транспорта учитывающие 
актуальные данные о предпочтениях жителей по перемещению и интенсивности пассажиропотоков в городе.

\chapter{Описание решаемых задач}
В данной работе рассматриваются к решению следующие задачи:
\begin{enumerate}\itemsep-5pt
    \item генерация псевдореалистичных данных кластеров предпочтений;
    \item разработка метода маршрутизации между кластерами предпочтений;
    \item модификация и использование существующих алгоритмов для задачи маршрутизации;
    \item разработка критериев оценки качества построенных маршрутов;
    \item представление построенных маршрутов на карте;
\end{enumerate}

Первая задача заключается в создании псевдореалистичных данных для замены отсутствующий реальных 
на данный момент. Они нужны для работы над последующими задачами как некий приближенный аналог.

Вторая задача заключается в разработке метода обхода кластеров предпочтений, который используя 
информацию о пассажиропотоках генерирует оптимальный список обхода кластеров. Метод будет 
основывается на эволюционных алгоритмах. На текущем этапе используется алгоритм имитации отжига 
для генерации субоптимальных списков обхода. В дальнейшем планируется задействовать алгоритм 
табу поиска и возможно разработка нового метода на их основе.

Третья задача заключается в анализе и модификации существующий алгоритмов поиска маршрутов из 
точки \( A \) в точку \( B \), но для применения на графе дорог с учётом рельефа и других 
специфичных городских препятствий. Используемый алгоритм должен быть оптимален по времени работы 
и требуемой памяти для частого построения маршрутов.

Четвёртая задача заключается в разработке критериев по которым можно будет оценить качество 
построенного маршрута. Также предоставить данную информацию пользователю и модуля построения 
маршрутов для последующей оптимизации.

Пятая задача заключается в разработке web-инструмента для отображения и редактирования построенных 
маршрутов по предыдущим пунктам.

\section{Ожидаемые результаты}
% добавить текст