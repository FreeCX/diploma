\part{Введение}
Быстрый изменения происходящие в городской среде, как следствие технического прогресса требуют формирования 
новых методов в планирования инфраструктуры города для организации комфортной жизни людей. Это относится, в 
том числе, и к организации транспортной инфраструктуры, в частности к построению маршрутов общественного 
транспорта. Несмотря на кажущуюся хаотичность все перемещения пассажиров, подчиняются определенным 
закономерностям, связанным с масштабом и планировкой городской среды. Для принятия обоснованных решений по 
планированию или изменению маршрутной сети города необходимо выявить закономерности поведения населения и 
сформировать обобщенную модель, на основе которой можно строить и оценивать варианты транспортной системы. 
В связи с этим, можно сформулировать научную проблему, связанную с совершенствованием маршрутной сети 
пассажирского транспорта на основе методов обработки больших данных о предпочтениях жителей по перемещению. 

\emph{Актуальность.} Изменения в городской среде требуют формирования новых механизмов планирования 
инфраструктуры города. Для получения эффективных результатов, следует осуществлять принятие решений на основе 
актуальных данных, отражающих предпочтения жителей. В рамках магистерской диссертации следует разработать 
метод построения маршрутов общественного транспорта на основе предпочтений жителей.

\emph{Объект и предмет исследования}. Объектом исследования магистерской работы является построения маршрутов общественного транспорта на основе актуальных данных о предпочтениях жителей по перемещениям в современной 
городской среде. Предметом исследования является разработка и применение методов методы построения маршрутов общественного транспорта учитывающие актуальные данные о предпочтениях жителей по перемещению и интенсивности пассажиропотоков в городе.

% ...возможно здесь чего-то не хватает...

В связи с этим целью данной работы являлось разработка метода генерации маршрутов общественного транспорта на 
основе предпочтений жителей для минимизации дискомфорта перемещения в городе.

Для достижения поставленной цели решались следующие задачи:
\vspace*{-1em}
\begin{enumerate}\itemsep-5pt
    \item генерация псевдореалистичных данных кластеров предпочтений;
    \item разработка методов маршрутизации между кластерами предпочтений;
    \item модификация и использование существующих алгоритмов для задачи маршрутизации;
    \item разработка критериев оценки качества построенных маршрутов;
    \item представление построенных маршрутов на карте;
\end{enumerate}

Первая задача заключается в создании псевдореалистичных данных для замены отсутствующий реальных на данный 
момент. Они нужны для работы над последующими задачами как некий приближенный аналог.

Вторая задача заключается в разработке метода обхода кластеров предпочтений, который используя информацию о 
пассажиропотоках генерирует оптимальный список обхода кластеров. Метод будет основывается на эволюционных 
алгоритмах. На текущем этапе используется алгоритм имитации отжига для генерации субоптимальных списков 
обхода. В дальнейшем планируется задействовать алгоритм табу поиска и возможно разработка нового метода на 
их основе.

Третья задача заключается в анализе и модификации существующий алгоритмов поиска маршрутов из точки \( A \) 
в точку \( B \), но для применения на графе дорог с учётом рельефа и других специфичных городских 
препятствий. Используемый алгоритм должен быть оптимален по времени работы и требуемой памяти для частого 
построения маршрутов.

Четвёртая задача заключается в разработке критериев по которым можно будет оценить качество построенного 
маршрута. Также предоставить данную информацию пользователю и модуля построения маршрутов для последующей 
оптимизации.

Пятая задача заключается в разработке web-приложения для визуализации построенных маршрутов по предыдущим 
пунктам.

В первой главе произведён анализ рассматриваемой предметной области, рассмотрено текущее состояние сложившейся
проблемы и существующие методы, описанные в работах других исследователей, для решения данной ситуации. 

Во второй главе рассмотрены и проанализированы существующие методы применимые к задаче формирования 
маршрутов, а также разработаны методы на их основе для формирования маршрутных сетей используя 
кластеризованные данные о предпочтения по перемещению в городе.

В третье глава описана методология проектирования ПО, разработана методика проведения эксперимента, 
произведено испытание разработанных алгоритмов, а также обсуждены полученные результаты в ходе эксперимента 
и сделан вывод на их основе.

В заключении работы сформулированы общие выводы по проделанной работе.