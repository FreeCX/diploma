\part{Введение}
Быстрые изменения происходящие в городской среде, как следствие технического прогресса требуют формирования 
новых методов в планирования инфраструктуры города для организации комфортной жизни людей. Это относится, в 
том числе, и к организации транспортной инфраструктуры, в частности к построению маршрутов общественного 
транспорта. Несмотря на кажущуюся хаотичность все перемещения пассажиров, подчиняются определенным 
закономерностям, связанным с масштабом и планировкой городской среды. Для принятия обоснованных решений по 
планированию или изменению маршрутной сети города необходимо выявить закономерности поведения населения и 
сформировать обобщенную модель, на основе которой можно строить и оценивать варианты транспортной системы. 
В связи с этим, можно сформулировать научную проблему, связанную с совершенствованием маршрутной сети 
пассажирского транспорта на основе методов обработки больших данных о предпочтениях жителей по перемещению. 

\emph{Актуальность}. Изменения в городской среде требуют формирования новых механизмов планирования 
инфраструктуры города. Для получения эффективных результатов, следует осуществлять принятие решений на основе 
актуальных данных, отражающих предпочтения жителей. В рамках магистерской диссертации следует разработать 
метод построения маршрутов общественного транспорта на основе предпочтений жителей.

\emph{Объект и предмет исследования}. Объектом исследования магистерской работы является построения маршрутов 
общественного транспорта на основе актуальных данных о предпочтениях жителей по перемещениям в современной 
городской среде. Предметом исследования является разработка и применение методов методы построения маршрутов 
общественного транспорта учитывающие актуальные данные о предпочтениях жителей по перемещению и интенсивности 
пассажиропотоков в городе.

\emph{Цель и задачи работы} данной работы является разработка метода генерации маршрутов общественного 
транспорта на основе предпочтений жителей для минимизации дискомфорта перемещения в городе. Для достижения 
поставленной цели решались следующие задачи:
\begin{itemize}
    \item генерация псевдореалистичных данных кластеров предпочтений;
    \item разработка методов маршрутизации между кластерами предпочтений;
    \item модификация и использование существующих алгоритмов для задачи маршрутизации;
    \item разработка критериев оценки качества построенных маршрутов;
    \item представление построенных маршрутов на карте;
\end{itemize}

Первая задача заключается в создании псевдореалистичных данных для замены отсутствующий реальных на данный 
момент. Они нужны для работы над последующими задачами как некий приближенный аналог.

Вторая задача заключается в разработке метода обхода кластеров предпочтений, который используя информацию о 
пассажиропотоках генерирует оптимальный список обхода кластеров. В данной работе используется разработанный 
метод минимального увеличения длины маршрута, основанный на идее, что идеальный маршрут должен минимально 
отличаться по длине от проложенного для автомобиля.

Третья задача заключается в анализе и модификации существующий алгоритмов поиска маршрутов из точки \( A \) 
в точку \( B \), но для применения на графе дорог с учётом рельефа и других специфичных городских 
препятствий. Используемый алгоритм должен быть оптимален по времени работы и требуемой памяти для частого 
построения маршрутов.

Четвёртая задача заключается в разработке критериев по которым можно будет оценить качество построенного 
маршрута. Также предоставить данную информацию пользователю и модуля построения маршрутов для последующей 
оптимизации.

Пятая задача заключается в разработке web-приложения для визуализации построенных маршрутов по предыдущим 
пунктам.

В первой главе приводятся результаты исследования предметной области. Произведён анализ общего состояние 
существующей проблемы в транспортной инфраструктуре. Описаны существующие программные продукты частично 
решающие данную проблему в полуавтоматическом режиме, но требующие вмешательства транспортного инженера для 
проектирования транспортной сети. Также рассмотрена литература по современным исследованиям в данной области 
и методам предлагаемых в них. В результате сформирован подход органично вписывающийся в существующую систему 
построения для решения поставленной задачи.

Во второй главе рассмотрены и проанализированы существующие методы применимые к задаче формирования 
маршрутов, а также разработаны методы на их основе для формирования маршрутных сетей используя 
кластеризованные данные о предпочтения по перемещению в городе.

В третье главе описана методология проектирования ПО, разработана методика проведения эксперимента, 
произведено испытание разработанных алгоритмов, а также обсуждены полученные результаты в ходе эксперимента и 
сделан вывод на их основе. Предложенные алгоритмы были реализованы с использованием языка программирования 
Python и сервиса построения маршрутов Open Source Routing Machine (OSRM) для расчёта расстояния между узлами 
графа по городским дорогам.

В четвёртой главе произведена оценка эффективности разработанных алгоритмов, были проведены эксперименты в 
ходе которых менялось количество узлов в дорожном графе и количество создаваемых маршрутов в городской сети, 
а также несколько вариантов реализации данного метода. Наиболее интересным случаем является, когда 
обрабатывается большое число узлов в графе или большое число геопространственных данных.

В заключении работы сформулированы общие выводы по проделанной работе в рамках магистерской диссертации. 
Обсуждены полученные результаты и предложено направление для дальнейшего развития данной работы.