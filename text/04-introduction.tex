% желательно переписать
\part{Введение}

% !!! 2-3 страницы !!!
% Краткая характеристика (1-3 предложения)
% Актуальность
% Научная новизна
% Практическая значимость
% Предыдущие труды в данной области
% ! никакого обзорного материала
% в конце краткое описание разделов работы

% место для общего текста
% введение своими словами
Развитие городов сопряжено с модернизацией существующей сети общественного транспорта. Современные 
коммуникационные технологии позволяют собирать данные о перемещениях людей в городской среде, на основе 
которых можно сделать выводы о предпочтениях людей по перемещению внутри городской среды. Фактически, 
подобные предпочтения можно рассматривать как требования к структуре сети общественного транспорта. 
% ---
Быстрые изменения городской среды как следствие технического прогресса требуют формирования новых 
подходов планирования инфраструктуры города для организации комфортной жизни людей. Это относится, в том 
числе, и к организации транспортной инфраструктуры, в частности к построению маршрутов общественного 
транспорта. Следует отметить, что развитие транспортной инфраструктуры основывается на устаревших 
нормативах, не учитывающих стремительное увеличение личного транспорта и изменений функционального 
назначения городских пространств \cite{bib:1}. Это ведет к ухудшению транспортной ситуации, снижению 
качества транспортного обслуживания, увеличению пробок, и как следствие к усилению неудовлетворенности 
жителей. 

Критичным фактором является игнорирование фактических предпочтений жителей по перемещению по городу при 
проектировании маршрутов городского транспорта.

Для оптимизации маршрутной сети общественного транспорта необходимо проанализировать большой объем данных, 
характеризующих численность и мобильность населения, среднее время перемещения, расположение мест приложения 
труда и жилых массивов. Источниками этих данных выступают статистические сборники, выписки о численности 
сотрудников крупных предприятий, собираемые муниципальными предприятиями общественного транспорта, 
информация о количестве проданных билетах на маршрутах общественного транспорта. Для сбора данных о 
перемещениях жителей организуется целый комплекс мероприятий по натурному подсчету пассажиропотока в 
подвижном составе общественного транспорта и на остановочных пунктах существующих маршрутов, а также 
анкетированию жителей \cite{bib:2,bib:3}. Такие традиционные методы являются достаточно трудоемкими, а 
полученные данные не в полной мере отражают динамично меняющуюся ситуацию. В связи с этим, необходимо 
использовать современные технологии и новые ресурсы для получения актуальных данных о предпочтениях жителей 
по перемещениям в городе и интенсивности пассажиропотоков. Основываясь на современных подходах к анализу 
данных можно получить ценную информацию для поддержки принятия решений в процессе планирования развития 
транспортной системы города. 

Несмотря на кажущуюся хаотичность все перемещения пассажиров, подчиняются определенным закономерностям, 
связанным с масштабом и планировкой городской среды. Для принятия обоснованных решений по планированию или 
изменению маршрутной сети города необходимо выявить закономерности поведения населения и сформировать 
обобщенную модель, на основе которой можно строить и оценивать варианты транспортной системы. В связи с 
этим, можно сформулировать научную проблему, связанную с совершенствованием маршрутной сети пассажирского 
транспорта  на основе методов обработки больших данных о предпочтениях жителей по перемещению. 
% ---

\emph{Актуальность.} Изменения в городской среде требуют формирования новых механизмов 
планирования инфраструктуры города. Для получения эффективных результатов, следует осуществлять 
принятие решений на основе актуальных данных, отражающих предпочтения жителей. В рамках 
магистерской диссертации следует разработать метод построения маршрутов общественного транспорта 
на основе предпочтений жителей.

% -- смотри в доки --

% цель работы
В связи с этим целью данной работы являлось разработка метода генерации маршрутов общественного 
транспорта на основе предпочтений жителей для минимизации дискомфорта перемещения в городе.

\emph{Теоретический этап} -- рассмотрение информации по существующим алгоритмам, свободных систем 
маршрутизации; составление теоретической базы проекта и систематизация полученных знаний; 
составление технического задания. 

\emph{Практический этап} -- реализация системы на основе теоретической базы, составленной ранее с 
использование технического задания.

\emph{Финальный этап} -- внедрение готового продукта, получение обратной информации и исправление 
ошибок.

\emph{Теоретические задачи:}
\vspace*{-1em}
\begin{itemize}\itemsep-5pt
    \item разработка алгоритма формирования маршрутов на основе имеющихся данных;
    \item выбор критериев качества для оценки построенных маршрутов;
    \item методы построение маршрута по заданным критериям;
    \begin{itemize}\itemsep-5pt
        \item предпочтения жителей;
        \item длина маршрута;
        \item дискомфорт перемещения;
        \item и др.
    \end{itemize}
    \item разработка критериев для оценки качества построенного маршрута;
\end{itemize}
\emph{Практические задачи:}
\vspace*{-1em}
\begin{itemize}\itemsep-5pt
    \item разработка механизма генерация исходных данных;
    \item реализация разработанных алгоритмов и методов;
    \item отображение результатов генерации маршрутов на карте;
    \item оценка качества построенных маршрутов.
\end{itemize}

\chapter{Постановка задачи исследования}

\emph{Объект исследования} -- построения маршрутов общественного транспорта на основе актуальных 
данных о предпочтениях жителей по перемещениям в современной городской среде.

\emph{Предмет исследования} -- методы построения маршрутов общественного транспорта учитывающие 
актуальные данные о предпочтениях жителей по перемещению и интенсивности пассажиропотоков в городе.

В данной работе рассматриваются к решению следующие задачи:
\begin{enumerate}\itemsep-5pt
    \item генерация псевдореалистичных данных кластеров предпочтений;
    \item разработка метода маршрутизации между кластерами предпочтений;
    \item модификация и использование существующих алгоритмов для задачи маршрутизации;
    \item разработка критериев оценки качества построенных маршрутов;
    \item представление построенных маршрутов на карте;
\end{enumerate}

Первая задача заключается в создании псевдореалистичных данных для замены отсутствующий реальных 
на данный момент. Они нужны для работы над последующими задачами как некий приближенный аналог.

Вторая задача заключается в разработке метода обхода кластеров предпочтений, который используя 
информацию о пассажиропотоках генерирует оптимальный список обхода кластеров. Метод будет 
основывается на эволюционных алгоритмах. На текущем этапе используется алгоритм имитации отжига 
для генерации субоптимальных списков обхода. В дальнейшем планируется задействовать алгоритм 
табу поиска и возможно разработка нового метода на их основе.

Третья задача заключается в анализе и модификации существующий алгоритмов поиска маршрутов из 
точки \( A \) в точку \( B \), но для применения на графе дорог с учётом рельефа и других 
специфичных городских препятствий. Используемый алгоритм должен быть оптимален по времени работы 
и требуемой памяти для частого построения маршрутов.

Четвёртая задача заключается в разработке критериев по которым можно будет оценить качество 
построенного маршрута. Также предоставить данную информацию пользователю и модуля построения 
маршрутов для последующей оптимизации.

Пятая задача заключается в разработке web-инструмента для отображения и редактирования построенных 
маршрутов по предыдущим пунктам.