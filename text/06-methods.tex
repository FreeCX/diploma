% переструктурировать и перерабоать
% мало подробностей
% больше методов построения маршрутов
% псевдокод!
\chapter{Методы построения маршрутов}

% ---
Сформулируем постановку задачи формирования маршрутов. Пусть заданы \( N_s \) – число узлов (остановочных 
пунктов), \( N_r \) – число маршрутов. Требуется построить \( N_r \) последовательностей узлов, при которых 
целевая функция качества маршрутной сети будет принимать максимальное значение. 

Задача тесно связана с задачей маршрутизацией <<из пункта А в пункт Б>>, но имеет некоторые отличия. 
Во-первых, в типовой задаче маршрутизации целевая функция – это время поездки от начала до конца маршрута, 
которое необходимо минимизировать. В случае с построением сети маршрутов общественного транспорта целевая 
функция -- интегральная, учитывающая средние показатели времени пешего хода до остановки, длины пути, 
количества пересадок, и пр. [16,17]. Во-вторых, в типовой задаче маршрутизации для движения выбираются 
любые промежуточные точки, сокращающие маршрут, тогда как в рассматриваемой задаче промежуточные точки 
расположены там же, где и центры кластеров.
% ---
Предлагается две стратегии генерации исходной сети маршрутов общественного транспорта. Первая предполагает 
генерацию одного <<длинного>> маршрута, обходящего все исходные узлы. Далее осуществляется разрез маршрута 
на \( N_r \) маршрутов и осуществляется модификация сети в соответствии с эволюционным алгоритмом. Вторая 
стратегия: формирование начальной сети, число маршрутов в которой соответствует заданному значению \( N_r \) 
и применение эволюционных алгоритмов для поиска сети маршрутов с минимальной длиной.
% ---

\section{\#1}

% ---
\textbf{Первая стратегия: формирование <<длинного>> маршрута с использованием <<жадного>> алгоритма}

Формально данная задача связана с поиском оптимального пути на графе. Были проанализированы различные и 
широко используемые алгоритмы построения пути: A*, IDA*, поиск в ширину (Breadth-First-Search), поиск 
<<лучший-первый>> (Best-First-Search), алгоритм Дейкстры, двунаправленный A*, Jump Point Search, 
Orthogonal Jump Point Search, двухэтапные алгоритмы (ALT, Reach), а так же генетические алгоритмы 
случайного поиска [18]. 

В качестве базового алгоритма выбран <<жадный>> алгоритм. В результате кластеризации исходных данных на 
предыдущем этапе получен полносвязанный граф, вершины которого -- центры кластеров (заданы географическими 
координатами и количеством точек отправления или назначения в кластере). Веса рёбер -- потребность в 
перевозке пассажиров между кластерами (транспортный спрос). Последовательность работы алгоритма 
описывается следующими шагами:
\begin{enumerate}
    \item 1) определяем ребро графа с максимальной потребностью в транспорте;
    \item 2) выбираем из двух вершин, которые это ребро соединяет, одну, с максимальным количеством 
        пассажиров в ней;
    \item 3) делаем первый шаг из этой вершины в другую, выбирая то ребро, которое имеет максимальную 
        потребность;
    \item 4) из этой вершины делаем шаг в следующую, тоже выбирая ребро с максимальной потребностью:
    \begin{itemize}
        \item если два ребра имеют одинаковую пропускную способность, то выбираем любое;
        \item если ребро имеется в списке, то выбираем следующее по пропускной способности;
    \end{itemize}
    \item 5) строим до тех пор, пока длина маршрута не превысит пороговое значение;
    \item 6) переходим к пункту 1, выбрав следующее по пропускной способности ребро.
\end{enumerate}

Пример работы алгоритма приведен на рисунке 3.
\begin{enumerate}
    \item 1. Выбираем ребро с потребностью перевозки 300.
    \item 2. Выбираем кластер с количеством точек 500.
    \item 3. Первый шаг: начинаем маршрут с ребра 300.
    \item 4. Второй шаг: выбираем ребро 200.
    \item 5. Третий шаг: выбираем ребро 100.
\end{enumerate}
% Рисунок 3: Пример реализации алгоритма a – исходный граф; b – первый построенный маршрут 
% (вершина начала пути выделена красным); c – второй построенный маршрут 
% (вершина начала пути выделена синим)
% нарисовать/найти pdf-ку для жадного алгоритма
% ---

\subsection{Общее описание}
\subsubsection{Схематическое представление}
\subsubsection{Идея метода}
\subsection{Подробное описание}

\section{\#2}

% ---
\textbf{Формирование начальной сети маршрутов, использующий принцип <<минимального>> увеличения длины 
маршрута при включении нового пункта}

Идея алгоритма заключается в итеративном добавлении в существующие маршруты узлы, минимально увеличивающие 
длину исходных маршрутов. Вход: \( N_r \) -- число маршрутов в сети, \( N_c \) –- число узлов 
(центров кластеров), \( C_t \) – множество терминальных узлов (определенных посредством построения 
окружности, содержащей все узлы), \( C_{nt} \) – множество нетерминальных узлов (сумма элементов множества 
терминальных и нетерминальных узлов равна числу центров кластера), матрица длин размером 
\( ||{C_{nt}} + {C_{t}}|| X ||{C_{nt}} + {C_{t}}|| \).

Выход: \( RN \) – список маршрутов в сети (каждый маршрут -- список узлов, порядок важен).  

Алгоритм был апробирован на различных тестовых данных для различных значений числа маршрутов. 
Результат работы алгоритма представлен на рисунке 4.
% Рисунок 4: Визуализация результата работы алгоритма формирования начальной сети маршрутов, использующий 
% принцип <<минимального>> увеличения длины маршрута при включении нового пункта.
% ---

\subsection{Общее описание}
\subsubsection{Схематическое представление}
\subsubsection{Идея метода}
\subsection{Подробное описание}

% 3 метод
% ---
\textbf{Модификация начального варианта сети маршрутов общественного транспорта с целью минимизации функции 
затрат и генерация вариантов альтернатив сетей}

Модификация начального варианта сети маршрутов общественного транспорта осуществляется с использованием идеи 
итерационного эволюционного преобразования исходного маршрута. Предлагается эволюционный алгоритм, 
использующий операции мутации и кроссовера для оптимизации длины сети маршрутов [19].
Эволюционный алгоритм представлен следующей последовательностью шагов. 
\begin{enumerate}
    \item 1. Если выбрана стратегия формирования единственного начального маршрута, то 
    \begin{enumerate}
        \item 1.1 Сформировать единственный маршрут жадным алгоритмом.
        \item 1.2. Разрезать маршрут на \( k \) маршрутов (где \( k \) -- изначально заданное число 
            маршрутов).
    \end{enumerate}
    \item 2. Если выбрана стратегия формирования начальной сети, то перейти на шаг 3.
    \item 3. Оценить качество сети маршрутов (с использованием критериев качества, например длины 
        маршрутной сети).
    \item 4. Применить операцию кроссовера и мутации для получения новой генерации транспортной сети [19].
    \begin{enumerate}
        \item 4.1. Оценить качество новой сети маршрутов.
        \item 4.2. Если новая популяция лучше предыдущей, сохранить ее.
        \item 4.3. Если новая популяция хуже предыдущей, отклонить. 
    \end{enumerate}
    \item 5. Повторить, пока не выполняется условие останова.
\end{enumerate}
% ---