\chapter{Введение}

\section{Актуальность работы}
Существенно возросший в последнее время интерес к исследованию вихревого 
состояния обусловлен широкими потенциальными возможностями применения 
сверхпроводников в современной микроэлектронике и энергетике, а также 
интересом к самой физике процессов происходящих в смешанном состоянии 
сверхпроводников. Развитие нанотехнологии и открытие новых сверхпроводящих 
соединений (в частности, высокотемпературных сверхпроводников) стимулировали 
новые теоретические и экспериментальные исследования смешанного состояния. 
Изучение строения и свойств вихревых структур необходимо для получения ряда 
основных характеристик смешанного состояния сверхпроводников, таких как 
критические магнитные поля, кривые намагничивания, транспортные 
характеристики.

На протяжении нескольких десятилетий изучение вихревого состояния неизменно 
привлекает внимание исследователей. Впервые смешанное состояние с неполным 
эффектом Мейсснера-Оксенфельда (фаза Шубникова) в сверхпроводниках, 
находящихся во внешнем магнитном поле, было обнаружено группой Л. В. 
Шубникова в 1937 году \cite{shubnikov}. В 1957 году А. А. Абрикосов, 
основываясь на теории Гинзбурга-Ландау \cite{ginzburg-landau}, показал, что в 
массивных сверхпроводниках второго рода внешнее магнитное поле проникает в 
сверхпроводник в виде нитей магнитного потока (вихрей Абрикосова)
\cite{abrikosov}. Каждая нить окружена вихревым током и несет один квант 
магнитного потока 
\( Ф_0 = hc/2e \simeq 2.07\cdot10^{-7} \text{Гс}\cdot\text{см}^2 \). Вихрь 
представляет собой топологическую особенность сверхпроводящего параметра 
порядка, вокруг которой циркуляция градиента фазы \( \phi \) параметра порядка 
\( \Psi \) отлична от нуля и кратна \( 2\pi \). Важной топологической 
характеристикой вихря является завихренность \( N \), определяемая циркуляцией 
градиента фазы \( \phi \) вдоль контура \( \mathcal{L} \) охватывающего 
особенность 
\begin{equation}
    \oint\limits_{\mathcal{L}} \nabla_\phi d\vec{l} = 2\pi N
\end{equation}

Фактически, именно существование ненулевой циркуляции фазы (завихренности) 
вокруг особой линии и является определением вихря.

Следствием существования такой особенности является обращение в ноль параметра 
порядка на оси вихря. Модуль параметра порядка в абрикосовском вихре в слабых 
полях аксиально симметричен. При удалении от центра вихря модуль параметра 
порядка растет и выходит на свое равновесное значение \( |\Psi|_\infty \) на 
расстоянии порядка длины когерентности \( \xi \) от центра вихря. Область 
размером порядка \( \xi \) где параметр порядка подавлен, называется кором 
вихря. Магнитное поле, индуцированное вихрем в массивном сверхпроводнике, 
максимально в области нормального кора и экспоненциально спадает при удалении 
от него на расстоянии порядка \( \lambda \) -- глубины проникновения магнитного 
поля в массивном сверхпроводнике. В массивном сверхпроводнике без дефектов 
энергетически выгодными являются одноквантовые вихри с \( N = 1 \) которые 
образуют гексагональную вихревую решетку.

Большое влияние на строение и различные свойства вихревых линий могут 
оказывать различные анизотропные факторы, присущие конкретной сверхпроводящей 
системе. В ряде случаев, под действием этих факторов в сверхпроводниках могут 
образовываться многоквантовые вихри (c \( |N|>1 \)) и связанные вихревые
состояния. Связанные вихревые состояния -- это совокупность вихрей, 
локализованных в малой области порядка размера самого вихря (то есть, 
несколько длин когерентности), которую можно рассматривать как структурную 
единицу смешанного состояния.

Задача о сосуществование сверхпроводящего и магнитного порядков привлекает 
внимание исследователей на протяжение последних десятилетий. Можно выделить 
два основных механизма взаимодействия сверхпроводящего параметра порядка с 
магнитной подсистемой: электромагнитный механизм, когда куперовские пары 
взаимодействуют с магнитным полем индуцированным ферромагнетиком (впервые 
такое взаимодействие было рассмотрено В. Л. Гинзбургом в 1956 
\cite{ginzburg}): и обменное взаимодействие магнитных моментов с куперовскими 
парами \cite{buzdin,bulaev}. Если ферромагнетик и сверхпроводник разделены 
тонкой диэлектрической прослойкой, то эффект близости подавлен и единственным 
фактором, определяющим взаимодействие подсистем, является магнитное поле, 
создаваемое неоднородным распределением намагниченности в ферромагнетике.

Исследование сверхпроводящих свойств систем сверхпроводник—ферромагнетик (SF 
систем) также привлекает большое внимание в связи с большим потенциалом их 
применения в современной электронике. В частности, такие структуры 
рассматриваются как кандидаты на создание систем с контролируемым пиннингом 
вихрей. Увеличение тока депиннинга наблюдалось экспериментально в 
сверхпроводящих пленках с ансамблем магнитных наночастиц [56, 57, 58, 59, 60], 
перфорированной магнитной пленкой (antidots) [61] и в двухслойных SF системах 
с доменной структурой ферромагнитной пленки [62]. Вихревые структуры и 
специфика пиннинга вихрей в SF структурах в слабых полях были рассмотрены в 
лондоновском приближении в работах 
[63, 64, 65, 66, 67, 68, 69, 70, 71, 72, 73, 74, 75].

\section{Методы исследования}

\section{Достоверность результатов}

Достоверность результатов обеспечена оптимальным выбором физических моделей, 
учитывающих основные свойства исследуемых систем, и адекватным выбором 
методов численного моделирования.

\section{Научная новизна}

\section{Научная и практическая значимость}