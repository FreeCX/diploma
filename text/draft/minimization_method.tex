Дополнительный материал для "Нетрадиционных состояний за счет не-попарно
межвихревого взаимодействия в многокомпонентных сверхпроводниках"

\section{Минимизация конечностных разностей энергий}

Для расчёт энергии межвихревого взаимодействия мы используем минимизацию 
конечностных разностей энергии.

Функционал свободной энергии ГЛ
\begin{equation}
    F = \frac{1}{2}\Sum\limits_{i=1,2}\left[ 
        \left|\left( \nabla + ie\vec{A}\right)\psi_i\right|^2 + 
        \left( 2\alpha_i + \beta_i |\psi_i|^2 \right)|\psi_i|^2 \right] + 
        \frac{1}{2}\left( \nabla\times\vec{A} \right)^2 - 
        \eta|\psi_1||\psi_2|\cos(\theta_2-\theta_1)
    \label{eq:1}
\end{equation}

Основные состояния вихревых систем и энергии взаимодействия между вихрями 
находятся с помощью минимизации этого функционала при условии соблюдения 
соответствующих ограничений, таких как расположение вихрей.

Для этого численно, мы дискретизации систему регулярной сеткой. Чтобы иметь 
объективный численный результат мы используем адаптивно-узловой сетки, где шаг 
сетки \( h \) во всей рассматриваемой области. Дискретизация гамильтониана 
производим конечно-разностным методом

Градиенты определяются как
\begin{equation}
    \left( \nabla f \right)_{i,i+1} = \frac{f(i+1)-f(i)}{h}
\end{equation}
и магнитный поток вычисляется путём интегрирования
\begin{equation}
    B_{i,i+1,j,j+1} = \frac{1}{h^2}\oint\limits_{\omega} \vec{A}\cdot d\vec{r}
\end{equation}
где \( \omega \) это квадрат с углами \( i, i+1, j, j+1 \). Плотность энергии 
в точке сетки \( (i,j) \) зависит от точки \( i, j \) и её соседей.

Схема оптимизации, которая используется в первой части работы -- измененная 
версия метода Ньютона-Рафсона. Есть 6 полей в задаче; действительные и мнимые 
части комплексных полей и двух компонент \( bfA \). Обозначим через \( P \). 
Оптимизация проводится следующим образом:
\begin{enumerate}
    \item Посчитать \( dP \)
        \begin{equation}
            dP_i = -\frac{\delta E}{\delta P_i} \Big/
                \frac{\delta^2 E}{\delta P^2_i}
        \end{equation}
    \item Проверить, если \( dP \) меньше, т.е. если \( E(P+dP) < E(P) \).
\end{enumerate}
Если это так, то после обновления \( P \leftarrow P + dP \). Затем мы перейдем 
к другой точке сетки и пытаться обновить его. Как правило, мы находим, что 
отношение удачных попыток обновить сетку близка к 1. Таким образом, это 
отличается от обычной схемы Ньютона-Рафсона в двух направлениях. Во-первых, 
мы всегда проверяем, что обновление действительно понижает энергию. Во-вторых, 
мы не вычисляем полный матрицу \( \delta^2 E / \delta P_i \delta P_j \), только 
диагональные элементы.

Чтобы свести к минимуму краевые эффекты мы используем свободные граничные 
условия. Вихри вставляются с использованием различных начальных конфигураций.

Это эффективно предотвращает движения вихря, но не препятствует основного 
расщепление \( |\psi_1| \) и \( |\psi_2| \) за счет магнитного давления.

Для того чтобы вычислить энергию межвихревого взаимодействия, мы должны 
исправить положение вихрей. Фиксация позиции вихря требует особой осторожности, 
чтобы избежать ситуации, когда закрепление на расчетной сетки существенно 
влияет на вихревое решение. Мы фиксируем положение вихря следующим образом. 
В центре вихря плотность condensate равна нулю. Затем фиксируем плотность 
только доминирующей центральной составляющей компоненты \( |\psi_i| \) вихря 
равным нулю в данной позиции расчетной сетки. Это эффективно предотвращает 
движения вихря, но не препятствует основного расщепление \( |\psi_1| \) и 
\( |\psi_2| \) за счет магнитного давления. Этот метод "точки закрепления" 
также имеет преимущество перед "миниинвазивным", так как только фиксируется 
положение ядра особенной точки. Таким образом она позволяет вычислить 
средние и дальнодействующие силы с наибольшей точностью. Тем не менее, в то 
же время, очевидно, этот способ не работает для слишком малого межвихревого 
расстояния. Слишком малое межвихревое расстояние приводит к следующему 
легкоузнаваемому артефакту: ядро вихря из одного вихря удлиняется до нуля в 
обоих центрах закрепления, позволяющих снять и убрать второй вихрь, в то 
время, удовлетворяющих минимизации энергии связи. Такое поведение может быть 
легко исправлено используя различные схемы закрепления, а потому, закрепления 
вихрей на малом расстоянии не имеет отношения к вопросам, изучаемых в данной 
работе, а также для обеспечения согласованности мы используем одну процедуру 
фиксации.

Межвихревое взаимодействие установленное с использованием этой схемы фиксации 
согласуется с формирования структуры в безусловной минимизации полученного 
двумя различными способами в второй части статьи.

Сходимость определяется следующим образом:
\begin{enumerate}
    \item Мы выбираем конкретный шаг сетки \( h_1 \) и число точек сетки 
        \( N_1 = N_{1x} \cdot N_{1y} \) даваемое размером системы
        \( L_x = h \cdot (N_{1x}-1) \), \( L_y = h \cdot (N_{1y}-1) \). 
        Тогда мы минимизируем энергию, пока она не измениться в несколько
        тысячах иттераций. Это даёт нам \( E(h_1) \).
    \item Мы уменьшаем шаг сетки \( h \) на коэффициент 2 или 3 при сохранении 
        размера системы \( L_x, L_y \) с помощью сплайн-интерполяции. 
        Затем мы ещё раз перебираем энергию, пока она не измениться в 
        нескольких тысячах итераций, давая нам \( E(h_2) \) и так далее. 
        Затем определяем сходимость с помощью формулы
\end{enumerate}
\begin{equation}
    \frac{E(h_n) - E(h_{n+1})}{E(h_n)} = C
\end{equation}

Мы используем сетки размером до \( N \approx 10^7 \) что дает очень высокую 
точность, обычно \( C < 10^{-4} \).

\section{Элементы минимизации конечностных разностей энергий}

Во второй части статьи мы используем минимизацию свободной энергии. Связанное 
состоянии вихревой конфигурации минимума энергии Гинзбурга-Ландау \eqref{eq:1}. 
Это означает, что функциональная минимизация \eqref{eq:1}, из соответствующего 
исходного состояния, описывающего несколько квантов магнитного потока, должно 
привести к связанному состоянию (если оно существует). Рассмотрим двумерную 
задачу \( \mathcal{F} = \int\limits_\Omega F \) определенной на ограниченной 
области \( \Omega \in \mathbb{R}^2 \), дополняемую свободными граничными 
условиями.

Одна проблема математически корректна, численный алгоритм оптимизации 
используется для решения вариационной нелинейной задачи (то есть, чтобы найти 
минимум \( \mathcal{F} \)). Мы использовали здесь Нелинейный метод Взаимных 
Градиентов, пока относительное изменения нормы градиента функционала 
\( \mathcal{F} \) по отношению ко всем степеням свободы меньше \( 10^{-6} \). 
Чтобы убедиться, что наши результаты не являются численными артефактами этого 
конкретной схемы минимизации, мы также провели стандартные расчеты 
наискорейшего спуска и это привело к аналогичным результатам.

Минимизация начинается с начального приближения: конфигурацию поля, несущего
\( N_v \) квантов потока, описываемого
\begin{gather}
    \Phi_a = u_a \prod\limits_{i=1}^{N_v}
        \sqrt{\frac{1}{2}\left( 1 + tanh\left( \frac{4}{\xi} 
        \left( \mathcal{R}_i(x,y) - \xi \right)\right)}e^{i\Theta_i}, 
    \nonumber \\
    \vec{A} = \frac{1}{e\mathcal{R}}\left( sin\Theta, -\cos\Theta \right)
    \label{eq:6}
\end{gather}
где \( a = 1,2, u_a \) является вакуумное среднее каждого скалярного поля, 
параметр \( \xi \) даёт размер ядра, пока \( \Theta \) и
\( \mathcal{R} \) являются
\begin{gather}
    \Theta(x,y) = \Sum\limits_{i=1}^{N_v} \Theta_i(x,y), \nonumber \\
    \Theta_i(x,y) = \tan^{-1}\left(\frac{y-y_i}{x-x_i} \right), \nonumber \\
    \mathcal{R}(x,y) = \Sum\limits_{i=1}^{N_v} \mathcal{R}_i(x,y), \nonumber \\
    \mathcal{R}_i(x,y) = \sqrt{(x-x_i)^2+(y-y_i)^2}.
\end{gather}
\( (x_i,y_i) \) является начальным положение данного вихря. Тогда, все степени 
свободы находятся в расслабленном состоянии одновременно без каких-либо 
ограничений для получения высокоточного решения уравнений Гинзбурга-Ландау.