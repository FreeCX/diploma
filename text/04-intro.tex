\chapter*{Введение}
\addcontentsline{toc}{chapter}{Введение}

В последнее время наблюдается повышенный интерес к сверхпроводникам с 
несколькими сверхпроводящими компонентами. Основные ситуации, когда возникают 
множественные сверхпроводящие компоненты: многозонные сверхпроводники второго 
рода \cite{bib:6,bib:7,bib:8,bib:9,bib:10,bib:11}, сверхпроводимость первого 
рода в смеси независимых консервативных конденсатов, таких как прогнозируемая 
сверхпроводимость в атомарном водороде и богатых водородом сплавов 
\cite{bib:12.1,bib:12.2,bib:13,bib:14} и сверхпроводников с другим типом 
симметрии, отличной от поперечно-волновой симметрии. Интерес к исследованию 
вихревого состояния обусловлен широкими потенциальными возможностями 
применения сверхпроводников в современной микроэлектронике и энергетике, а 
также интересом к самой физике процессов происходящих в смешанном состоянии 
сверхпроводников. Развитие нанотехнологии и открытие новых сверхпроводящих 
соединений (в частности, высокотемпературных сверхпроводников) стимулировали 
новые теоретические и экспериментальные исследования смешанного состояния. 

В работе проведено исследование появления сверхпроводимости 1,5-го рода с 
акцентом на случай многополосной сверхпроводимости, демонстрируя сохранение 
этого типа сверхпроводимости в присутствии различных видов межкомпонентных 
соединений (например, межзонное джозефсоновской связи, смешанных градиентных 
связей, связи типа плотность-плотность, и других видов).

\textbf{Актуальность.}

Главным фактором в изучении явления сверхпроводимости является временные 
затраты на проведение модельных экспериментов. Поэтому для решения 
поставленных задач нужно использовать более быстрые и эффективные алгоритмы.

\textbf{Методы исследования.}

Достоверность результатов обеспечена оптимальным выбором физических моделей, 
учитывающих основные свойства исследуемых систем, и адекватным выбором метода 
численного моделирования.

\textbf{Научная новизна.}

С целью использования градиентных алгоритмов минимизации (типа LBFGS), для 
которых знание градиента минимизируемой функции в явном виде является 
желательным условием, был использован метод автоматического дифференцирования 
(библиотека CppAD). Для минимизации функционала свободной энергии в модели 
Гинзбурга -- Ландау впервые был использован алгоритм LBFGS.

\textbf{Научная и практическая значимость.}

Используемый алгоритм Бройдена -- Флетчера -- Гольдфарба -- Шанно сокращает
временные затраты на минимизацию функционала Гинзбурга -- Ландау по сравнению 
с другими методами.

\newpage