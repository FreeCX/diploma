\chapter{Введение}

Согласно теории Гинзбурга-Ландау обычно сверхпроводник вблизи \( T_c \) 
описывается одним комплексным параметром поля. Физика этих систем определяется 
двумя фундаментальными масштабными длинами, глубиной проникновения магнитного 
поля \( \lambda \) и длиной когерентности \( \xi \), а также коэффициентом 
\( \kappa \), который определяет реакцию на внешнее поле, разделяя их на две  
категории следующим образом; сверхпроводники первого рода, где 
\( \kappa < 1/\sqrt{2} \) и второго рода, где \( \kappa > 1/\sqrt{2} \) 
\cite{bib:3}.

Сверхпроводники первого рода исключают слабые магнитные поля, в то время как 
сильные поля порождают формирования макроскопически нормальных областей с 
магнитным потоком \cite{bib:4}. Реакция сверхпроводников второго рода 
совершенно иная; ниже некоторого критического значения \( H_{c1} \), поле 
выталкивается. Выше этого значения сверхпроводник образует кристаллическую 
решетку или жидкостный вихрь, который имеет циркулирующее сверхпроводящее 
ядро проводящие магнитный поток через систему (!). И, наконец, при значении 
больше второго критического, \( H_{c2} \) сверхпроводимость разрушается. Эти 
различные реакции, как правило, рассматривается как последствия взаимодействия 
вихрей в этих системах, расход энергии на границе между сверхпроводящем и 
нормальном состояниях и термодинамической устойчивости вихревых возбуждений(!). 
В сверхпроводниках второго рода расход энергии на границе между нормальной и 
сверхпроводящим состоянием является отрицательным, а взаимодействие между 
вихрями является отталкивающим\cite{bib:3}. Это приводит к образованию 
устойчивых вихревых решеток и жидкостей(!). В сверхпроводниках первого рода 
ситуация противоположная; вихрь притягивающего взаимодействия (что делает их 
неустойчивым относительно сжатия в один крупный вихрь), а граничная энергии 
между нормальным и сверхпроводящим состояниями положительная. С 
термодинамической точки зрения принципиальное отличие сверхпроводников первого 
рода от второго следующее: (i) В сверхпроводниках второго рода напряженность 
внешнего магнитного поля, необходимая, чтобы образование вихревых возбуждений 
было энергетически выгодными, нужно чтобы \( H_{c1} \) было меньше, чем 
термодинамическое значение магнитного поля \( H_{ct} \) (поле, энергетическая
плотность которого равна энергии конденсации сверхпроводника, т.е. области, в 
которой равномерное сверхпроводящее состояние становится термодинамически 
неустойчивым); (ii) В сверхпроводниках первого рода напряженность поля требует 
создание возбуждения вихря больше, чем критическое термодинамическое значение 
магнитного поля, т.е. вихри не могут образовываться. Можно выделить также 
специальный "нульмерный" пограничный случай, когда \( \kappa \) имеет 
критическое значение точно на границе первого/второго рода, что в самом общей 
модели ГЛ соответствует \( \kappa = 1/\sqrt{2} \). В этом случае вихри не 
взаимодействуют\cite{bib:5} в теории Гинзбурга-Ландау.

Вышеуказанные обстоятельства приводят к тому, что, в сильном внешнем магнитном 
поле, сверхпроводники первого рода обычно имеют тенденцию к минимизации 
энергии на границе между нормальной и сверхпроводящей фазой, что приводит к 
образованию крупных образований нормальной фазы, которые часто имеют слоистую 
структуру\cite{bib:4}. 

В последнее время наблюдается повышенный интерес к сверхпроводникам с 
несколькими сверхпроводящими компонентами. Основные ситуации, когда возникают 
множественные сверхпроводящие компоненты (i) многозонные сверхпроводники
\cite{bib:6,bib:7,bib:8,bib:9,bib:10,bib:11}, (ii) смеси независимых 
консервативных конденсатов, таких как прогнозируемая сверхпроводимость в 
атомарном водороде и богатых водородом сплавов \cite{bib:12,bib:13,bib:14} и 
(iii) сверхпроводников с другим типом симметрии, отличной от 
поперечно-волновой симметрии. В этой работе будет поставлен акцента на случаи 
(i) и (ii). Принципиальная разница между случаями (i) и (ii) является 
отсутствие межкомпонентной джозефсоновской связи в случае с (ii).

В двухщелевых сверхпроводниках (i) сверхпроводящие компоненты происходят из
электронного куперовского спаривания в различных зонах \cite{bib:6}. Поэтому 
эти конденсаты не могут априори быть быть независимо сохраняющимися.

В случае (ii) две сверхпроводящие компоненты были предсказаны, происходящие 
из электронного и протонного куперовского спаривания в атомарном водороде 
или богатых водородом сплавов. В прогнозируемом жидкометаллическом дейтерии и 
сплавов богатых дейтерием, была предсказано существование электронной 
проводимости на сверхвысоких давлениях с дейтронной конденсацией  
\cite{bib:12,bib:13,bib:14}. Поскольку электроны не могут быть преобразованы в 
протонов или дейтрон с независимо сохраняющимися конденсатами(!!), и, 
следовательно, в эффективной модели Джозефсона межкомпонентная связь запрещена 
на основаниях симметрии. Этот эффект в настоящее время является предметом 
возобновлённых экспериментальных исследований. Ожидается, что они возникают 
при высоких, но экспериментально доступных давлениях (\( \approx 400 \)~ГПа). 
Текущие статические эксперименты сжатия достигают давлений 
\( \approx 350 \)~ГПа с давлением порядка 1 ТПа будучи ожидаемым в камере с 
алмазными наковальнями в связи с недавними опытами получения ультра жёстких 
алмазов. Обсуждались похожие заряженные двухкомпонентные модели в контексте 
физики нейтронных звёзд, где они представляют сосуществующие протонную и 
\( \Sigma^\text{--} \)-гиперионную куперовскую пару внутри нейтронной звезды.
\cite{bib:15}. 

Это большое разнообразие систем вызывает необходимость истолковать и 
классифицировать возможные магнитные отклики многокомпонентных 
сверхпроводников. В различных источниках обсуждалось, что в многокомпонентных 
системах, где магнитный отклик гораздо сложнее, чем в обычных системах, и что 
разделение на сверхпроводники первого/второго рода не является достаточным для 
классификации. Скорее всего, в широком диапазоне параметров, как следствие 
существовании трех основных масштабных длины, есть отдельный сверхпроводящий 
режим, при котором вихри имеют дальнодействующее притяжение, близкодействующее 
отталкивающее взаимодействие и форму вихревых узлов, погруженные в областях 
двухкомпонентного эффекта Мейснера (!!). \cite{bib:1,bib:2}. Последние 
экспериментальные работы \cite{bib:16,bib:17} выдвинули предположение, что 
это состояние реализуется в двухщелевом материале \( MgB_2 \), что вызвало 
растущий интерес к этой теме. В частности были подняты вопросы по поводу того, 
что сверхпроводимости типа 1,5 (как это было названо Мощалковым в 
\cite{bib:16}) возможна даже в случае различных неисчезающих связей (например, 
внутренняя джозефсоновская связь, связь смешанных градиентов и т.д.) 
сверхпроводящими компонентами в разных диапазонах.\cite{bib:main}

В этой работе проведено исследование появления сверхпроводимости типа 1,5 с 
акцентом на случай многополосной сверхпроводимости, демонстрируя сохранение 
этого типа сверхпроводимости в присутствии различных видов межкомпонентных 
соединений (например, межзонное джозефсоновской связи, смешанных градиентных 
связей, плотность-плотность, и другие виды соединения).

\textbf{Актуальность работы.} Возросший в последнее время интерес к 
исследованию вихревого состояния обусловлен широкими потенциальными 
возможностями применения сверхпроводников в современной микроэлектронике и 
энергетике, а также интересом к самой физике процессов происходящих в 
смешанном состоянии сверхпроводников. Развитие нанотехнологии и открытие 
новых сверхпроводящих соединений (в частности, высокотемпературных 
сверхпроводников) стимулировали новые теоретические и экспериментальные 
исследования смешанного состояния. 

На протяжении нескольких десятилетий изучение вихревого состояния неизменно 
привлекает внимание исследователей. Впервые смешанное состояние с неполным 
эффектом Мейсснера-Оксенфельда (фаза Шубникова) в сверхпроводниках, 
находящихся во внешнем магнитном поле, было обнаружено группой Л. В. 
Шубникова в 1937 году \cite{shubnikov}. В 1957 году А. А. Абрикосов, 
основываясь на теории Гинзбурга-Ландау \cite{ginzburg-landau}, показал, что в 
массивных сверхпроводниках второго рода внешнее магнитное поле проникает в 
сверхпроводник в виде нитей магнитного потока (вихрей Абрикосова)
\cite{abrikosov}. Каждая нить окружена вихревым током и несет один квант 
магнитного потока 
\( Ф_0 = hc/2e \simeq 2.07\cdot10^{-7} \text{Гс}\cdot\text{см}^2 \). Вихрь 
представляет собой топологическую особенность сверхпроводящего параметра 
порядка, вокруг которой циркуляция градиента фазы \( \phi \) параметра порядка 
\( \Psi \) отлична от нуля и кратна \( 2\pi \). Важной топологической 
характеристикой вихря является завихренность \( N \), определяемая циркуляцией 
градиента фазы \( \phi \) вдоль контура \( \mathcal{L} \) охватывающего 
особенность 
\begin{equation}
    \oint\limits_{\mathcal{L}} \nabla_\phi d\vec{l} = 2\pi N
\end{equation}

Следствием существования такой особенности является обращение в ноль параметра 
порядка на оси вихря. Модуль параметра порядка в абрикосовском вихре в слабых 
полях аксиально симметричен. При удалении от центра вихря модуль параметра 
порядка растет и выходит на свое равновесное значение \( |\Psi|_\infty \) на 
расстоянии порядка длины когерентности \( \xi \) от центра вихря. Область 
размером порядка \( \xi \) где параметр порядка подавлен, называется кором 
вихря. Магнитное поле, индуцированное вихрем в массивном сверхпроводнике, 
максимально в области нормального кора и экспоненциально спадает при удалении 
от него на расстоянии порядка \( \lambda \) -- глубины проникновения магнитного 
поля в массивном сверхпроводнике. В массивном сверхпроводнике без дефектов 
энергетически выгодными являются одноквантовые вихри с \( N = 1 \) которые 
образуют гексагональную вихревую решетку.

Большое влияние на строение и различные свойства вихревых линий могут 
оказывать различные анизотропные факторы, присущие конкретной сверхпроводящей 
системе. В ряде случаев, под действием этих факторов в сверхпроводниках могут 
образовываться многоквантовые вихри (c \( |N|>1 \)) и связанные вихревые
состояния. Связанные вихревые состояния -- это совокупность вихрей, 
локализованных в малой области порядка размера самого вихря (то есть, 
несколько длин когерентности), которую можно рассматривать как структурную 
единицу смешанного состояния.

Задача о сосуществование сверхпроводящего и магнитного порядков привлекает 
внимание исследователей на протяжение последних десятилетий. Можно выделить 
два основных механизма взаимодействия сверхпроводящего параметра порядка с 
магнитной подсистемой: электромагнитный механизм, когда куперовские пары 
взаимодействуют с магнитным полем индуцированным ферромагнетиком (впервые 
такое взаимодействие было рассмотрено В. Л. Гинзбургом в 1956 
\cite{ginzburg}): и обменное взаимодействие магнитных моментов с куперовскими 
парами \cite{buzdin,bulaev}. Если ферромагнетик и сверхпроводник разделены 
тонкой диэлектрической прослойкой, то эффект близости подавлен и единственным 
фактором, определяющим взаимодействие подсистем, является магнитное поле, 
создаваемое неоднородным распределением намагниченности в ферромагнетике.

\textbf{Методы исследования.}

\textbf{Достоверность результатов.}

Достоверность результатов обеспечена оптимальным выбором физических моделей, 
учитывающих основные свойства исследуемых систем, и адекватным выбором 
методов численного моделирования.

\textbf{Научная новизна.}

\textbf{Научная и практическая значимость.}

\newpage