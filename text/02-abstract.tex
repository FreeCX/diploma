\begin{center}
	РЕФЕРАТ
\end{center}

Целью данной работы является моделирование структуры абрикосовских вихрей в 
сверхпроводниках полуторного рода. Это относительно новый тип сверхпроводимости 
обладающий свойствами сверхпроводников 1-го и 2-го рода. В процессе работы был 
проведён анализ математической модели, написана программа для численного 
моделирования и произведены модельные эксперименты при различных параметра 
уравнения Гинзбурга-Ландау.
\vspace*{1cm}

\noindent Ключевые слова: сверхпроводимость, вихри Абрикосова, метод БФГШ, 
квазиньютовноские методы, ТГЛ, теория БКШ, эффект Мейсснера
\vspace*{1cm}

\begin{center}
    ABSTRACT
\end{center}

The purpose of this work is to simulate the structure of Abrikosov vortices in
superconductors Type-1.5. This relatively new type of superconductivity
having the properties of superconductors of the type-I and type-II. In the 
process, was an analysis of the mathematical model, written program for 
numerical modeling simulations and experiments performed under different 
parameters Ginzburg-Landau equation.

\vspace*{1cm}

\noindent Key words: superconductivity, Abrikosov vortices, BFGS, quasi-Newton 
methods, TGL, BCS theory, Meissner effect

\newpage