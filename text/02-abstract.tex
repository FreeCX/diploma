% переработать тексты
\tocless\part{Аннотация}
Развитие городов сопряжено с модернизацией существующей сети общественного транспорта. Современные 
коммуникационные технологии позволяют собирать данные о перемещениях людей в городской среде, на основе 
которых можно сделать выводы о предпочтениях людей по перемещению внутри городской среды. Фактически, 
подобные предпочтения можно рассматривать как требования к структуре сети общественного транспорта. 
Имея данные о ежедневных передвижениях людей, мы можем понять реальные предпочтения и потребности людей 
в системе городского транспорта. Следовательно, может быть предложена модификация транспортной сети или 
даже набор возможных альтернатив. Эта модификация отражает реальные потребности людей и сокращает время 
передвижения и повышает общий уровень удовлетворенности. Однако для решением задача анализа 
геопространственных данных необходимо получить набор субоптимальных маршрутов для выбора. Получение 
оптимальных маршрутов сети является повторяющейся процедурой, которая требует вмешательство эксперта.
В данной работе предлагаются методы, позволяющий на основе данных о предпочтениях пользователей 
общественного транспорта формировать схему маршрутов общественного транспорта.

Список ключевых слов: геоданные, транспортная сеть, транспорт, генетические алгоритмы, 
алгоритмы оптимизации, итерационные алгоритмы, \ldots

\tocless\part{Abstract}
Urban development is connected with the existing public transportation network modernization. Modern 
communication technologies make it possible to collect data on the movements of people in the urban 
environment, based on which is possible to draw conclusions about people's preferences on the movement 
inside the urban space. In fact, these preferences may be considered as requirements for the public 
transportation network. Having data about people's everyday movements, we can understand the real 
people preferences and needs in the urban transport system. Hence, as the results of analysis the 
modified transport network (or even a bunch of alternative) can be suggested. This new solutions reflects 
real people needs and reduce transfer time and increase satisfaction level. However the problem of 
geospatial data analysis is need to be solved to get (sub)optimal routes for choosing by authorities. 
Getting optimal routes network is an iterative procedure which requires human (expert) intervention.
The proposed methods allows to create a public transportation routes scheme based on data about the 
public transportation users preferences.

List of keywords: geodata, transport network, transport, genetic algorithms, optimization algorithms,
iterative algorithms, \ldots