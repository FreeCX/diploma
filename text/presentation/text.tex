\title{Моделирование структуры абрикосовских вихрей в сверхпроводниках 
    типа 1,5}
\author{Голубев А.В.}
\institute{ВолгГТУ}
\date{}

\begin{frame}
    \titlepage
\end{frame}

\begin{frame}
    \frametitle{Сверхпроводимость 1-го рода}
\end{frame}

\begin{frame}
    \frametitle{Сверхпроводимость 2-го рода}
\end{frame}

\begin{frame}
    \frametitle{Сверхпроводимость 1,5-го рода}
\end{frame}

\begin{frame}
    \frametitle{Двухкомпонентная модель Гинзбурга-Ландау}
    \begin{gather}
        F = \frac{1}{2}(D\psi_1)(D\psi_1)^* + 
            \frac{1}{2}(D\psi_2)(D\psi_2)^* -
            \nu Re\left( (D\psi_1)(D\psi_2)^* \right) + \nonumber \\ +
            \frac{1}{2}\left(\nabla\times\vec{A}\right)^2 + F_p
    \end{gather}
    Здесь \( D = \nabla + ie\vec{A} \) и \( \psi_a = |\psi_a|e^{i\theta_a} \), 
    \( a = 1,2 \).
\end{frame}

\begin{frame}
    \frametitle{Моделирование структуры вихрей}
\end{frame}

\begin{frame}
    \frametitle{Результаты}
\end{frame}