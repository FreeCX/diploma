\APPENDIX{Приложение А}{Техническое задание}

\thispagestyle{empty}
\begin{center}
    Министерство образования и науки Российской Федерации \\
    Федеральное государственное бюджетное образовательное учреждение высшего образования\\
    <<Волгоградский государственный технический университет>>\\
    Факультет электроники и вычислительной техники\\
    Кафедра <<Системы автоматизированного проектирования и поискового конструирования>>
    \vspace{1em}
\end{center}
\begin{flushright}
    \begin{center}
        \hspace*{10.5em}Утверждаю
    \end{center}
    И.о.зав. кафедрой <<САПР~и~ПК>>\\
    \UNDER{\LINE{3cm}}{\TINY{(подпись)}}\quad\UNDER{Щербаков~М.~В.}{\TINY{(инициалы, фамилия)}}\\
    <<\underline{\hspace{2em}}>> \underline{\hspace{7.5em}} \the\year\ г.
\end{flushright}
\begin{center}
    Программа формирования маршрутов общественного транспорта на основании обработки данных\\
    ТЕХНИЧЕСКОЕ ЗАДАНИЕ\\
    \vspace{2em}
    ВРБ.40461806.10.27-\SPECIFICATION.11-91\\
    ЛИСТОВ \SPAGES
\end{center}
\vspace{5em}
\begin{minipage}[t]{0.6\textwidth}
    \vspace{4em}
    \begin{flushleft}
        Нормоконтролер\\
        Садовникова~Н.~П.\\
        <<\LINE{1.5em}>>\ \LINE{7em} \the\year\ г.
    \end{flushleft}
\end{minipage}
\begin{minipage}[t]{0.39\textwidth}
    \begin{flushleft}
        Научный руководитель\\
        \underline{М.~В.~Щербаков\hspace{6em}}\\
        <<\LINE{1.5em}>>\ \LINE{7em} \the\year\ г.\\
        Исполнитель\\
        Студент группы\\
        \underline{А.~В.~Голубев\hspace{7em}}\\
        <<\LINE{1.5em}>>\ \LINE{7em} \the\year\ г.\\
    \end{flushleft}
\end{minipage}
\vspace{\fill}
\begin{center}
    Волгоград \the\year\ г.
\end{center}
\newpage

% конец содержания для ПЗ
\stopcontents

\tocless\part{Аннотация}
\newpage

% содержание для ТЗ
\startcontents[sections]
\printcontents[sections]{ }{2}{\contentsname}
% обнуляем счётчик глав
\setcounter{chapter}{0}
% убираем автотабуляцию
\setlength\parindent{0pt}

% ТЗ ГОСТ 34
% http://www.rugost.com/index.php?option=com_content&view=article&id=108:34-4-8&catid=25&Itemid=62
% http://www.prj-exp.ru/gost/gost_34-602-89.php

\chapter{Общие сведения}
\section{Полное наименование системы и её условное обозначение}
Полное наименование системы -- <<Программа формирования маршрутов 
общественного транспорта на основании обработки данных>>.

Краткое наименование системы -- <<АС>>. В дальнейшем просто -- система.

\section{Шифр темы или шифр (номер) договора}
Тема работы: <<Разработка эволюционного алгоритма формирования маршрутов 
общественного транспорта на основании обработки данных>>

% написать
\section{Наименование предприятий (объединений) разработчика и заказчика (пользователя)}
Заказчик -- ...\\
Исполнитель -- студент группы САПР-2п1 Голубев Алексей

% написать
\section{Плановые сроки начала и окончания работы по созданию системы}
Начало разработки -- DD.MM.201Y г.\\
Окончание разработки -- DD.MM.2016 г.

\section{Порядок оформления и предъявления заказчику результатов работ по созданию системы}

\chapter{Назначение и цели создания (развития) Системы}
\section{Назначение Системы}
Система предназначена для формирования начальной маршрутной сети общественного транспорта на основе 
кластеризованных данных о перемещении.

\section{Цели создания Системы}
Автоматизация процесса построения маршрутной сети общественного транспорта.

\chapter{Характеристика объекта автоматизации}
\section{Краткие сведения об объекте автоматизации или ссылки на документы, содержащие такую информацию}
В ходе проведения работ по разработке Системы автоматизируются процессы построения маршрутов общественного 
транспорта. Система будет эксплуатироваться на персональном компьютере по выбору Исполнителя.

\section{Сведения об условиях эксплуатации объекта автоматизации}
Система должна быть рассчитана на эксплуатацию в составе программно–технического комплекса и учитывать 
разделение ИТ инфраструктуры на внутреннюю и внешнюю. Для нормальной эксплуатации разрабатываемой системы 
должно быть обеспечено бесперебойное питание ПЭВМ. При эксплуатации система должна быть обеспечена 
соответствующая стандартам хранения носителей и эксплуатации ПЭВМ температура и влажность воздуха. 
Периодическое техническое обслуживание используемых технических средств должно проводиться в соответствии 
с требованиями технической документации изготовителей, но не реже одного раза в год. В процессе проведения 
периодического технического обслуживания должны проводиться внешний и внутренний осмотр и чистка 
технических средств, проверка контактных соединений, проверка параметров настроек работоспособности 
технических средств и тестирование их взаимодействия.

Объектом автоматизации является формирование маршрутов общественного транспорта.

\chapter{Требования к системе}
\section{Требования к структуре и функционированию системы}
В состав Системы должны входить следующие подсистемы
\begin{enumerate}
    \item Подсистема построения и расчёта длины маршрута;
    \item Подсистема генерации маршрутной сети;
    \item Подсистема сохранения, загрузки и преобразования данных.
\end{enumerate}

\subsection{перечень подсистем, их назначение и основные характеристики, требования к числу уровней 
    иерархии и степени централизации системы}
Подсистема построения и расчёта длины маршрута предназначена для:
\begin{itemize}
    \item взаимодействия с программных обеспечение OSRM;
    \item построения маршрута проходящего через контрольные точки;
    \item расчёта длины построенного маршрута.
\end{itemize}

Подсистема генерации маршрутной сети предназначена для:
\begin{itemize}
    \item инициализации маршрутной сети;
    \item нахождения терминальных кластеров;
    \item построения маршрутной сети по заданной метрике.
\end{itemize}

Подсистема сохранения, загрузки и преобразования данных предназначена для:
\begin{itemize}
    \item загрузки кластеризованных данных о перемещении;
    \item преобразованию загруженных данных;
    \item сохранения расчётных данных для последующей обработки.
\end{itemize}

\section{Требования к режимам функционирования системы}
Для АС определены следующие режимы функционирования:
\begin{itemize}
    \item Нормальный режим функционирования;
\end{itemize}
В нормальном режиме функционирования системы:
\begin{itemize}
    \item клиентское программное обеспечение и технические средства пользователей системы 
        обеспечивают возможность круглосуточного функционирования;
    \item исправная работа оборудования, составляющее комплекс технических средств;
    \item исправное функционирует системных и прикладного программного обеспечение системы.
\end{itemize}
Для обеспечения нормального режима функционирования системы необходимо выполнять требования и 
выдерживать условия эксплуатации программного обеспечения и комплекса технических средств системы, 
указанные в соответствующих технических документах (техническая документация и т.д.).

\section{Требования по диагностированию системы}
Система должна удовлетворять следующим требованиям по диагностированию:
\begin{itemize}
    \item запись при возникновении системных ошибок в ходе выполнения работы в системный журнал;
    \item журналирование работы подсистем;
    \item выдача пользователю сообщений, содержащих адекватное описание нарушения 
        работоспособности.
\end{itemize}

\section{Перспективы развития, модернизации системы}
Для приведения Системы к готовности для эксплуатации по результатам работы могут быть 
проведены работы в следующих направлениях:
\begin{itemize}
    \item масштабируемость системы за счёт вынесения функций в отдельные модули с 
        последующей структуризацией;
    \item создания модификаций на основе системы (замена используемых метрик, логических 
        структур и т.п.);
    \item адаптации логики работы системы к изменениям в документах, регламентирующих 
        деятельность Заказчика.
\end{itemize}

\section{Требования к надежности}
Надежное (устойчивое) функционирование программы должно быть обеспечено выполнением совокупности 
организационно-технических мероприятий, перечень которых приведен ниже: 
\begin{itemize}
    \item использованием лицензионного программного обеспечения; 
    \item использованием нового программного обеспечения;
    \item использованием отказоустойчивого оборудования;
    \item соблюдение сохранности входных данных.
\end{itemize}

\section{Требования к эргономике и технической эстетике}
Требования к пользовательскому интерфейсу не специфицируются.

\section{Требования к эксплуатации, техническому обслуживанию, ремонту и хранению компонентов системы}
Требования к эксплуатации, техническому обслуживание, ремонту и хранению компонентов системы 
не предъявляются.

\section{Требования к защите информации от несанкционированного доступа}
Обеспечение требований по защите информации от несанкционированного доступа возлагается на систему 
безопасности операционной системы.

\section{Требования по сохранности информации при авариях}
При авариях не должна нарушаться целостность данных. 

Требования надежности работы в целом и сохранности информации во время аварии должны быть 
учтены при выборе аппаратного обеспечения и квалификации обслуживающего персонала.

\section{Требования к видам обеспечения}
\subsection{Информационное обеспечение системы}
Технические средства, обеспечивающие хранение информации, должны использовать современные 
технологии, позволяющие обеспечить повышенную надежность хранения данных и оперативную замену 
оборудования (распределенная избыточная запись/считывание данных; зеркалирование; независимые 
дисковые массивы; кластеризация). Подсистемы должны работать только со своими структурами данных.

\subsection{Программное обеспечение системы}
В Системе должны максимально использоваться программные продукты с открытой лицензией. 

Используемые программные продукты:
\begin{itemize}
    \item Python 3 -- высокоуровневый язык программирования общего назначения, ориентированный на повышение 
        производительности разработчика и читаемости кода;
    \item JavaScript -- прототипно-ориентированный сценарный язык программирования;
    \item Leaflet -- библиотека для отображения карт на веб-сайтах;
    \item Geographiclib -- библиотека для работы с географическими данными;
    \item Polyline -- библиотека для декодирования геоинформационных данных;
    \item Requests -- библиотека для работы с HTTP-запросами;
    \item Numpy -- библиотека высокоуровневых математических функций;
    \item Geojson -- библиотека сериализации данных.
\end{itemize}

Реализация программных модулей должна соответствовать текущим требованиям оформления программного кода 
с открытой лицензией. Среди которых:
\begin{itemize}
    \item Форматирование кода в соответствии с PEP;
    \item Комментирование программного кода;
    \item Переносимость программного кода;
\end{itemize}

\subsection{Техническое обеспечение системы}
\subsubsection{Требования к клиентскому аппаратному обеспечению}
Система должна функционировать на аппаратном обеспечении, на котором может быть запущено 
клиентское программное обеспечение, но для достижения оптимальной производительности 
необходима конфигурация приведенная ниже:
\begin{itemize}
    \item Одноядерный процессор не менее 1 ГГц
    \item Не менее 512 Мб оперативной памяти
    \item По крайней мере, 100 Мб свободного места на диске
\end{itemize}

\subsection{Требования к математическому обеспечению}
При выборке и разработке моделей, методов и алгоритмов необходимо учитывать следующие требования:
\begin{itemize}
    \item Универсальность;
    \item Алгоритмическая надёжность.
\end{itemize}

Разделы математического обеспечения требуемые для реализации:
\begin{itemize}
    \item Линейная алгебра;
    \item Разделы тригонометрии и геометрии.
\end{itemize}

Разделы информатики требуемые для реализации:
\begin{itemize}
    \item Теория алгоритмов;
    \item Алгоритмы и структуры данных;
    \item Прикладная информатика;
    \item Компьютерное моделирование и численные методы;
    \item Распределенные вычисления;
    \item Интеллектуальны Анализ данных.
\end{itemize}

\subsubsection{Эксплуатационные требования}
Особых эксплуатационных требований к Системе не предъявляется.

% написать
\chapter{Состав и содержание работ по созданию (развитию) системы}
\emph{Этап 1.}\\
Сроки исполнения первого этапа: DD.MM.YYYY -- DD.MM.YYYY\\
На первом этапе будут проведены следующие работы:
\begin{itemize}
    \item work 1
    \item work 2
\end{itemize}

\emph{Этап 2.}\\
Сроки исполнения второго этапа: DD.MM.YYYY -- DD.MM.YYYY\\
На втором этапе будут проведены следующие работы:
\begin{itemize}
    \item work 1
    \item work 2
\end{itemize}

\chapter{Порядок контроля и приемки Системы}
\section{Состав, объем и методы испытаний системы и ее составных частей}
Первая версия Системы должна пройти предварительные испытания, состоящие из функционального 
тестирования. Будут проведены испытания работы модулей системы с целью сбора перечня 
предложений и выявления недостатков. 

\section{Общие требования к приемке работ}
В процессе приёмки работ должна быть осуществлена проверка Системы на соответствии требованиям 
настоящего <<Технического задания>>.

В процессе приёмочных испытаний должен вестись журнал, в котором будут фиксироваться результаты 
выполненных работ, замечания по работе программного обеспечения и предложения по изменению работы 
программного обеспечения.

По результатам испытаний возможны доработки и исправления. Выявленные в ПО и документации 
недостатки Исполнитель исправляет за свой счёт в специально оговоренные после проведения 
испытаний сроки.

\chapter{Требования к документированию}
\begin{enumerate}
    \item Пояснительная записка к техническому проекту;
\end{enumerate}

\setlength\parindent{15mm}
\stopcontents[sections]