\APPENDIX{Приложение А}{Техническое задание}

\thispagestyle{empty}
\begin{center}
    Министерство образования и науки Российской Федерации \\
    Федеральное государственное бюджетное образовательное учреждение высшего образования\\
    <<Волгоградский государственный технический университет>>\\
    Факультет электроники и вычислительной техники\\
    Кафедра <<Системы автоматизированного проектирования и поискового конструирования>>
    \vspace{1em}
\end{center}
\begin{flushright}
    \begin{center}
        \hspace*{10.5em}Утверждаю
    \end{center}
    И.о.зав. кафедрой <<САПР~и~ПК>>\\
    \UNDER{\LINE{3cm}}{\TINY{(подпись)}}\quad\UNDER{Щербаков~М.~В.}{\TINY{(инициалы, фамилия)}}\\
    <<\underline{\hspace{2em}}>> \underline{\hspace{7.5em}} \the\year\ г.
\end{flushright}
\begin{center}
    Программа формирования маршрутов общественного транспорта на основании обработки данных\\
    ТЕХНИЧЕСКОЕ ЗАДАНИЕ\\
    \vspace{2em}
    ВРБ.40461806.10.27-\SPECIFICATION.11-91\\
    ЛИСТОВ \SPAGES
\end{center}
\vspace{5em}
\begin{minipage}[t]{0.6\textwidth}
    \vspace{4em}
    \begin{flushleft}
        Нормоконтролер\\
        Садовникова~Н.~П.\\
        <<\LINE{1.5em}>>\ \LINE{7em} \the\year\ г.
    \end{flushleft}
\end{minipage}
\begin{minipage}[t]{0.39\textwidth}
    \begin{flushleft}
        Научный руководитель\\
        \underline{\smash{М.~В.~Щербаков\hspace{6em}}}\\
        <<\LINE{1.5em}>>\ \LINE{7em} \the\year\ г.\\
        Исполнитель\\
        Студент группы\\
        \underline{\smash{А.~В.~Голубев\hspace{7em}}}\\
        <<\LINE{1.5em}>>\ \LINE{7em} \the\year\ г.\\
    \end{flushleft}
\end{minipage}
\vspace{\fill}
\begin{center}
    Волгоград \the\year\ г.
\end{center}
\newpage

% конец содержания для ПЗ
\stopcontents

\tocless\part{Аннотация}
Документ содержит задание на разработку системы <<Программа формирования маршрутов общественного транспорта 
на основании обработки данных>> для автоматизации построения маршрутов общественного транспорта. В документе 
дано общее описание системы, её название, цель создания и назначение. Приведено описание основных требований 
предъявляемых к системе в целом и функциям системы, на основе которых выделены и описаны подсистемы. 
Разработана предварительная структура разрабатываемой системы. Описаны основные алгоритмы, которые будут 
использованы.
\newpage

% содержание для ТЗ
\startcontents[sections]
\printcontents[sections]{ }{2}{\contentsname}
% обнуляем счётчик глав
\setcounter{chapter}{0}
\begingroup

\makeatletter
\renewcommand\chapter{\par%
  \thispagestyle{plain}%
  \global\@topnum\z@
  \@afterindentfalse
  \secdef\@chapter\@schapter}
\makeatother

\newpage

% ТЗ ГОСТ 34
% http://www.rugost.com/index.php?option=com_content&view=article&id=108:34-4-8&catid=25&Itemid=62
% http://www.prj-exp.ru/gost/gost_34-602-89.php

\chapter{Общие сведения}
\section{Полное наименование системы и её условное обозначение}
Полное наименование системы -- <<Программа формирования маршрутов 
общественного транспорта на основании обработки данных>>.

Краткое наименование системы -- <<АС>>. В дальнейшем просто -- система.

\section{Шифр темы или шифр (номер) договора}
Тема работы: <<Разработка эволюционного алгоритма формирования маршрутов 
общественного транспорта на основании обработки данных>>

\section{Перечень документов, на основании которых создается система}
Основанием для проведения работ по созданию системы является задание приказ \ldots

\section{Наименование предприятий (объединений) разработчика и заказчика (пользователя)}
Заказчик -- Волгоградский государственный технический университет.\\
Исполнитель -- студент группы САПР-2п1 Голубев Алексей.

\section{Плановые сроки начала и окончания работы по созданию системы}
Начало разработки -- 01.10.2015 г.\\
Окончание разработки -- 31.05.2016 г.

\section{Порядок оформления и предъявления заказчику результатов работ по созданию системы}
Система передаётся в виде исходных кодов на языке Python в установленные сроки по созданию системы.
Приемка системы осуществляется комиссией в составе уполномоченных представителей Заказчика и Исполнителя. 
Порядок предъявления системы, ее испытаний и окончательной приемки определен в разделе \ref{sec:acceptance} 
настоящего ТЗ. Одновременно с предъявлением системы производится сдача разработанного Исполнителем комплекта 
документации согласно разделу \ref{sec:document} настоящего ТЗ.

\vspace{3em}
\chapter{Назначение и цели создания (развития) Системы}
\section{Назначение Системы}
Система предназначена для формирования начальной маршрутной сети общественного транспорта на основе 
кластеризованных данных о перемещении.

\section{Цели создания Системы}
Автоматизация процесса построения маршрутной сети общественного транспорта.

\vspace{3em}
\chapter{Характеристика объекта автоматизации}
\section{Краткие сведения об объекте автоматизации или ссылки на документы, содержащие такую информацию}
В ходе проведения работ по разработке Системы автоматизируются процессы построения маршрутов общественного 
транспорта. Система будет эксплуатироваться на персональном компьютере по выбору Исполнителя.

\section{Сведения об условиях эксплуатации объекта автоматизации}
Система должна быть рассчитана на эксплуатацию в составе программно– технического комплекса и учитывать 
разделение ИТ инфраструктуры на внутреннюю и внешнюю. Для нормальной эксплуатации разрабатываемой системы 
должно быть обеспечено бесперебойное питание ПЭВМ. При эксплуатации система должна быть обеспечена 
соответствующая стандартам хранения носителей и эксплуатации ПЭВМ температура и влажность воздуха. 
Периодическое техническое обслуживание используемых технических средств должно проводиться в соответствии 
с требованиями технической документации изготовителей, но не реже одного раза в год. В процессе проведения 
периодического технического обслуживания должны проводиться внешний и внутренний осмотр и чистка 
технических средств, проверка контактных соединений, проверка параметров настроек работоспособности 
технических средств и тестирование их взаимодействия.

Объектом автоматизации является формирование маршрутов общественного транспорта.

\vspace{3em}
\chapter{Требования к системе}
\section{Требования к системе в целом}
Общими требованиями к автоматизированной системе являются:
\begin{itemize}
    \item система должна иметь модульную архитектуру и высокую степень в масштабировании;
    \item система должна использовать методы параллельного программирования.
\end{itemize}

Основными принципами создания Системы являются:
\begin{itemize}
    \item использование современного программного и аппаратного обеспечения;
    \item высокая степень масштабирования программных и аппаратных средств.
\end{itemize}

\section{Требования к структуре и функционированию системы}
В состав Системы должны входить следующие подсистемы
\begin{enumerate}
    \item Подсистема построения и расчёта длины маршрута;
    \item Подсистема генерации маршрутной сети;
    \item Подсистема сохранения, загрузки и преобразования данных.
\end{enumerate}

\subsection{Перечень подсистем, их назначение и основные характеристики}
Подсистема построения и расчёта длины маршрута предназначена для:
\begin{itemize}
    \item взаимодействия с программных обеспечение OSRM;
    \item построения маршрута проходящего через контрольные точки;
    \item расчёта длины построенного маршрута.
\end{itemize}

Подсистема генерации маршрутной сети предназначена для:
\begin{itemize}
    \item инициализации маршрутной сети;
    \item нахождения терминальных кластеров;
    \item построения маршрутной сети по заданной метрике.
\end{itemize}

Подсистема сохранения, загрузки и преобразования данных предназначена для:
\begin{itemize}
    \item загрузки кластеризованных данных о перемещении;
    \item преобразованию загруженных данных;
    \item сохранения расчётных данных для последующей обработки.
\end{itemize}

\section{Требования к режимам функционирования системы}
Для АС определены следующие режимы функционирования:
\begin{itemize}
    \item Нормальный режим функционирования;
\end{itemize}
В нормальном режиме функционирования системы:
\begin{itemize}
    \item клиентское программное обеспечение и технические средства пользователей системы 
        обеспечивают возможность круглосуточного функционирования;
    \item исправная работа оборудования, составляющее комплекс технических средств;
    \item исправное функционирует системных и прикладного программного обеспечение системы.
\end{itemize}
Для обеспечения нормального режима функционирования системы необходимо выполнять требования и 
выдерживать условия эксплуатации программного обеспечения и комплекса технических средств системы, 
указанные в соответствующих технических документах (техническая документация и т.д.).

\section{Требования по диагностированию системы}
Система должна удовлетворять следующим требованиям по диагностированию:
\begin{itemize}
    \item запись при возникновении системных ошибок в ходе выполнения работы в системный журнал;
    \item журналирование работы подсистем;
    \item выдача пользователю сообщений, содержащих адекватное описание нарушения 
        работоспособности.
\end{itemize}

\section{Перспективы развития, модернизации системы}
Для приведения Системы к готовности для эксплуатации по результатам работы могут быть 
проведены работы в следующих направлениях:
\begin{itemize}
    \item масштабируемость системы за счёт вынесения функций в отдельные модули с 
        последующей структуризацией;
    \item создания модификаций на основе системы (замена используемых метрик, логических 
        структур и т.п.);
    \item адаптации логики работы системы к изменениям в документах, регламентирующих 
        деятельность Заказчика.
\end{itemize}

\section{Требования к надежности}
Надежное (устойчивое) функционирование программы должно быть обеспечено выполнением совокупности 
организационно-технических мероприятий, перечень которых приведен ниже: 
\begin{itemize}
    \item использованием лицензионного программного обеспечения; 
    \item использованием нового программного обеспечения;
    \item использованием отказоустойчивого оборудования;
    \item соблюдение сохранности входных данных.
\end{itemize}

\section{Требования к эргономике и технической эстетике}
Требования к пользовательскому интерфейсу не специфицируются.

\section{Требования к эксплуатации, техническому обслуживанию, ремонту и хранению компонентов системы}
Требования к эксплуатации, техническому обслуживание, ремонту и хранению компонентов системы 
не предъявляются.

\section{Требования к защите информации от несанкционированного доступа}
Обеспечение требований по защите информации от несанкционированного доступа возлагается на систему 
безопасности операционной системы.

\section{Требования по сохранности информации при авариях}
Для сохранности информации при авариях в качестве компонентов технических средств организацией должны 
использоваться только высококачественные комплектующие и технические средства с высоким значением времени 
наработки на отказ (до отказа).

Требования надежности работы в целом и сохранности информации во время аварии должны быть учтены при выборе 
аппаратного обеспечения и квалификации обслуживающего персонала.

\section{Требования к видам обеспечения}
\subsection{Информационное обеспечение системы}
Технические средства, обеспечивающие хранение информации, должны использовать современные 
технологии, позволяющие обеспечить повышенную надежность хранения данных и оперативную замену 
оборудования (распределенная избыточная запись/считывание данных; зеркалирование; независимые 
дисковые массивы; кластеризация). Подсистемы должны работать только со своими структурами данных.

\subsection{Программное обеспечение системы}
\label{sec:software}
В Системе должны максимально использоваться программные продукты с открытой лицензией. 

Используемые программные продукты:
\begin{itemize}
    \item Python 3 -- высокоуровневый язык программирования общего назначения, ориентированный на повышение 
        производительности разработчика и читаемости кода;
    \item JavaScript -- прототипно-ориентированный сценарный язык программирования;
    \item Leaflet -- библиотека для отображения карт на веб-сайтах;
    \item Geographiclib -- библиотека для работы с географическими данными;
    \item Polyline -- библиотека для декодирования геоинформационных данных;
    \item Requests -- библиотека для работы с HTTP-запросами;
    \item Numpy -- библиотека высокоуровневых математических функций;
    \item Geojson -- библиотека сериализации данных.
\end{itemize}

Реализация программных модулей должна соответствовать текущим требованиям оформления программного кода 
с открытой лицензией. Среди которых:
\begin{itemize}
    \item Форматирование кода в соответствии с PEP;
    \item Комментирование программного кода;
    \item Переносимость программного кода;
\end{itemize}

\subsection{Требования к лингвистическому обеспечению системы}
Для лингвистического обеспечения системы приводятся требования к применению в системе языков программирования 
высокого уровня, а также требования к кодированию и декодированию данных.

Должны выполняться следующие требования к кодированию и декодированию данных: UTF-8 для подсистемы хранения 
данных; UTF-8 информации, поступающей из систем-источников.

\subsection{Требования к языку программирования}
Языком программирования должен быть выбран -- Python 3. К среде разработки особых требований не предъявляется.

\subsection{Техническое обеспечение системы}
\subsubsection{Требования к программному обеспечению системы}
Программное обеспечение системы должно быть достаточным для выполнения всех реализуемых функций системы, а 
также иметь средства организации всех требуемых процессов обработки данных, позволяющих своевременно 
выполнять все автоматизируемые функции во всех регламентных режимах функционирования системы. Необходимый 
набор программных средств для работы системы должен быть следующим:
\begin{itemize}
    \item операционная система поддерживающая установку интерпретатора Python;
    \item интерпретатор языка Python не ниже 3;
    \item библиотеки из пункта \ref{sec:software}.
\end{itemize}

\subsection{Требования к техническому обеспечению системы}
Система должна функционировать на аппаратном обеспечении, на котором может быть запущено клиентское 
программное обеспечение, но для достижения оптимальной производительности необходима конфигурация приведенная 
ниже:
\begin{itemize}
    \item Многоядерный процессор не менее 2 ГГц;
    \item Не менее 4 Гб оперативной памяти;
    \item По крайней мере, 100 Мб свободного места на диске.
\end{itemize}

\subsection{Требования к математическому обеспечению}
При выборке и разработке моделей, методов и алгоритмов необходимо учитывать следующие требования:
\begin{itemize}
    \item Универсальность;
    \item Алгоритмическая надёжность.
\end{itemize}

Разделы математического обеспечения требуемые для реализации:
\begin{itemize}
    \item Линейная алгебра;
    \item Разделы тригонометрии и геометрии.
\end{itemize}

Разделы информатики требуемые для реализации:
\begin{itemize}
    \item Теория алгоритмов;
    \item Алгоритмы и структуры данных;
    \item Прикладная информатика;
    \item Компьютерное моделирование и численные методы;
    \item Распределенные вычисления;
    \item Интеллектуальны Анализ данных.
\end{itemize}

\subsubsection{Эксплуатационные требования}
Особых эксплуатационных требований к Системе не предъявляется.

\vspace{3em}
\chapter{Состав и содержание работ по созданию (развитию) системы}
Проектирование системы должно происходить следующим образом:
\begin{itemize}
    \item разработка, согласование и утверждение технического задания по ГОСТ 34.602-89 на 
        проектирование -- 3 недели;
    \item разработка рабочего проекта программы -- 10 недель.
\end{itemize}

Этап рабочего проекта разбивается на следующие под этапы:
\begin{itemize}
    \item разработка и отладка программы -- 4 недели;
    \item разработка программной документации -- 3 недели;
    \item испытания программы -- 3 недели.
\end{itemize}

Остальные сроки рассчитываются исходя из сроков утверждения ТЗ. Дата начала должна быть не позднее 
01.10.2015. Дата окончания работ не позднее 31.05.2016.

\vspace{3em}
\chapter{Порядок контроля и приемки Системы}
\label{sec:acceptance}
\section{Состав, объем и методы испытаний системы и ее составных частей}
Первая версия Системы должна пройти предварительные испытания, состоящие из функционального тестирования. 
Будут проведены испытания работы модулей системы с целью сбора перечня предложений и выявления недостатков. 

\section{Общие требования к приемке работ}
В процессе приёмки работ должна быть осуществлена проверка Системы на соответствии требованиям настоящего 
<<Технического задания>>.

В процессе приёмочных испытаний должен вестись журнал, в котором будут фиксироваться результаты выполненных 
работ, замечания по работе программного обеспечения и предложения по изменению работы программного 
обеспечения.

По результатам испытаний возможны доработки и исправления. Выявленные в ПО и документации недостатки 
Исполнитель исправляет за свой счёт в специально оговоренные после проведения испытаний сроки.

\vspace{3em}
\chapter{Требования к документированию}
\label{sec:document}
В состав программной документации, сопровождающей проектируемое изделие должны входить следующие документы:
\begin{enumerate}
    \item Пояснительная записка к техническому проекту;
    \item техническое задание;
\end{enumerate}

\endgroup
\stopcontents[sections]