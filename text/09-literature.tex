\addcontentsline{toc}{chapter}{Список используемой литературы}
\renewcommand{\bibname}{%
    \vspace{-2em}\begin{center}
        Список используемой литературы
    \end{center}\vspace{-2em}
}

\pagestyle{empty}

\begin{thebibliography}{10}
    \bibitem{bib:1} Evaluating the sustainability of Volgograd / N. Sadovnikova, D. Parygin, E. Gnedkova, 
        B. Sanzhapov, and N. Gidkova. // In The Sustainable City VIII. WIT Press, 2013.
    \bibitem{bib:2} Нестерова А. Новая маршрутная сеть г. Томска представлена общественности 
        [Электронный ресурс] // Сетевое издание Центр дорожной информации. -- 2015. -- Режим доступа: 
        \url{http://road.perm.ru/index.php?id=1475} (дата обращения: 01.12.2015).
    \bibitem{bib:3} Nielsen G., Lange T. Network Design for Public Transport Success -- theory and 
        examples // Norwegian Ministry of Transport and Communications, Oslo. -- 2008.
    \bibitem{bib:4} Holsapple, C. Decisions and Knowledge. Handbook on Decision Support Systems 1, 
        (Cosgrove). [Electronic Resource]. -- 2008. -- URL: 
        \url{http://www.springerlink.com/index/g182q711470w2510.pdf}.
    \bibitem{bib:5} Tennenhouse D. Proactive computing //
        Communications of the ACM.~--\\2000. -- Т. 43. -- № 5. -- С. 43-50.
    \bibitem{bib:6} PTV Visum [Электронный ресурс] // PTV Group. -- 2014. -- Режим доступа: 
        \url{http://vision-traffic.ptvgroup.com/en-uk/products/ptv-visum/} 
        (дата обращения: 16.11.2014).
    \bibitem{bib:7} INRO [Электронный ресурс] // Emme. -- 2014. -- Режим доступа:\\
        \url{https://www.inrosoftware.com/en/products/emme/} (дата обращения: 11.11.2014).
    \bibitem{bib:8} Cube [Электронный ресурс] // Citilabs. -- 2014. -- Режим доступа:\\
        \url{http://www.citilabs.com/software/cube/} (дата обращения: 23.11.2014).
    \bibitem{aimsun} Transport Simulation Systems [Электронный ресурс] // Aimsun. -- Режим доступа: 
        \url{https://www.aimsun.com/wp/} (дата обращения: 01.03.2016).
    \bibitem{transims} Transims [Электронный ресурс] // NASA. -- Режим доступа:\\
        \url{https://code.google.com/archive/p/transims/} (дата обращения: 01.03.2016).
    \bibitem{osrm} Open Source Routing Machine [Электронный ресурс]. -- 2015. -- Режим доступа: 
        \url{http://project-osrm.org/} (дата обращения: 25.11.2015).
    \bibitem{bib:9} Ceder A. Designing public transport network and routes //
        Advanced Modeling for Transit Operations and Service Planning. -- 2003. -- Т. 3. -- С. 59-91.
    \bibitem{bib:17} Sadovnikova N. et al. Models and Methods for the Urban Transit System Research //
        Creativity in Intelligent Technologies and Data Science. -- Springer International Publishing, 
        2015. -- С. 488-499. -- (Ser. Communications in Computer and Information Science. Vol. 535)
    \bibitem{bib:19} Гладков Л.А., Курейчик В.В., Курейчик В.М. Генетические алгоритмы / 
        Под ред. В.М. Курейчика. -- 2-е изд., испр. и доп. -- М.: ФИЗМАТЛИТ, 2006. -- 320 с.
    \bibitem{bib:20} Strategway: web solutions for building public transportation routes using big geodata 
        analysis / Golubev A., Chechetkin I., Solnushkin K.S., Sadovnikova N., Parygin D., Shcherbakov M., 
        Brebels A. // Proceedings of The 17th International Conference on Information Integration and 
        Web-based Applications \& Services (iiWAS2015) (December 11 - 13, 2015 Brussels, Belgium) 
        ACM New York, New York pp. 665 - 668
    \bibitem{bib:20.2} Комплекс инструментов интеллектуального анализа данных strategway для поддержки 
        принятия решений по управлению развитием инфраструктуры города / Садовникова Н.П., Щербаков М.В., 
        Парыгин Д.С., Солнушкин К.С., Голубев А.В., Чечёткин И.А. // В сборнике: Развитие средних 
        городов: замысел, модели, практика Материалы III Международной научно-практической конференции. 
        Волгоград, 2015. С. 147-150
    \bibitem{bib:20.3} Автоматизация поддержки принятия решений по разработке маршрутов общественного 
        транспорта на основе анализа данных о корреспонденциях жителей / М. В. Щербаков, 
        Н. П. Садовникова, Д. С. Парыгин, А. В. Голубев, И. А. Чечеткин // Вестник компьютерных и 
        информационных технологий. -- М. : Издательский дом <<Спектр>>, 2016. -- Принята к печати.

    %%% --- NEW ---
    % \bibitem{bib:20.4} Maxim Shcherbakov and Alexey Golubev, An algorithm for initial public transport network 
    %     design over geospatial data // 2016 IEEE International Smart Cities Conference (ISC2) (ISC2 2016) 
    %     (September, 2016 Trento, Italy)
    %%% --- NEW ---

    \bibitem{bib:21} Bast H. et al. Route planning in transportation networks //
        arXiv preprint arXiv:1504.05140. -- 2015.
    \bibitem{bib:22} Чалой Е. В., Шамрай Н. Б. Построение матрицы корреспонденций для транспортной 
        сети г. Владивостока.
    \bibitem{bib:23} Гасников А. и др. (ред.). Введение в математическое моделирование транспортных 
        потоков. -- Litres, 2015.
    \bibitem{bib:24} Гасников А. В., Гасникова Е. В. О возможной динамике в модели расчета матрицы 
        корреспонденций (А. Дж. Вильсона) // ТРУДЫ МФТИ. -- 2010. -- Т. 2. -- № 4. -- С. 45.
    \bibitem{bib:25} Werneck R. F. Public Transit Labeling // Experimental Algorithms: 
        14th International Symposium, SEA 2015, Paris, France, June 29–July 1, 2015, 
        Proceedings. -- Springer, 2015. -- Т. 9125. -- С. 273.
    \bibitem{ceder2007} Ceder A. Public Transit Planning and Operation: Theory, Modeling and Practice. 2007.
    \bibitem{rodeheffer2013symmetric} Rodeheffer T. L. The Symmetric Shortest-Path Table Routing 
        Conjecture. -- 2013.
    \bibitem{delling2014round} Delling D., Pajor T., Werneck R. F. Round-based public transit routing~//
        Transportation Science. -- 2014. -- Т. 49. -- № 3. -- С. 591-604.
    \bibitem{delling2015customizable} Delling D. et al. Customizable route planning in road networks //
        Transportation Science. -- 2015.
    \bibitem{delling2015public} Delling D. et al. Public transit labeling // Experimental Algorithms. -- 
        Springer International Publishing, 2015. -- С. 273-285.
    \bibitem{abraham2013alternative} Abraham I. et al. Alternative routes in road networks // Journal of 
        Experimental Algorithmics (JEA). -- 2013. -- Т. 18. -- С. 1.3.
    \bibitem{wei2012constructing} Wei L. Y., Zheng Y., Peng W. C. Constructing popular routes from 
        uncertain trajectories // Proceedings of the 18th ACM SIGKDD international conference on 
        Knowledge discovery and data mining. -- ACM, 2012. -- С. 195-203.
    \bibitem{dwyer2009fast} Dwyer T., Nachmanson L. Fast edge-routing for large graphs // 
        Graph Drawing. -- Springer Berlin Heidelberg, 2009. -- С. 147-158.
    \bibitem{bib:27} Wang Y., Zheng Y., Xue Y. Travel time estimation of a path using sparse 
        trajectories // Proceedings of the 20th ACM SIGKDD international conference on Knowledge 
        discovery and data mining. -- ACM, 2014. -- С. 25-34.
    \bibitem{bib:28} Berlingerio M. et al. AllAboard: a System for Exploring Urban Mobility and 
        Optimizing Public Transport Using Cellphone Data // Mobile Phone Data for Development, Analysis 
        of mobile phone datasets for the development of Ivory Coast, 
        viewed. -- 2014. -- Т. 9. -- С. 397-411.
    \bibitem{bib:29} Krushel E. G. et al. An Experience of Optimization Approach Application to Improve 
        the Urban Passenger Transport Structure // Knowledge-Based Software Engineering. -- Springer 
        International Publishing, 2014. -- С. 27-39.
    \bibitem{bib:30} Mees P. et al. Public transport network planning: a guide to best practice in NZ 
        cities. -- 2010. -- № 396.
    \bibitem{bib:31} Гузенко А. В. Развитие городского пассажирского транспорта мегаполиса: проблемы 
        и перспективы // Вестник Томского государственного университета. -- 2009. -- № 321.
    \bibitem{bib:32} Кузьмич С. И., Федина Т. О. Транспортные проблемы современных городов и 
        моделирование загрузки улично-дорожной сети // Известия Тульского государственного 
        университета. Технические науки. -- 2008. -- № 3.
    \bibitem{bib:34} Шуравина Е. Н. Проблемы современной транспортной системы россии // 
        Вестник Самарского государственного университета. -- 2011. -- № 90.
    \bibitem{bib:39} Хегай Ю. А. Проблемы автомобильного транспорта в россии // Теория и практика 
        общественного развития. -- 2014. -- № 8.
    \bibitem{bib:40} Синицына Е. Б., Лазарев Ю. Г. Современное состояние проблемы совершенствования 
        транспортной инфраструктуры // Технико-технологические проблемы сервиса. -- 2013. -- № 4 (26).
    \bibitem{bib:33} Андрианов В. Ю. Геоинформационные системы для транспорта и коммуникаций // 
        T-Comm-Телекоммуникации и Транспорт. -- 2010. -- № S2.
    \bibitem{bib:35} Корягин М. Е. Теоретические аспекты оптимизации управления движением городского 
        транспорта // Вестник Кузбасского государственного технического университета. -- 2012. -- № 1 (89).
    \bibitem{bib:36} Кочегурова Е. А., Мартынова Ю. А. Оптимизация составления маршрутов общественного 
        транспорта при создании автоматизированной системы поддержки принятия решений // 
        Известия Томского политехнического университета. -- 2013. -- Т. 323. -- № 5.
    \bibitem{bib:37} Агуреев И. Е., Митюгин В. А., Пышный В. А. Подготовка и обработка исходных данных 
        для математического моделирования автомобильных транспортных систем // Известия Тульского 
        государственного университета. Технические науки. -- 2014. -- № 6.
    \bibitem{bib:38} Ефимова Е. А. Сравнительный анализ создания имитационной модели пропускной 
        способности городской транспортной сети // Известия высших учебных заведений. Поволжский регион. 
        Технические науки. -- 2009. -- № 1.
    \bibitem{bib:41} Денисов М. В., Агуреев И. Е. Математическое описание динамики пассажирских 
        транспортных систем // Известия Тульского государственного университета. 
        Технические науки. -- 2010. -- № 4-2.
    \bibitem{bib:42} Палант А. Ю. Обзор Методов Обследования Пассажиропотоков // Бизнес Информ. -- 
        2014. -- № 11.
    \bibitem{bib:43} Чернов В. П., Кабалина Т. В. Исследование оценки качества в системе критериев 
        эффективности перевозок пассажиров // Актуальные проблемы экономики и права. -- 2010. -- № 4 (16).
    \bibitem{bib:44} Лойко В. И., Параскевов А. В. Меры по обеспечению эффективной организации городского 
        дорожного движения // Политематический сетевой электронный научный журнал Кубанского 
        государственного аграрного университета. -- 2010. -- № 64.
    \bibitem{bib:45} Кочетов Ю. А. Методы локального поиска для дискретных задач размещения // Специальность 
        05.13. 18 математическое моделирование, численные методы и комплексы программ. -- 2011.
    \bibitem{bib:46} Блох И. И., Дураков А. В. Алгоритмы построения маршрута на карте по параметрам.
    \bibitem{bib:47} Дасгупта С., Пападимитриу Х., Вазирани У. Алгоритмы // М.: МНЦМО. -- 2014.
    \bibitem{bib:48} Шербина О. А. Метаэвристические алгоритмы для задач комбинаторной оптимизации (ОБЗОР).
        [Электронный ресурс]. -- Режим доступа: \url{http://tvim.info/files/56\_72\_Shcherbina.pdf} 
        (дата обращения: 01.08.2015).
    \bibitem{bib:50} Ипатов А. В. Модифицированный метод имитации отжига в задаче маршрутизации 
        транспорта // Труды Института математики и механики УрО РАН. -- 2011. -- Т. 17. -- № 4. -- 
        С. 121-125.
    \bibitem{bib:51} Ипатов А. В. Решение задачи маршрутизации транспорта методом имитации отжига //
        Проблемы теорет. и прикл. математики: тр. -- С. 290-294.
    \bibitem{bib:52} Карпенко А. П. Современные алгоритмы поисковой оптимизации // Алгоритмы, 
        вдохновленные природой: учеб. пособие. М.: Изд-во МГТУ им. НЭ Баумана. -- 2014.
    \bibitem{bib:53} Ковалев М. Я. Теория алгоритмов. Курс лекций: в 2 ч // Минск: БГУ. -- 2003.
    \bibitem{bib:54} Кочетов Ю. А. Вероятностные методы локального поиска для задач дискретной 
        оптимизации // Дискретная математика и ее приложения. Сборник лекций молодежных и научных школ 
        по дискретной математике и ее приложениям. М: МГУ. -- 2001. -- С. 87-117.
    \bibitem{bib:55} Кулаков Ю. А., Воротников В. В. Формирование оптимальных маршрутов в мобильных сетях 
        на основе модифицированного алгоритма Дейкстры~// Вестник НТУУ <<КПИ>>: Информатика, управление 
        и вычислительная техника. -- 2012. -- Т. 2012. -- № 56.
    \bibitem{bib:56} Романовский И. В. Алгоритмы решения экстремальных задач. -- Наука, 1977.
    \bibitem{bib:57} Кочетов Ю. А., Младенович Н., Хансен П. Локальный поиск с чередующимися окрестностями // 
        Дискретный анализ и исследование операций.~-- 2003. -- Т. 10. -- № 1. -- С. 11-43.
\end{thebibliography}

\pagestyle{plain}