\part{Список используемой литературы}

% ссылки с отчёта ВКИТ
% [1] N. Sadovnikova, D. Parygin, E. Gnedkova, B. Sanzhapov, and N. Gidkova. Evaluating the sustainability of Volgograd. In The Sustainable City VIII. WIT Press, 2013.
% [2] ссылка на Томское исследование 
% [3] G. Nielsen and T. Lange. Network design for public transport success–theory and examples. Norwegian Ministry of Transport and Communications, Oslo, 2008.
% [4] Holsapple, C. Decisions and Knowledge. Handbook on Decision Support Systems 1, (Cosgrove). [Электронный ресурс]. — 2008. — Режим доступа: http://www.springerlink.com/index/g182q711470w2510.pdf.
% [5] Tennenhouse, D., Proactive computing / Communications of the ACM, 2000, Vol. 43, pp. 43--50.
% [6] PTV Group. PTV Visum. — Режим доступа: http://vision-traffic.ptvgroup.com/en-uk/products/ptv-visum/.
% [7] INRO Emme
% [8] Citilabs. Cube. — Режим доступа: http://www.citilabs.com/software/cube/.
% [9] A. Ceder. Designing public transport network and routes. Advanced Modeling for Transit Operations and Service Planning, pages 59–91, 2003.
% [10] OpenStreetMap — Режим доступа: http://www.openstreetmap.org 
% [11] Agafonkin V.. Leaflet. — Режим доступа: http://leafletjs.com/.
% [12] Flask is a microframework for Python . — Режим доступа: http://flask.pocoo.org/ 
% [13] “k-means++: The advantages of careful seeding” Arthur, David, and Sergei Vassilvitskii, Proceedings of the eighteenth annual ACM-SIAM symposium on Discrete algorithms, Society for Industrial and Applied Mathematics (2007) 
% [14] “Mean shift: A robust approach toward feature space analysis.” D. Comaniciu and P. Meer, IEEE Transactions on Pattern Analysis and Machine Intelligence (2002)
% [15] Open Source Routing Machine 
% [16] Models and Methods for the Urban Transit System Research / Н.П. Садовникова, Д. Парыгин, М. Калинкина, Б. Санжапов, Ni Ni Trieu // Creativity in Intelligent Technologies and Data Science. CIT&DS 2015 : First Conference (Volgograd, Russia, September 15-17, 2015) : Proceedings / ed. by A. Kravets, M. Shcherbakov, M. Kultsova, O. Shabalina. – [Switzerland] : Springer International Publishing, 2015. – P. 488-499. – (Ser. Communications in Computer and Information Science. Vol. 535).
% [17] A. Ceder. Designing public transport network and routes. Advanced Modeling for Transit Operations and Service Planning, pages 59–91, 2003.
% [18] Гладков Л.А., Курейчик В.В., Курейчик В.М. Генетические алгоритмы / Под ред. В.М. Курейчика.  – 2-е изд., испр. и доп.  - М.: ФИЗМАТЛИТ,  2006. – 320  с.
% [19] Golubev A, Chechetkin I, Solnushkin K.S., Sadovnikova N., Parygin D., Shcherbakov M., Brebels A., Strategway: web solutions for building public    transportation routes using big geodata analysis // Proceedings of The 17th International Conference on Information Integration and Web-based Applications & Services (iiWAS2015) (December 11 - 13, 2015 Brussels, Belgium ) ACM New York, New York pp. 665 - 668
% [20] Mean shift clustering  . – Режим доступа: http://scikit-learn.org/stable/modules/generated/sklearn.cluster.MeanShift.html 