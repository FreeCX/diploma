\chapter{Явление сверхпроводимости}

\section{Основные характеристики сверхпроводящего состояния}

Теория и эксперимент показывают, что лондоновская глубина проникновения 
\( \lambda \), длина когерентности \( \xi \) и энергетическая щель 
\( \Delta \) не являются константами, а зависят от температуры и обладают 
сугубо индивидуальными для заданного материала величинами. Величины 
\( \lambda \) и \( \xi \) имеют минимальное значение при \( T = 0 \) и 
монотонно возрастают с увеличением температуры, стремясь к бесконечности при 
\( T = T_c \) (это объясняется тем, что выше критической температуры никаких 
куперовских пар нет, а магнитное поле беспрепятственно пронизывает вещество). 
Энергетическая щель \( \Delta \), наоборот, имеет максимум при \( T = 0 \) и 
становится равной нулю при \( T = T_c \) (что можно трактовать как отсутствие 
каких-либо корреляций между электронами).

Теория БКШ дает исчерпывающее описание сверхпроводящих свойств материала во 
всём температурном интервале от \( 0 \) до \( T_c \), но является сложной с 
математической точки зрения. Поэтому часто физики прибегают к другому, 
относительно более простому способу анализа сверхпроводящего состояния -- ТГЛ, 
которая прекрасно описывает, качественно и количественно, поведение 
сверхпроводника, но работает только в ограниченном интервале вблизи 
критической температуры.

Теория Гинзбурга–Ландау основывается на теории фазовых переходов 2-го рода. В 
этой теории, наряду с критической температурой, длиной когерентности и 
лондоновской глубиной проникновения, вводится еще одна характеристика, -- 
параметр порядка. С точностью до некоторого коэффициента пропорциональности 
можно считать, что модуль параметра порядка -- это энергетическая щель в 
теории БКШ. Параметр порядка равен нулю при \( T = T_c \) и выше и принимает 
максимальное значение, когда температура достигла абсолютного нуля. Отметим, 
что есть и иная трактовка физического смысла параметра порядка: квадрат его 
модуля определяет концентрацию куперовских пар.

Параметр порядка играет ключевую роль в теории Гинзбурга–Ландау. Через него 
выражается энергия (с точки зрения термодинамики корректнее говорить 
свободная энергия) сверхпроводника.\cite{bib:net}

\section{Сверхпроводимость 1-го и 2-го рода}

Несмотря на то что теория Гинзбурга–Ландау является феноменологической, то 
есть она не объясняет причины возникновения явления, которое она описывает, с 
ее помощью был получен ряд важных результатов. Применив эту теорию, ее авторы 
вычислили поверхностную энергию, возникающую на границе сверхпроводника и 
нормального металла в присутствии внешнего магнитного поля. Оказалось, что 
результат зависит от безразмерной величины, называемой параметром 
Гинзбурга–Ландау \( \kappa \): \( \kappa = \lambda/\xi \) (отношение 
лондоновской глубины проникновения к длине когерентности). Из расчетов 
следовало, что при \( \kappa < 1/\sqrt{2} \) поверхностная энергия оказывается 
положительной. Для сверхпроводника цилиндрической формы, ось которого 
параллельна силовым линиям магнитного поля, данный результат означал, что 
переход в нормальное состояние происходит моментально, как только индукция 
магнитного поля превышает некоторое критическое значение \( B_c \) для данной 
температуры на рисунке~\ref{img:01}. В принципе, ничего нового Гинзбург и 
Ландау не получили, они лишь теоретически подтвердили хорошо известный уже на 
тот момент экспериментальный факт поведения сверхпроводников.

\begin{figure}[h!]
    \center
    \includegraphics[width=.8\textwidth]{img_01}
    \caption{Фазовая диаграмма состояния сверхпроводников 1-го (а) и 
        2-го (б) рода, показывающая, как меняются состояния сверхпроводника 
        при изменении температуры и индукции внешнего магнитного поля. В 
        мейсснеровском состоянии силовые линии магнитного поля не могут 
        проникнуть в вещество. Смешанное или вихревое состояние означает 
        сосуществование сверхпроводимости и нормальных несверхпроводящих 
        тонких нитей, вытянутых вдоль линий магнитного поля. Такие нити 
        называют вихрями Абрикосова, или квантовыми вихрями.}
    \label{img:01}
\end{figure}

Советский физик Николай Заварицкий, исследуя тонкие сверхпроводящие пленки, 
обнаружил, что их поведение в магнитном поле не согласуется с предсказаниями 
теории Гинзбурга–Ландау. Чтобы понять причину расхождения, Алексей Абрикосов, 
основываясь на теории Гинзбурга–Ландау, решил рассмотреть случай, когда 
поверхностная энергия является отрицательной, -- иными словами, попытаться 
понять картину поведения сверхпроводника в магнитном поле с 
\( \kappa > 1/\sqrt{2} \).

Из расчетов следовало, что пока индукция магнитного поля не превосходит нижнее 
критическое значения поля \( B_{c1} \) при фиксированной температуре, 
сверхпроводник находится в мейсснеровском состоянии. После того как индукция 
магнитного поля стала больше \( B_{c1} \), сверхпроводник начинают пронизывать 
своеобразные нити микронных размеров, вытянутые вдоль силовых линий внешнего 
поля. Чем больше индукция поля, тем больше ниток будет в сверхпроводнике. 
Абрикосов установил, что эти образования представляют собой вихри (теперь они 
называются абрикосовскими), ядра которых являются несверхпроводящими, 
нормальными, с размером порядка длины когерентности \( \xi \), а вокруг них 
протекают циркулирующие сверхпроводящие токи, которые экранируют нормальную 
область вихря (ширина области экранировки равна лондоновской глубине 
проникновения \( \lambda \)). Кроме того, в ходе вычислений обнаружилось, что 
вихри несут в себе как бы одну силовую линию внешнего магнитного поля, или 
квант магнитного потока, флюксоид 
\( \Phi_0 = h/2e = 2,07\cdot10^{–15} \text{Тл}\cdot{м}^2 \). Вихри формируют в 
сверхпроводнике треугольную решетку, образуя смешанное (оно же вихревое) 
состояние на рисунке~\ref{img:01}.

Если при заданной температуре продолжить усиливать магнитное поле, то при 
верхнем критическом значении \( B_{c2} \) вихрей станет настолько много, что 
их ядра начнут перекрываться, и они заполнят весь объем сверхпроводника, 
переводя его в нормальное состояние на рисунке~\ref{img:01}.\cite{bib:net}

\section{Сверхпроводимость 1,5-го рода и двухщелевые сверхпроводники}
В 2001 году в дибориде магния \( MgB_2 \) была открыта сверхпроводимость с 
неожиданно высокой критической температурой 39 К. Применяя различные 
экспериментальные техники, ученые установили, что большое значение \( T_c \) 
достигается за счет наличия в \( MgB_2 \) не одной энергетической щели, а 
двух. То есть в сверхпроводящем дибориде магния присутствует как бы два сорта 
куперовских пар. Их взаимодействие и обеспечивает высокую \( T_c \). Важно 
отметить, что у каждого сорта электронных пар есть свой размер, или своя длина 
когерентности. При этом диборид магния имеет лишь одну величину лондоновской 
глубины проникновения.

Возможность нового типа сверхпроводимости, отличного от первого и второго рода 
в многокомпонентных системах \cite{bib:1,bib:2} основана на следующих 
соображениях. Краевая задача в уравнении типа Гинзбурга-Ландау в присутствии 
оборота фазы сводится к одномерной задачи в целом. Кроме того, как указано в 
\cite{bib:1,bib:2}, в общей двухкомпонентной модели есть три фундаментальные 
масштабные длины, которые указывают на невозможно параметризовать модель с 
точки зрения одного безразмерного параметра \( \kappa \). В случае, когда 
конденсаты не связаны межзонной джозефсоновской связью, при условии, что 
векторный потенциал этих масштабных длин представлен двумя независимыми 
величинами: длиной когерентности (устанавливается обратной массой 
соответствующей плотности скалярного поля) и глубиной проникновения магнитного 
поля (устанавливается обратной массой полученного калибровочного поля). В 
противоположность этому, в случае, когда конденсаты соединяются с межзонными 
Джозефсоновскими условиями, можно не различить независимые длины когерентности 
и отнести их к различным конденсатам. Тем не менее, в этом случае колебания 
плотности также могут обладать двумя основными пространственными длинами
\cite{bib:2}, в отличие от однокомпонентных теорий. В \cite{bib:1,bib:2} 
вихревых решениях в двухкомпонентной теории были найдены компоненты, которые 
имеют немонотонное вихревое взаимодействие, с доминирующими частями отвечающие 
за дальнодействующее плотность-плотность взаимодействие и отталкивающей части 
близкодействия типа ток-ток, и электромагнитного взаимодействия. Важным 
обстоятельством, которое было продемонстрировано, что эти вихри 
термодинамически стабильны, несмотря на существование притягивающего хвоста во 
взаимодействии.

\begin{figure}[h!]
    \center
    \includegraphics[width=.8\textwidth]{1-01}
    \caption{Сравнение фазовых диаграмм магнитных фаз чистых сверхпроводников
        первого, второго и полуторного рода при нулевой температуре. В 
        полумейсснеровском режиме макроскопическое разделение фаз в 
        двухкомпонентном Мейсснеровском состоянии и вихревых скоплений, где 
        один из режимов плотности подавляется внутренним перекрытием.}
    \label{fig:1}
\end{figure}

Немонотонный межвихревой потенциал взаимодействия должен привести к 
образованию вихревых скоплений в слабомагнитном поле, погруженного в 
безвихревые пространство, эффект упомянутый в \cite{bib:1} как 
"полумейсснеровское состояние". На рисунке \ref{fig:1} схематические показана 
фазовая диаграмма сверхпроводника тип 1,5.

Если вихри образуют кластеры, то нельзя не использовать обычный одномерный 
аргумент относительно энергии сверхпроводника в нормальном состоянии границы 
раздела(!) для классификации магнитного отклика системы. Прежде всего, энергия 
на вихрь в таком случае зависит от того, находится ли вихрь в кластере или 
нет: т.е. формирование единого изолированного вихря может быть энергетически 
невыгодным, в то время как формирование вихревых кластеров выгодно, потому что 
в кластере, где помещены вихри -- минимум потенциала взаимодействия, энергия 
на квант потока меньше, чем для изолированного вихря (термодинамически 
немонотонный потенциал взаимодействия двух вихрей предусматривает, что 
наименьшей энергией на квант потока будет в случае равномерной сетки с шагом 
равным не менее двухчастичного межвихревого потенциала).

Таким образом, помимо энергии вихря в кластере, появляется дополнительная 
энергическая характеристика, связанная с границей кластера. Другими словами, в 
этой ситуации, для определения магнитного отклика системы недостаточно 
изучения краевой задачи и задачи одиночного вихря, в отличии от системы 
отдельных составляющих. В кластере система стремится к минимуму граничной 
энергии (так же, как и в сверхпроводниках 1-го рода), в то время нарушение 
структуры одного квантового вихря внутри кластера (аналогично 
сверхпроводимости второго рода с отрицательной энергией межфазного 
взаимодействия). Таким образом, увеличение магнитного поля образуется 
посредством фазового перехода первого рода. Магнитная фаза отлична от вихря в 
мейснеровском состоянии, затем возникает макроскопическое фазовое 
распределение в двухкомпонентной области Мейснера и вихревых кластерах, где 
один из режимов плотностей подавляется основным перекрытием.\cite{bib:main}

\newpage