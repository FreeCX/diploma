% написать черновой варинат по теории 
\chapter{Введение в проблему синтеза маршрутов}
% ----------------------------------------------
% обзор литературы
% ----------------------------------------------
\section{Анализ предметной области}
\section{Состояние современных исследований}
% ----------------------------------------------
\section{Требования к методам}
\section{Заключение}

% -------- разместить в главе 05-problem -------
Формально данная задача связана с поиском оптимального пути на графе. Были проанализированы различные и 
широко используемые алгоритмы построения пути: \( A^\star \), \( IDA^\star \), поиск в ширину 
(Breadth-First-Search), поиск <<лучший-первый>> (Best-First-Search), алгоритм Дейкстры, двунаправленный 
\( A^\star \), Jump Point Search, Orthogonal Jump Point Search, двухэтапные алгоритмы (ALT, Reach), а так 
же генетические алгоритмы случайного поиска [18]. 
% --- больше ссылко на источники (алгоритмы) ---
% ----------------------------------------------