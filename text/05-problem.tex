% написать черновой вариант по теории 
\chapter{Введение в проблему синтеза маршрутов}
% ---
Следует отметить, что построение систем поддержки принятия решений рассматривается в контексте систем 
сбора, хранения и обработки данных, реализующих алгоритмы генерации информации в соответствии с различными 
уровнями представления \cite{bib:4}. В рамках данной концепции человек -- лицо принимающий решение -- 
должен быть исключен из контура оперативного управления и перемещен на уровень супервизорного управления 
\cite{bib:5}.

На основе анализа программных продуктов, специализирующихся на задачах транспортного моделирования 
(PTV Visum \cite{bib:6}, INRO Emme \cite{bib:7}, Citilabs Cube Cloud \cite{bib:8} и др.), были выявлены 
основные функции, которые в основном ориентированы на оценку качества действующей или проектируемой 
маршрутной сети.

PTV Visum (компания PTV Group): предлагает инструменты для проведения комплексного анализа действующих 
сетей, позволяет оценивать технико-экономические характеристики проектируемой сети маршрутов. На основе 
анализа матриц корреспонденций между районами, реализована возможность прогнозирования транспортного спроса, 
что помогает оценить количество необходимого подвижного состава и составить расписание его работы. 
Emme (компания INRO Software) и Cube Cloud (компания Citilabs): SaaS-программные продукты со схожей 
функциональностью.

Таким образом, если рассматривать задачу поддержки принятия решений по совершенствованию маршрутов 
общественного транспорта с точки зрения (а) анализа существующих маршрутов на предмет удовлетворения 
спроса и (б) модификации существующих (через предложение новых вариантов маршрутов) с целью удовлетворения 
спроса жителей (уменьшение среднего времени пешего подхода до остановки и уменьшение времени поездки), то 
рассмотренные системы не решают сформулированные задачи и имеют следующие недостатки: (а) не учитывают 
предпочтения жителей по перемещениям в современной городской среде; (б) не адаптированы к работе с 
большими данными.

Для решения задачи предлагается подход, состоящий из следующих шагов. 
\begin{enumerate}
    \item Сбор и/или моделирование геораспределенных данных о перемещении жителей в городе.
    \item Кластеризация полученных данных с целью выявления узлов (кандидаты остановочных пунктов) для 
        включения в маршруты сети.
    \item Построение вариантов сетей маршрутов общественного транспорта.
    \begin{enumerate}
        \item Задание характеристик проектируемой сети и функции, определяющей эффективность 
            функционирования сети.
        \item Формирование начального варианта сети маршрутов общественного транспорта.
    \end{enumerate}
    \item Модификация начального варианта сети маршрутов общественного транспорта (генерация 
        альтернативных сетей и их сравнение по критерию минимума функции затрат).
    \item Определение эффективности каждого варианта сети.
\end{enumerate}
Следует отметить, что данный подход не противоречит существующим практикам построения маршрутов, но 
обеспечивает реализацию 1-3 шагов в автоматическом режиме. Тогда как в существующих методиках \cite{bib:9} 
эти этапы не автоматизированы. Пользователю предлагается либо вручную задать матрицу корреспонденций, 
либо загрузить готовую маршрутную сеть.

% ---
Cities are growing, almost all of them have the complex transport infrastructure. Management of urban 
infrastructure or urban planning is a set of complicated procedures as well, where a certain solution 
might have a strong impact or do not satisfy all inhabitants. Urban planning covers transportation, 
communications, and distribution networks design and many others towards to improving people’s life. 
A new challenge for fast-growing cities is in making public transportation system more reliable to 
citizens. With the ideal public transportation network certain person spends as little time as possible 
for comfort travelling like he or she has a personal carrier. 

Often current transport system development procedures do not take into consideration real preference of 
the most of the inhabitants. It leads to fundamental issues: public transport becomes not so convenient 
to use, people prefer their personal cars and spend much time in traffic. All these negative effects are 
the reasons for the dissatisfaction of cities residents. 

Nowadays, new ubiquitous data collection and processing technologies which getting cheaper and cheaper 
bring new opportunities for planners. At least, the most of the people who live in cities have mobile 
phones and smartphones. Cell operators are able to track people calls and received a huge amount of data 
about users travelling. Based on the assumption that a certain person has their own ‘transfer’ patterns 
(the most using origin-destination points) which can be considered as personal requirements for 
transportation systems. Having this data we able to (i) evaluate the efficiency of the current 
transportation network and (ii) suggest modification based on citizen’s preferences. 

To solve this problem it is necessary to perform the following procedures: (i) collecting of data, 
(ii) data reduction, (iii) creating a set of (sub)optimal routes and (iv) pick up the appropriate network 
according to the quality criteria. Data collection requires techniques for data receiving and 
preprocessing for further efficient storing. It is not necessarily real-time data processing, data 
snapshots or dumps at the certain moment of time have been used. The crucial points of the data gathering 
is data quality assurance, such as gap detection. Since we have transfer data of every single persons, 
we are able to evaluate the most optimal stops as nodes in public transportation routes. Suppose we get 
information about 10\% of inhabitants of a city with 3.5 mln population. In this case, the correspondence 
matrix has \( 350 000^2 \) elements. Reduction of original nodes count allows to detect the centers of 
nodes clusters and to understand where stops could be located. 

The next step is to connect clusters centers and get the public transportation network with (sub) 
optimal routes. This is a very complicated multi-objective problem with a set of constraints. This 
problem lays at an intersection of the shortest path problem and vehicle routing problem with the 
parameter of passenger traffic flow. The shortest path problem is the well known problem and there 
are many efficient and fast program implementations e.g. OpenStreetMap. However, there are several 
specifics of the problem and basic algorithms do not applicable as is. In contrast with the shortest 
path problem with the timewise cost function, the considered cost function is more complicated, as it 
includes walk distance (time), travel time, the number of changes. All these variables are averaged by 
a number of residents. Also the important difference, the final route could not be arbitrary, as every 
route must contain nodes (stops) which represent centers of clusters defined in the previous step. 

Getting the optimal public network with the one step is a very difficult and requires intensive human 
intervention. As the result, evolution-based algorithms can be applied to avoid additional costs at 
the initial stage. There are a number of activities associated with the proposed approach: 
(i) creating the initial routes network, (ii) change the initial network by applying evolutionary 
operation such as mutation and crossover; (iii) evaluate the current networks and by the cost function. 

% ----------------------------------------------
% обзор литературы
% ----------------------------------------------
\section{Анализ предметной области}
% описание предметной области

\section{Состояние современных исследований}
Public transportation planning and developing public transport network is well studied domain. 
According to ~\cite{ceder2007} the planning methodology includes the sequence of actions: 
(i) public transportation network design; (ii) setting timetables; (iii) scheduling vehicles 
to trips, (iv) and assignment of drivers and other maintenance work. 

Also we observe some theoretical results in path design and route planning. Thomas L. Rodeheffer 
suggested the Symmetric Shortest-Path Table Routing Conjecture. In the work 
\cite{rodeheffer2013symmetric} he describes a symmetric shortest-path table routing, and presented 
counterexamples to the hypothesis: every possible symmetric shortest-path table routing can be 
selected in this way by some edge weight salting.

Daniel Delling et al suggested round-based public transit routing approach. In the paper 
\cite{delling2014round} authors explore the problem of computing all Pareto-optimal journeys in a 
dynamic public transit network and introducing round-based public transit router algorithm (named RAPTOR).

Another work \cite{delling2015customizable} explains customizable route planning in road networks. In 
this paper, authors propose routing engine for computing driving directions in large-scale road networks. 
Also Daniel Delling et al, in the paper \cite{delling2015public} suggested public transit labeling as the 
solution of the journey planning problem in public transit networks. 

Ittai Abraham et al, in the paper \cite{abraham2013alternative} suggested alternative routes in Road 
Networks. In this paper, authors explore the problem of finding good alternative routes in road networks. 
They search for routes which (i) are substantially differ from the shortest path, (ii) have small stretch, 
and (iii) are locally optimal. 

Ling-Yin Wei et al, in the work \cite{wei2012constructing} suggested RICK algorithm for constructing 
popular routes from uncertain trajectories using information about sequence of locations and time 
intervals.

Tim Dwyer and Lev Nachmanson, in the work \cite{dwyer2009fast} suggested fast edge-routing for large 
graphs. In this paper two methods for achieving faster edge routing using approximate shortest-path 
techniques are suggested and discussed.

% -------- разместить в главе 05-problem -------
Формально данная задача связана с поиском оптимального пути на графе. Были проанализированы различные и 
широко используемые алгоритмы построения пути: \( A^\star \), \( IDA^\star \), поиск в ширину 
(Breadth-First-Search), поиск <<лучший-первый>> (Best-First-Search), алгоритм Дейкстры, двунаправленный 
\( A^\star \), Jump Point Search, Orthogonal Jump Point Search, двухэтапные алгоритмы (ALT, Reach), а 
так же генетические алгоритмы случайного поиска \cite{bib:19}. 

Algorithms such as A*, IDA*, Breadth-First-Search, ALT (\( A^\star \), landmarks and triangle inequality) 
basically work without data preprocessing, resulting in the need to traverse large graphs and considerable 
associated time costs. In contrast, in algorithms such as HLC (Hub Label Compression), HL (Hub Labeling), 
TNR (Transit Node Routing) and table search, the main costs are associated with data preprocessing and 
disk storage, whereas the actual run time of the algorithm is modest \cite{bib:21}.
% --- больше ссылок на источники (алгоритмы) ---
% ----------------------------------------------

% --- duplicate (see header) ---
Besides on theoretical results, software implementations for decision support in urban transportation 
planning have been observed. PTV Visum~\cite{bib:6} is software for evaluation existed (defined by 
responsible person) public transport network based on transportation matrix. Another software called 
Emme, proposed by ``INRO Software''~\cite{bib:7} and SaaS solution by Citilabs named 
Cube Cloud~\cite{bib:8}.
% --- duplicate (see header) ---

The Transportation Analysis and Simulation System (TRANSIMS) developed by \cite{transims} is a set of 
tools for analysis of regional transportation systems. TRANSIMS is an open source project which is 
available for public usage under NASA Open Source Agreement Version 1.3.
 
Also, in urban development traffic modelling software is used, like Aimsun that allows to model fusing 
travel demand and simulate networks of difference complexity \cite{aimsun}.

% ----------------------------------------------
% other information
% ----------------------------------------------
В книге Алгоритмы решения экстремальных задач\cite{bib:56} подробно излагается теория и численные методы 
для решения важных задач: от задач линейного программрования до ряда дискретных задач. Также сюда входят 
и транспортные задачи, которой посвящены главы с третьей по четвёртую, а также пятая глава, где 
рассматриваются многоэкстремальные задачи на графах. Изложение численных методов сопровождается разбором 
алгоритмов, а также у обращение внимания на эффективную организацию вычислительного процесса. Хотя книга 
и вышла в 77 году она не потеряла своей актуальности.

В статье Алгоритмы построения маршрута на карте по параметрам\cite{bib:46} предоставляется обоснование 
о необходимости создания алгоритмов построения маршрута по параметрам, для того чтобы он соотвествовал 
необходимым предпочтениям жителей, а также о пользе использования генетических алгоритмах в задаче 
построения маршрутов.

В диссертационной работе <<Методы локального поиска для дискретных задач размещения>>\cite{bib:45} 
рассматриваються математические модели размещения, а конкретно дискретные оптимизационные задачи. Данная 
работа носит в основном теоретический характер и является отличным источником информации о эвристических 
методах и их теоретической базе.

В статье <<Формирование оптимальных маршрутов в мобильных сетях на основе модифицированного алгоритма 
Дейкстры>>\cite{bib:55} представлен модифицированный алгоритм Дейкстры. Модификация предложенная в статье 
предлагает учитывать динамику загруженности сети, что было бы неплохо для динамического построения 
актуальных маршрутных сетей транспорта.

В книге Алгоритмы\cite{bib:47} подробно излагаются основные методы построения и анализа эффективных 
алгоритмов. Материал в книге довольно отличается от других изданий методом подачи информации -- баланс в 
математической строгости и использовании формальных подходов, а также в выборе освещаемых вопросов. 
Издание является наиболее свежим источником информации по алгоритмам читаемым студентам в университете Беркли.

Автор статьи <<Метаэвристические алгоритмы для задач комбинаторной оптимизации >>\cite{bib:48} описывает 
теоретическую базу метаэвристических алгоритмов и подробно рассматривает методы на их основе. Статья носит 
обзорных характер и предназначена для краткого ознакомления с метаэвристическими методами.

Автор статей \cite{bib:50,bib:51} описывает применение метода имитации отжига к задаче маршрутизации 
транспорта, а также производит сравнение с другими эвристическими методами и показывает, что метод 
используемый в статье даёт меньшую погрешность.

В книге <<Современные алгоритмы поисковой оптимизации>>\cite{bib:52} автор подробно изложил современные 
стохастические алгоритмы решения однокритериальной задачи оптимизации, а также методы повышения 
эффективности этих алгоритмов. Также в данной учебном пособии рассматривается задача многокритериальной 
оптимизации и алгоритмы для её решения.

В книге <<Теория алгоритмов>>\cite{bib:53} предоставляется информация по основным вопросам разработки и 
анализа приближенных алгоритмов для задач оптимизации. Рассмотрены методы построения алгоритмов 
локального поиска и метаэвристики, а также основное внимание уделено математическим методам решения задач 
и методам анализа их вычислительной сложности.

В работах \cite{bib:54,bib:57} автор приводит обзор методов для решения NP-трудных задач, рассматриваются 
общие схемы алгоритмов. Особое внимание уделено математическому аппарату рассматриваемых алгоритмов. 
Если работа \cite{bib:54} носит более общих характер, то \cite{bib:57} рассматривает более узконе 
направление -- методы локального поиска с чередующимися окрестностями.


% ----------------------------------------------

% ----------------------------------------------
\section{Требования к методам}
\section{Заключение}