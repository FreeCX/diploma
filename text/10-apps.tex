\addcontentsline{toc}{chapter}{Приложение А Результаты экспериментов}
\label{ch:A}
\begin{center}
    Приложение А\\
    Результаты экспериментов
\end{center}
\newpage

\addcontentsline{toc}{chapter}{Приложение Б Единицы измерения}
\label{ch:B}
\begin{center}
    Приложение В\\
    Единицы измерения
\end{center}

В этом разделе дадим явное отображение используемого представления модели ГЛ к 
более общему виду представленного в учебниках. Рассмотрим модель 
Гинзбурга-Ландау в общем представлении
\begin{align}
  F = & \frac{\hbar^2}{2m_1} \abs{\left( \nabla - i\frac{e^*}{\hbar c} A\right)
    \psi_1}^2 + \frac{\hbar^2}{2m_2} \abs{\left( \nabla - i\frac{e^*}{\hbar c}A
    \right) \psi_2}^2 - \nonumber \\
  & - \nu\hbar^2 \Re\left\{ \left( \nabla - i\frac{e^*}{\hbar c}
    \right)\psi_1 \left( \nabla + i\frac{e^*}{\hbar c}A \right)
    \psi^*_2\right\} + \frac{1}{8\pi}\left( \nabla\times A \right)^2 +
    \alpha_1\abs{\psi_1}^2 + \nonumber \\
  & \frac{1}{2}\beta_1\abs{\psi_1}^4 + \alpha_2\abs{\psi_2}^2 + \frac{1}{2}
    \beta_2\abs{\psi_2}^4 - \eta_1\abs{\psi_1}\abs{\psi_2}
    \cos(\theta_1 - \theta_2) + \eta_2\abs{\psi_1}^2\abs{\psi_2}^2.
    \label{eq:B-1}
\end{align}

Определим обозначенные величины
\begin{gather}
  \tilde{F} = \frac{4\pi}{\hbar^2 c^2}F; \quad
    \tilde{A} = -\frac{A}{\hbar c}; \quad
    \tilde{\psi}_a = \sqrt{\frac{4\pi}{m_a c^2}}\psi_a; \quad
    \tilde{\nu} = \frac{m_1m_2}\nu \nonumber \\
  \tilde{\alpha}_a = \frac{m_a}{\hbar^2}\alpha_a; \quad
    \tilde{\beta}_a = \frac{m^2_a c^2}{4\pi\hbar^2}\beta_a; \quad
    \tilde{\eta}_1 = \frac{\sqrt{m_1 m_2}}{\hbar^2}\eta_1; \quad
    \tilde{\eta}_2 = \frac{m_1 m_2 c^2}{4\pi\hbar^2}\eta_2. \label{eq:B-2}
\end{gather}

Тогда
\begin{align}
  \tilde{F} = & \frac{1}{2}\abs{\left( \nabla + ie^*\tilde{A} \right)
    \tilde{\psi}_2}^2 + \frac{1}{2}\abs{\left( \nabla + ie^*\tilde{A} \right)
    \tilde{\psi}_2}^2 - \nonumber \\ 
  & \tilde{\nu} \Re\left\{ \left( \nabla + ie^*\tilde{A}
    \right)\tilde{\psi}_1 \cdot \left( \nabla + ie^*\tilde{A} \right)
    \tilde{\psi}^*_2 \right\} + \frac{1}{2}\abs{\nabla\times\tilde{A}}^2 - 
    \nonumber \\
  & \tilde{\alpha}_1
    \abs{\tilde{\psi}_1}^2 + \frac{\tilde{\beta}_1}{2}\abs{\tilde{\psi}_1}^2 +
    \tilde{\alpha}_2\abs{\tilde{\psi}_2}^2 +
    \frac{\tilde{\beta}_2}{2}\abs{\tilde{\psi}_2}^2 + \nonumber \\
  & + \tilde{\eta}_1 \abs{\tilde{\psi}_1} \abs{\tilde{\psi}_2}
    \cos(\theta_1 - \theta_2) + \tilde{\eta}_2\abs{\tilde{\psi}_1}^2
    \abs{\tilde{\psi}_2}^2, \label{eq:B-3}
\end{align}
которая, опустив знак тильды, совпадает с представленным \eqref{eq:2} в
данной работе.

На протяжении всей работы, предполагается, что зона 1 активна, то есть, 
\( \alpha_1 < 0 \). Перепишем уравнение \eqref{eq:2} для \( F \), так, чтобы 
\( \alpha_1 \) была нормирована на \( -1 \) и \( \beta_1 \) нормирована на 
\( 1 \). Это может быть достигнуто следующим образом. Напомним, что в 
отсутствии межзонной связи (т.е. когда \( \eta_1 = \eta_2 = \nu = 0 \)) 
конденсат 1 имеет затухание линейного масштаба 
\( 1/\hat{\mu}_1 = (-4\alpha_1)^{-1/2} \). Эта размер обычно определяется 
длиной когерентности
\begin{equation}
    \hat{\xi}_1 = \frac{\sqrt{2}}{\hat{\mu}_1} = \frac{1}{\sqrt{-2\alpha_1}}
    \label{eq:A-4}
\end{equation}

Ещё раз подчеркнём, что в присутствии межзонной связи, \( \hat{\xi}_1 \) не
является длиной когерентности конденсата 1. Такое определение параметра с 
шляпой, указывает, что это оригинальная длина когерентности только в 
несвязанном случае. Напомним также, что вакуумная плотность конденсата в 
несвязанной модели
\begin{equation}
    \hat{u}_1 = \sqrt{\frac{-\alpha_1}{\beta_1}}
    \label{eq:B-5}
\end{equation}

Второе перемасштабирование \( \sqrt{2}\hat{\xi}_1 \) как единицы длины
и \( \hat{u}_1 \) как единицы плотности конденсата (вместе с компенсирующим
масштабированием \( F \), \( e^* \) и \( A \)). Конкретно говоря
\begin{gather}
  \bar{r} = \frac{r}{\sqrt{2}\hat{\xi}_1} = \sqrt{-\alpha_1 r} \nonumber \\
  \bar{F} = \frac{2\hat{\xi}^2_1}{\hat{u}^4_1}F =
    \frac{\beta^2_1}{-\alpha^3_1}F \nonumber \\
  \bar{\psi}_a = \frac{\psi_a}{\hat{u}_1} =
    \sqrt{\frac{\beta_1}{-\alpha_1}}\psi_a \nonumber \\
  \bar{A} = \frac{A}{\hat{u}_1} \\
  \bar{e} = \frac{1}{\sqrt{2}}\hat{u}_1\hat{\xi}_1 e^* =
    \frac{e^*}{\sqrt{\beta_1}} \nonumber \\
  \bar{\alpha}_2 = 2\hat{\xi}^2_1 \alpha_2 = \frac{\alpha_2}{-\alpha_1}
    \nonumber \\
  \bar{\eta}_1 = 2\hat{\xi}^2_1 \eta_1 = \frac{\eta_1}{-\alpha_1} \nonumber \\
  \bar{\eta}_2 = 2\hat{\xi}^2_1 \hat{u}^2_1 \eta_2 =
    \frac{\eta_2}{\beta_1} \nonumber \\
  \bar{\nu} = \nu \label{eq:B-6}
\end{gather}

Подставляя их в \eqref{eq:2} получаем
\begin{align}
  \bar{F} = & \frac{1}{2}\abs{\left( \hat{\nabla} + i\hat{e}\hat{A} \right)
    \hat{\psi}_1}^2 + \frac{1}{2}\abs{\left( \hat{\nabla} + i\hat{e}\hat{A}
    \right)\hat{\psi}_2}^2 - \nonumber \\ 
  & \nu\Re\left\{ \left( \hat{\nabla} + i\hat{e}\hat{A}
    \right) \hat{\psi}_1 \cdot \left( \hat{\nabla} - i\hat{e}\hat{A} \right)
    \hat{\psi}^*_2 \right\} + \nonumber \\
  & + \frac{1}{2}\abs{\hat{\nabla}\times\hat{A}}^2 - \abs{\hat{\psi}_1}^2 +
    \frac{1}{2}\abs{\hat{\psi}_1}^2 - \hat{\alpha_2}\abs{\hat{\psi}_2}^2 +
    \frac{\hat{\beta}_2}{2}\left| \hat{\psi}_2 \right|^2 - \nonumber \\
  & - \hat{\eta}_1\abs{\hat{\psi}_1}\abs{\hat{\psi}_2} \cos(\theta_1 - \theta_2)
    + \hat{\eta}_2\abs{\hat{\psi}_1}^2 \abs{\hat{\psi}_2}^2, \label{eq:B-7}
\end{align}

Это (опуская верхнюю черту) и есть энергия ГЛ которая используется в данной 
работе для целей численного моделирования.

Окончательно отметим, что однокомпонентная модель ГЛ получается из 
\eqref{eq:B-7} определяя \( \psi_2 \equiv 0 \), глубину проникновения
\( \lambda = 1/\mu_a = 1/e \) и длину когерентности
\( \xi = 1/\sqrt{2} \), и, следовательно, параметр теории ГЛ
\begin{equation}
  \kappa = \lambda/\xi = \frac{\sqrt{2}}{e}
  \label{eq:B-8}
\end{equation}

Так, в данной работе, можно считать \( e \) в виде обратного параметра теории 
ГЛ для однощелевой модели. Значение \( e \) соответствующий предельной 
однокомпонентной теории Богомольного c \( e_c = 2 \) в этой интерпретации.

\newpage

\addcontentsline{toc}{chapter}{Приложение В Код программы}
\label{ch:C}
\begin{center}
    Приложение В\\
    Код Программы
\end{center}
\lstinputlisting[language=c++]{../code/calculation/vortex.cpp}