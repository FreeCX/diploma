\documentclass[a4paper, 14pt]{extreport}
\usepackage[T2A]{fontenc}
\usepackage[utf8x]{inputenc}
\usepackage[english, russian]{babel}
\usepackage{indentfirst}
\usepackage{amssymb, amsthm, amsfonts, amsmath, mathtext, wasysym}
\usepackage{cite, enumerate, float}
\usepackage{graphicx}
\usepackage{multicol, multirow, array}
\usepackage{times}
\usepackage{geometry}
\usepackage{setspace}
\usepackage{titlesec}
\usepackage[square, numbers, sort&compress]{natbib}
\usepackage{tocloft}
\usepackage{caption}
\usepackage{listings}
\usepackage{fancyhdr}
\usepackage[colorlinks, linkcolor=black, citecolor=black, urlcolor=blue]{hyperref}
\usepackage{algorithm2e}
\geometry{left=2.0cm}
\geometry{right=1.5cm}
\geometry{top=2.0cm}
\geometry{bottom=2.0cm}
\graphicspath{{images/},{plots/}}

% bibtex cyrillic hack
\newcommand{\spch}{:~}

% use Times New Roman for text
\renewcommand{\rmdefault}{ftm}
\lstset{
    language=C++,
    extendedchars=\true,
    inputencoding=utf8,
    basicstyle=\scriptsize\usefont{T2A}{fcr}{m}{n},
    keywordstyle=\bfseries,
    commentstyle=\itshape,
    numbers=left,
    numberstyle=\tiny,
    breakatwhitespace=\false,
    breaklines=\true,
    tabsize=2, 
}

% fancyhdr page style
\pagestyle{fancy}
\fancyhf{}
\fancyhead[R]{\thepage}
\fancyheadoffset{0mm}
\fancyfootoffset{0mm}
\setlength{\headheight}{17pt}
\renewcommand{\headrulewidth}{0pt}
\renewcommand{\footrulewidth}{0pt}
\fancypagestyle{plain}{
    \fancyhf{}
    \rhead{\thepage}}
\setcounter{page}{5}

% title style
\titleformat{\chapter}
    {\centering\normalsize}
    {\thechapter}
    {8pt}{\MakeUppercase}
\titleformat{\section}
    {\centering\normalsize}
    {\thesection}
    {1em}{}
\titleformat{\subsection}
    {\centering\normalsize}
    {\thesubsection}
    {1em}{}
  
\titlespacing*{\chapter}{0pt}{-30pt}{8pt}
\titlespacing*{\paragraph}{0pt}{-30pt}{8pt}
\titlespacing*{\section}{\parindent}{*4}{*4}
\titlespacing*{\subsection}{\parindent}{*4}{*4}

\makeatletter

\bibliographystyle{gost71u2003}
\renewcommand{\@biblabel}[1]{#1} 

% bibliography bibitem item indent
\renewenvironment{thebibliography}[1]
    {\chapter*{\bibname}%
    \@mkboth{\MakeUppercase\bibname}{\MakeUppercase\bibname}%
    \list{\@biblabel{\@arabic\c@enumiv}}%
        {\settowidth\labelwidth{\@biblabel{#1}}%
        \leftmargin=0pt
        \itemindent=50pt
        \@openbib@code
        \usecounter{enumiv}%
        \let\p@enumiv\@empty
        \renewcommand\theenumiv{\@arabic\c@enumiv}}%
    \sloppy
    \clubpenalty4000
    \@clubpenalty \clubpenalty
    \widowpenalty4000%
    \sfcode`\.\@m}
        {\def\@noitemerr
        {\@latex@warning{Empty `thebibliography' environment}}%
    \endlist}
    
\makeatother

\newcolumntype{C}[1]{>{\centering\arraybackslash}m{#1\textwidth}}
\renewcommand{\arraystretch}{1.2}

\DeclareCaptionLabelFormat{figure}{Рисунок #2}
\DeclareCaptionLabelFormat{table}{Таблица #2}
\DeclareCaptionLabelSeparator{sep}{~---~}
\captionsetup{labelsep=sep, justification=centering, font=small}
\captionsetup[figure]{labelformat=figure}
\captionsetup[table]{labelformat=table}

\renewcommand{\cfttoctitlefont}{\normalfont\hspace{0.38\textwidth}}
\renewcommand{\cftbeforetoctitleskip}{-1em}
\renewcommand{\cftchapfont}{\normalsize}
\renewcommand{\cftsecfont}{\hspace{-21pt}}
\renewcommand{\cftsubsecfont}{\hspace{-53pt}}
\renewcommand{\cftbeforechapskip}{0em}
\renewcommand{\cftparskip}{-1mm}
\renewcommand{\cftdotsep}{1}
\renewcommand{\cftsecpresnum}{}
\renewcommand{\cftchapleader}{\cftdotfill{\cftdotsep}}
\renewcommand{\cftsecleader}{\cftdotfill{\cftdotsep}}
\renewcommand{\cftsecaftersnum}{.}
\renewcommand{\cftsubsecaftersnum}{.}
\setcounter{tocdepth}{2}

\renewcommand{\theenumi}{\arabic{enumi}}
\renewcommand{\labelenumi}{\arabic{enumi})}
\renewcommand{\theenumii}{.\arabic{enumii}}
\renewcommand{\labelenumii}{\arabic{enumi}.\arabic{enumii})}
\renewcommand{\theenumiii}{.\arabic{enumiii}}
\renewcommand{\labelenumiii}{\arabic{enumi}.\arabic{enumii}.\arabic{enumiii})}

\newcommand{\pder}[2] {\frac{\partial #1}{\partial #2}}
% частная производная от #1 по #2
\newcommand{\ppder}[2]{\frac{\partial^2 #1}{\partial {#2}^2}}
% вторая частная производная от #1 по #2
\newcommand{\pcder}[3]{\frac{\partial^2 #1}{\partial #2 \partial #3}}
% вторая частная производная от #1 по #2 и #3
\newcommand{\der}[2]  {\frac{d #1}{d #2}}
% производная от #1 по #2
\newcommand{\dder}[2] {\frac{d^2 #1}{d {#2}^2}}
% вторая производная от #1 по #2
\newcommand{\abs}[1]{\left| #1 \right|}
% обозначение модуля

% Перевод плагина algorithm2e
\SetKwInput{KwData}{Исходные параметры}
\SetKwInput{KwResult}{Результат}
\SetKwIF{If}{ElseIf}{Else}{если}{тогда}{иначе\ если}{иначе}{конец\ условия}
\SetKwFor{While}{до\ тех\ пор,\ пока}{выполнять}{конец\ цикла}
\SetAlgorithmName{Алгоритм}{алгоритм}{Список алгоритмов}

\addto{\captionsrussian}{\renewcommand*{\contentsname}{СОДЕРЖАНИЕ}}

\begin{document}
    \onehalfspacing
    \begin{titlepage}
    \begin{center}
        Министерство образования и науки Российской Федерации \\
        \vspace{.5cm}
        Федеральное государственное бюджетное образовательное учреждение высшего образования\\
        <<Волгоградский государственный технический университет>>\\
        Факультет электроники и вычислительной техники\\
        \vspace{.5cm}
        Кафедра <<Системы автоматизированного проектирования и поискового конструирования>>
        \vspace{.5cm}
    \end{center}
    \begin{flushright}
        \begin{center}
            \hspace*{10.5em}Утверждаю
        \end{center}
        И.о.зав. кафедрой <<САПР~и~ПК>>\\
        \UNDER{\LINE{3cm}}{\TINY{(подпись)}}\quad\UNDER{Щербаков~М.~В.}{\TINY{(инициалы, фамилия)}}\\
        <<\underline{\hspace{2em}}>> \underline{\hspace{7.5em}} \the\year\ г.
    \end{flushright}
    \begin{center}
        \large ПОЯСНИТЕЛЬНАЯ ЗАПИСКА\\
        \UNDER{к \underline{\smash{\hspace{0.2\textwidth}магистерской диссертации\hspace{7em}}} на тему}
            {\TINY{(наименование вида работ)}}\\
        \vspace{5pt}\underline{\hspace{\textwidth}}
        \underline{\hspace{\textwidth}}
    \end{center}
    Автор \UNDER{\LINE{10em}}{\TINY{(подпись и дата подписания)}} \quad 
    \UNDER{\underline{\smash{\hspace{2.9em}\STUDENTO\hspace{2.85em}}}}{\TINY{(фамилия, имя, отчество)}}\\
    Обозначение \UNDER{\underline{\smash{\hspace{6em}МД-40461806-10.27-01-16.81\hspace{9.55em}}}}
        {\TINY{(обозначение документа)}}\\
    Группа \UNDER{\LINE{8em}}{\TINY{(шифр группы)}}\\
    Направление \underline{\hspace{1em}09.04.01 <<Информатика и вычислительная техника>>,}\\
    \vspace{5pt}\UNDER{\underline{программа\hspace{0.87\textwidth}}}{(код по ОКСО, наименование направления, программы)}\\
    Руководитель работы \UNDER{\LINE{12.3em}}{\TINY{(подпись и дата подписания)}}\hspace{2em}
        \UNDER{\underline{\smash{\hspace{1.5em}М.~В.~Щербаков\hspace{1.5em}}}}{\TINY{(инициалы и фамилия)}}\\
    Консультанты по разделам:\\
    \vspace{1em}\UNDER{\LINE{7.9cm}}{\TINY{(краткое наименование раздела)}}\hquad
        \UNDER{\LINE{4.7cm}}{\TINY{(подпись и дата подписания)}}\hquad
        \UNDER{\LINE{4cm}}{\TINY{(инициалы и фамилия)}}\\
    \vspace{1em}\UNDER{\LINE{7.9cm}}{\TINY{(краткое наименование раздела)}}\hquad
        \UNDER{\LINE{4.7cm}}{\TINY{(подпись и дата подписания)}}\hquad
        \UNDER{\LINE{4cm}}{\TINY{(инициалы и фамилия)}}\\
    \vspace{1em}\UNDER{\LINE{7.9cm}}{\TINY{(краткое наименование раздела)}}\hquad
        \UNDER{\LINE{4.7cm}}{\TINY{(подпись и дата подписания)}}\hquad
        \UNDER{\LINE{4cm}}{\TINY{(инициалы и фамилия)}}\\
    Нормоконтролер \UNDER{\LINE{7cm}}{\TINY{(подпись, дата подписания)}}\hquad
        \UNDER{\underline{\smash{\hspace{2em}Н.~П.~Садовникова\hspace{2em}}}}{\TINY{(инициалы и фамилия)}}\\
    \vspace{\fill}
    \begin{center}
        Волгоград \the\year\ г.
    \end{center}
\end{titlepage}

\begin{titlepage}
    \begin{center}
        Министерство образования и науки Российской Федерации \\
        \vspace{.5cm}
        Федеральное государственное бюджетное образовательное учреждение высшего образования\\
        <<Волгоградский государственный технический университет>>\\
        Факультет электроники и вычислительной техники\\
        \vspace{.5cm}
        Кафедра <<Системы автоматизированного проектирования и поискового конструирования>>
        \vspace{.5cm}
    \end{center}
    \begin{flushright}
        \begin{center}
            \hspace*{10.5em}Утверждаю
        \end{center}
        И.о.зав. кафедрой <<САПР~и~ПК>>\\
        \UNDER{\LINE{3cm}}{\TINY{(подпись)}}\quad\UNDER{Щербаков~М.~В.}{\TINY{(инициалы, фамилия)}}\\
        <<\underline{\hspace{2em}}>> \underline{\hspace{8em}} \the\yearг.
    \end{flushright}
    Задание на \UNDER{\underline{\smash{\hspace{0.2\textwidth}магистерскую диссертацию\hspace{10.5em}}}}
        {\TINY{(наименование вида работ)}}\\
    Студента \UNDER{\underline{\smash{\hspace{2em}\STUDENTT\hspace{12.8em}}}}{\TINY{(фамилия, имя, отчество)}}\\
    Код кафедры \vspace{1em}\underline{\hspace{2em}10.27\hspace{5em}}\quad\qquad
    Группа \underline{\hspace{3em}САПР-2п1\hspace{4.8em}}\\
    Тема \vspace{0.5em}\underline{\smash{Методы построения маршрутов общественного транспотра на основе\hspace{1.2em}}}\\
    \vspace{0.5em}\hspace*{12mm}\underline{\smash{предпочтения жителей.\hspace{21.2em}}}\\
    \vspace{0.5em}\hspace*{12mm}\underline{\hspace{0.93\textwidth}}\\
    \vspace{1em}Утверждена приказом по ВолгГТУ от <<\LINE{1em}>> \underline{\hspace{4em}}
        \the\year\ г. \textnumero \LINE{7em}\\
    Срок предоставления готовой работы \UNDER{\LINE{17.5em}}{\TINY{(дата, подпись студента)}}\\
    Содержание основной части пояснительной записки\\
    \LINE{\textwidth}\\\LINE{\textwidth}\\\LINE{\textwidth}\\\LINE{\textwidth}\\\LINE{\textwidth}\\
    \LINE{\textwidth}\\\LINE{\textwidth}\\\LINE{\textwidth}\\\LINE{\textwidth}\\\LINE{\textwidth}\\
    \LINE{\textwidth}\\\LINE{\textwidth}\\\LINE{\textwidth}\\
    \newpage
    \thispagestyle{empty}
    \begin{center}
        Перечень графических материалов
    \end{center}
    1) \LINE{0.97\textwidth}\\\LINE{\textwidth}\\
    2) \LINE{0.97\textwidth}\\\LINE{\textwidth}\\
    3) \LINE{0.97\textwidth}\\\LINE{\textwidth}\\
    4) \LINE{0.97\textwidth}\\\LINE{\textwidth}\\
    5) \LINE{0.97\textwidth}\\\LINE{\textwidth}\\
    6) \LINE{0.97\textwidth}\\\LINE{\textwidth}\\
    7) \LINE{0.97\textwidth}\\\LINE{\textwidth}\\
    8) \LINE{0.97\textwidth}\\\LINE{\textwidth}\\
    9) \LINE{0.97\textwidth}\\\LINE{\textwidth}\\
    10) \LINE{0.95\textwidth}\\\vspace{1em}\LINE{\textwidth}\\
    \vspace{1em}Руководитель работы \UNDER{\LINE{12.3em}}{\TINY{(подпись и дата подписания)}}\hspace{2em}
        \UNDER{\underline{\smash{\hspace{1.5em}М.~В.~Щербаков\hspace{1.5em}}}}{\TINY{(инициалы и фамилия)}}\\
    Консультанты по разделам:\\
    \vspace{1em}\UNDER{\LINE{7.9cm}}{\TINY{(краткое наименование раздела)}}\hquad
        \UNDER{\LINE{4.7cm}}{\TINY{(подпись и дата подписания)}}\hquad
        \UNDER{\LINE{4cm}}{\TINY{(инициалы и фамилия)}}\\
    \vspace{1em}\UNDER{\LINE{7.9cm}}{\TINY{(краткое наименование раздела)}}\hquad
        \UNDER{\LINE{4.7cm}}{\TINY{(подпись и дата подписания)}}\hquad
        \UNDER{\LINE{4cm}}{\TINY{(инициалы и фамилия)}}\\
    \vspace{1em}\UNDER{\LINE{7.9cm}}{\TINY{(краткое наименование раздела)}}\hquad
        \UNDER{\LINE{4.7cm}}{\TINY{(подпись и дата подписания)}}\hquad
        \UNDER{\LINE{4cm}}{\TINY{(инициалы и фамилия)}}\\
\end{titlepage}
    % переработать тексты
\tocless\part{Аннотация}
Развитие городов сопряжено с модернизацией существующей сети общественного транспорта. Современные 
коммуникационные технологии позволяют собирать данные о перемещениях людей в городской среде, на основе 
которых можно сделать выводы о предпочтениях людей по перемещению внутри городской среды. Фактически, 
подобные предпочтения можно рассматривать как требования к структуре сети общественного транспорта. 
Имея данные о ежедневных передвижениях людей, мы можем понять реальные предпочтения и потребности людей 
в системе городского транспорта. Следовательно, может быть предложена модификация транспортной сети или 
даже набор возможных альтернатив. Эта модификация отражает реальные потребности людей и сокращает время 
передвижения и повышает общий уровень удовлетворенности. Однако для решением задача анализа 
геопространственных данных необходимо получить набор субоптимальных маршрутов для выбора. Получение 
оптимальных маршрутов сети является повторяющейся процедурой, которая требует вмешательство эксперта.
В данной работе предлагаются методы, позволяющий на основе данных о предпочтениях пользователей 
общественного транспорта формировать схему маршрутов общественного транспорта.

Список ключевых слов: геоданные, транспортная сеть, транспорт, генетические алгоритмы, 
алгоритмы оптимизации, итерационные алгоритмы, \ldots

\tocless\part{Abstract}
Urban development is connected with the existing public transportation network modernization. Modern 
communication technologies make it possible to collect data on the movements of people in the urban 
environment, based on which is possible to draw conclusions about people's preferences on the movement 
inside the urban space. In fact, these preferences may be considered as requirements for the public 
transportation network. Having data about people's everyday movements, we can understand the real 
people preferences and needs in the urban transport system. Hence, as the results of analysis the 
modified transport network (or even a bunch of alternative) can be suggested. This new solutions reflects 
real people needs and reduce transfer time and increase satisfaction level. However the problem of 
geospatial data analysis is need to be solved to get (sub)optimal routes for choosing by authorities. 
Getting optimal routes network is an iterative procedure which requires human (expert) intervention.
The proposed methods allows to create a public transportation routes scheme based on data about the 
public transportation users preferences.

List of keywords: geodata, transport network, transport, genetic algorithms, optimization algorithms,
iterative algorithms, \ldots
    \tableofcontents
    \newpage
    \begin{center}
	ОПРЕДЕЛЕНИЯ, ОБОЗНАЧЕНИЯ И СОКРАЩЕНИЯ
\end{center}

В настоящей работе применяют следующие термины с соответствующими определениями:
\begin{itemize}
    \item ТГЛ -- теория Гинзбурга-Ландау
    \item ГЛ -- Гинзбург-Ландау
    \item БКШ -- Бардин - Купер - Шриффер
\end{itemize}

\newpage
    \chapter*{Введение}
\addcontentsline{toc}{chapter}{Введение}

В последнее время наблюдается повышенный интерес к сверхпроводникам с 
несколькими сверхпроводящими компонентами. Основные ситуации, когда возникают 
множественные сверхпроводящие компоненты: многозонные сверхпроводники второго 
рода \cite{bib:6,bib:7,bib:8,bib:9,bib:10,bib:11}, сверхпроводимость первого 
рода в смеси независимых консервативных конденсатов, таких как прогнозируемая 
сверхпроводимость в атомарном водороде и богатых водородом сплавов 
\cite{bib:12.1,bib:12.2,bib:13,bib:14} и сверхпроводников с другим типом 
симметрии, отличной от поперечно-волновой симметрии. Интерес к исследованию 
вихревого состояния обусловлен широкими потенциальными возможностями 
применения сверхпроводников в современной микроэлектронике и энергетике, а 
также интересом к самой физике процессов происходящих в смешанном состоянии 
сверхпроводников. Развитие нанотехнологии и открытие новых сверхпроводящих 
соединений (в частности, высокотемпературных сверхпроводников) стимулировали 
новые теоретические и экспериментальные исследования смешанного состояния. 

В работе проведено исследование появления сверхпроводимости 1,5-го рода с 
акцентом на случай многополосной сверхпроводимости, демонстрируя сохранение 
этого типа сверхпроводимости в присутствии различных видов межкомпонентных 
соединений (например, межзонное джозефсоновской связи, смешанных градиентных 
связей, связи типа плотность-плотность, и других видов).

\textbf{Актуальность.}

Главным фактором в изучении явления сверхпроводимости является временные 
затраты на проведение модельных экспериментов. Поэтому для решения 
поставленных задач нужно использовать более быстрые и эффективные алгоритмы.

\textbf{Методы исследования.}

Достоверность результатов обеспечена оптимальным выбором физических моделей, 
учитывающих основные свойства исследуемых систем, и адекватным выбором метода 
численного моделирования.

\textbf{Научная новизна.}

С целью использования градиентных алгоритмов минимизации (типа LBFGS), для 
которых знание градиента минимизируемой функции в явном виде является 
желательным условием, был использован метод автоматического дифференцирования 
(библиотека CppAD). Для минимизации функционала свободной энергии в модели 
Гинзбурга -- Ландау впервые был использован алгоритм LBFGS.

\textbf{Научная и практическая значимость.}

Используемый алгоритм Бройдена -- Флетчера -- Гольдфарба -- Шанно сокращает
временные затраты на минимизацию функционала Гинзбурга -- Ландау по сравнению 
с другими методами.

\newpage
    \chapter{Явление сверхпроводимости}

\section{Основные характеристики сверхпроводящего состояния}

Согласно теории Гинзбурга-Ландау обычно сверхпроводник вблизи \( T_c \) 
описывается одним комплексным параметром поля. Физика этих систем определяется 
двумя фундаментальными масштабными длинами, глубиной проникновения магнитного 
поля \( \lambda \) и длиной когерентности \( \xi \), а также коэффициентом 
\( \kappa \), который определяет реакцию на внешнее поле, разделяя их на две  
категории следующим образом; сверхпроводники первого рода, где 
\( \kappa < 1/\sqrt{2} \) и второго рода, где \( \kappa > 1/\sqrt{2} \) 
\cite{bib:3}.

Теория БКШ дает исчерпывающее описание сверхпроводящих свойств материала во 
всём температурном интервале от \( 0 \) до \( T_c \), но является сложной с 
математической точки зрения. Поэтому часто физики прибегают к другому, 
относительно более простому способу анализа сверхпроводящего состояния -- 
Теория Гинзбурга–Ландау, которая прекрасно описывает, качественно и 
количественно, поведение сверхпроводника, но работает только в ограниченном 
интервале вблизи критической температуры.

ТГЛ основывается на теории фазовых переходов 2-го рода. В этой теории, наряду 
с критической температурой, длиной когерентности и лондоновской глубиной 
проникновения, вводится еще одна характеристика, -- параметр порядка. С 
точностью до некоторого коэффициента пропорциональности можно считать, что 
модуль параметра порядка -- это энергетическая щель в теории БКШ. Параметр 
порядка равен нулю при \( T = T_c \) и принимает максимальное значение, когда 
температура достигла абсолютного нуля. Отметим, что есть и иная трактовка 
физического смысла параметра порядка: квадрат его модуля определяет 
концентрацию куперовских пар. Параметр порядка играет ключевую роль в теории 
Гинзбурга–Ландау. Через него выражается свободная энергия сверхпроводника. 
\cite{bib:net}

Задача о сосуществование сверхпроводящего и магнитного порядков привлекает 
внимание исследователей на протяжение последних десятилетий. Можно выделить 
два основных механизма взаимодействия сверхпроводящего параметра порядка с 
магнитной подсистемой: электромагнитный механизм, когда куперовские пары 
взаимодействуют с магнитным полем индуцированным ферромагнетиком (впервые 
такое взаимодействие было рассмотрено В.~Л.~Гинзбургом в 1956 
\cite{ginzburg}): и обменное взаимодействие магнитных моментов с куперовскими 
парами \cite{buzdin,bulaev}. Если ферромагнетик и сверхпроводник разделены 
тонкой диэлектрической прослойкой, то эффект близости подавлен и единственным 
фактором, определяющим взаимодействие подсистем, является магнитное поле, 
создаваемое неоднородным распределением намагниченности в ферромагнетике.
Впервые смешанное состояние с неполным эффектом Мейсснера-Оксенфельда 
(фаза Шубникова) в сверхпроводниках, находящихся во внешнем магнитном поле, 
было обнаружено группой Л.~В.~Шубникова в 1937 году \cite{shubnikov}. В 1957 
году А.~А.~Абрикосов, основываясь на теории Гинзбурга-Ландау 
\cite{ginzburg-landau}, показал, что в массивных сверхпроводниках второго рода 
внешнее магнитное поле проникает в сверхпроводник в виде нитей магнитного 
потока (вихрей Абрикосова) \cite{abrikosov}. можно рассматривать как 
структурную единицу смешанного состояния.

\section{Сверхпроводимость 1-го и 2-го рода}

Вышеуказанные обстоятельства приводят к тому, что, в сильном внешнем магнитном 
поле, сверхпроводники первого рода обычно имеют тенденцию к минимизации 
энергии на границе между нормальной и сверхпроводящей фазой, что приводит к 
образованию нормальной фазы, которая часто имеют слоистую структуру 
\cite{bib:4}. 

Несмотря на то что теория Гинзбурга–Ландау является феноменологической, то 
есть она не объясняет причины возникновения явления, которое она описывает, с 
её помощью был получен ряд важных результатов. Применив эту теорию, ее авторы 
вычислили поверхностную энергию, возникающую на границе сверхпроводника и 
нормального металла в присутствии внешнего магнитного поля. Оказалось, что 
результат зависит от безразмерной величины, называемой параметром 
Гинзбурга–Ландау \( \kappa \): \( \kappa = \lambda/\xi \) (отношение 
лондоновской глубины проникновения к длине когерентности). Из расчетов 
следовало, что при \( \kappa < 1/\sqrt{2} \) поверхностная энергия оказывается 
положительной. Для сверхпроводника цилиндрической формы, ось которого 
параллельна силовым линиям магнитного поля, данный результат означал, что 
переход в нормальное состояние происходит моментально, как только индукция 
магнитного поля превышает некоторое критическое значение \( B_c \) для данной 
температуры (рисунок~\ref{img:01}). В принципе, ничего нового Гинзбург и 
Ландау не получили, они лишь теоретически подтвердили хорошо известный уже на 
тот момент экспериментальный факт поведения сверхпроводников.

\begin{figure}[h!]
    \center
    \includegraphics[width=.8\textwidth]{img_01}
    \caption{Фазовая диаграмма состояния сверхпроводников 1-го (а) и 
        2-го (б) рода, показывающая, как меняются состояния сверхпроводника 
        при изменении температуры и индукции внешнего магнитного поля. В 
        мейсснеровском состоянии силовые линии магнитного поля не могут 
        проникнуть в вещество. Смешанное или вихревое состояние означает 
        сосуществование сверхпроводимости и нормальных не сверхпроводящих 
        тонких нитей, вытянутых вдоль линий магнитного поля. Такие нити 
        называют вихрями Абрикосова, или квантовыми вихрями.}
    \label{img:01}
\end{figure}

Советский физик Николай Заварицкий, исследуя тонкие сверхпроводящие пленки, 
обнаружил, что их поведение в магнитном поле не согласуется с предсказаниями 
теории Гинзбурга–Ландау. Чтобы понять причину расхождения, Алексей Абрикосов, 
основываясь на теории Гинзбурга–Ландау, решил рассмотреть случай, когда 
поверхностная энергия является отрицательной, -- иными словами, попытаться 
понять картину поведения сверхпроводника в магнитном поле с 
\( \kappa > 1/\sqrt{2} \).

Расчёты показали, что пока индукция магнитного поля не превосходит нижнее 
критическое значения поля \( B_{c1} \) при фиксированной температуре, 
сверхпроводник находится в состоянии мейсснера. После того как индукция 
магнитного поля стала больше \( B_{c1} \), сверхпроводник начинают пронизывать 
своеобразные нити микронных размеров, вытянутые вдоль силовых линий внешнего 
поля. Чем больше индукция поля, тем больше ниток будет в сверхпроводнике. 
Абрикосов установил, что эти образования представляют собой вихри (теперь они 
называются абрикосовскими), ядра которых являются не сверхпроводящими, 
нормальными, с размером порядка длины когерентности \( \xi \), а вокруг них 
протекают циркулирующие сверхпроводящие токи, которые экранируют нормальную 
область вихря (ширина области экранировки равна лондоновской глубине 
проникновения \( \lambda \)). Кроме того, в ходе вычислений обнаружилось, что 
вихри несут в себе как бы одну силовую линию внешнего магнитного поля, или 
квант магнитного потока, флюксоид 
\( \Phi_0 = hc/2e \simeq 2.07\cdot10^{-7} \text{ Гс}\cdot\text{см}^2 \). Вихри 
формируют в сверхпроводнике треугольную решетку, образуя смешанное (оно же 
вихревое) состояние (рисунок~\ref{img:01}).

Если при заданной температуре продолжить усиливать магнитное поле, то при 
верхнем критическом значении \( B_{c2} \) вихрей станет настолько много, что 
их ядра начнут перекрываться, и они заполнят весь объем сверхпроводника, 
переводя его в нормальное состояние (рисунок~\ref{img:01}). \cite{bib:net}

В сверхпроводниках второго рода для того чтобы напряженность внешнего 
магнитного поля, необходимая для образование вихревых возбуждений, была 
энергетически выгодной, нужно чтобы \( B_{c1} \) было меньше, чем 
термодинамическое значение магнитного поля \( B_{ct} \) (поле, энергетическая 
плотность которого равна энергии конденсации сверхпроводника, т.е. области, в 
которой равномерное сверхпроводящее состояние становится термодинамически 
неустойчивым); В сверхпроводниках первого рода напряженность поля требует 
создание вихревого возбуждения большего, чем критическое термодинамическое 
значение магнитного поля, т.е. вихри не могут образовываться. Можно выделить 
также специальный <<нульмерный>> пограничный случай, когда \( \kappa \) имеет 
критическое значение точно на границе первого/второго рода, что в самом общей 
модели ГЛ соответствует \( \kappa = 1/\sqrt{2} \). В этом случае вихри не 
взаимодействуют в теории Гинзбурга-Ландау.

В двухщелевых сверхпроводниках второго рода сверхпроводящие компоненты 
происходят из электронного куперовского спаривания в различных зонах 
\cite{bib:6}. Поэтому эти конденсаты не могут быть независимо сохраняющимися.

В случае сверхпроводников первого рода две сверхпроводящие компоненты были 
предсказаны, происходящие из электронного и протонного куперовского спаривания 
в атомарном водороде или богатых водородом сплавов. В прогнозируемом 
жидкометаллическом дейтерии и сплавов богатых дейтерием, было предсказано 
существование электронной проводимости на сверхвысоких давлениях с дейтронной 
конденсацией  \cite{bib:12.1,bib:12.2,bib:13,bib:14}. Поскольку электроны не 
могут быть преобразованы в протон или дейтрон с независимо сохраняющейся 
конденсацией, и, следовательно, в эффективной модели Джозефсона 
межкомпонентная связь запрещена на основаниях симметрии. Этот эффект в 
настоящее время является предметом возобновлённых экспериментальных 
исследований.

Это большое разнообразие систем вызывает необходимость истолковать и 
классифицировать возможные магнитные отклики многокомпонентных 
сверхпроводников. В различных источниках обсуждалось, что в многокомпонентных 
системах, где магнитный отклик гораздо сложнее, чем в обычных системах, что 
разделение на сверхпроводники первого/второго рода не является достаточным для 
классификации. Скорее всего существует отдельный сверхпроводящий режим, при 
котором вихри имеют дальнодействующее притяжение, близкодействующее 
отталкивание и форму вихревых узлов в области двухкомпонентного эффекта 
Мейсснера \cite{bib:1,bib:2}. Последние экспериментальные работы 
\cite{bib:16,bib:17} выдвинули предположение, что это состояние реализуется в 
двухщелевом материале \( MgB_2 \), что вызвало растущий интерес к этой теме. В 
частности были подняты вопросы по поводу того, что сверхпроводимость типа 1,5 
(как это было названо Мощалковым и другими в \cite{bib:16}) возможна даже в 
случае различных неисчезающих связей (например, внутренняя джозефсоновская 
связь, связь смешанных градиентов и т.д.) сверхпроводящими компонентами в 
разных диапазонах. \cite{bib:main}

\section{Сверхпроводимость 1,5-го рода и двухщелевые сверхпроводники}

В 2001 году в дибориде магния \( MgB_2 \) была открыта сверхпроводимость с 
неожиданно высокой критической температурой 39 К. Применяя различные 
экспериментальные техники, учёные установили, что большое значение \( T_c \) 
достигается за счет наличия в \( MgB_2 \) не одной энергетической щели, а 
двух. То есть в сверхпроводящем дибориде магния присутствует как бы два сорта 
куперовских пар. Их взаимодействие и обеспечивает высокую \( T_c \). Важно 
отметить, что у каждого сорта электронных пар есть свой размер, или своя длина 
когерентности. При этом диборид магния имеет лишь одну величину лондоновской 
глубины проникновения. \cite{bib:net}

Каждой из двух энергетических щелей \( \Delta_1 \) и \( \Delta_2 \) 
соответствует своя длина когерентности \( \xi_1 \), \( \xi_2 \) и лондоновская 
глубина проникновения \( \lambda_1 \), \( \lambda_2 \). Если применить 
критерий Абрикосова для \( MgB_2 \), то получится, что для первой щели 
\( \kappa = \lambda_1 / \xi_1 \approx 0,66 < 1/\sqrt{2} \), а для второй -- 
\( \kappa = \lambda_2 / \xi_2 \approx 3,68 > 1/\sqrt{2} \). Получается, что в 
дибориде магния одновременно <<живут>> две сверхпроводимости -- первого и 
второго рода.

В 2003 году в журнале <<Physical Review B>> появилась статья ученых из России, 
Швейцарии и США <<Vortex structure in \( MgB_2 \) single crystals observed by 
the Bitter decoration technique>>, в которой они сообщали о наблюдении 
абрикосовской решетки в чистом монокристалле \( MgB_2 \) в слабом магнитном 
поле. Эксперимент показал, что диборид магния, несмотря на наличие в нем двух 
энергетических щелей, можно отнести к сверхпроводникам второго рода. 
Результат подтверждался снимком треугольной вихревой решетки. Вихри 
распределены равномерно -- что, по мнению исследователей, доказывает факт 
наличия в \( MgB_2 \) сверхпроводимости второго рода.

\begin{figure}[h!]
    \center
    \includegraphics[width=.6\textwidth]{img_spc}
    \caption{Изображения вихревых структур в сверхпроводниках при 
        \( T = 4,2 K \) во внешнем магнитном поле напряженностью 1 эрстед.
        Вихри в монокристалле \( MgB_2 \) (a) распределены неравномерно, тогда 
        как в монокристалле \( NbSe_2 \) (b) они образуют абрикосовскую 
        треугольную решетку. Длина масштабной линейки 10 мкм.}
    \label{img:spc}
\end{figure}

Однако относительно недавно группа исследователей из Бельгии и Швейцарии 
опубликовала в Архиве препринтов результаты своего эксперимента под говорящим 
заголовком <<Type-1.5 Superconductors>>. Во внешнем магнитном поле 
напряженностью 1 эрстед (в 4 раза слабее, чем в эксперименте 2003 года) они 
получили для сверхпроводящего чистого монокристалла диборида магния необычную 
картину вихревой решетки -- с неравномерным распределением вихрей 
(рисунок~\ref{img:spc}а). Эта нерегулярность связана с тем, что, как 
предсказывает теория, взаимодействие магнитных вихрей должно напоминать 
поведение межмолекулярных сил: на близких расстояниях вихревые структуры 
отталкиваются, на далеких -- начинают притягиваться. Такое поведение вихрей 
авторы статьи считают главной особенностью сверхпроводимости <<полуторного>> 
рода. \cite{bib:superconductors}

Возможность нового типа сверхпроводимости, отличного от первого и второго рода 
в многокомпонентных системах \cite{bib:1,bib:2} основана на следующих 
соображениях. Краевая задача уравнения Гинзбурга-Ландау в присутствии 
циркуляции фазы сводится к одномерной задачи в целом. Кроме того, как указано в 
\cite{bib:1,bib:2}, в общей двухкомпонентной модели есть три фундаментальные 
масштабные длины, которые указывают на невозможно параметризовать модель с 
точки зрения одного безразмерного параметра \( \kappa \). В случае, когда 
конденсаты не связаны межзонной джозефсоновской связью, при условии, что 
векторный потенциал этих масштабных длин представлен двумя независимыми 
величинами: длиной когерентности и глубиной проникновения магнитного поля. В 
противоположность этому, в случае, когда конденсаты объединяются межзонными 
джозефсоновскими условиями, можно не увидеть различий между длинами 
когерентности и отнести их к различным конденсатам. Тем не менее, в этом 
случае колебания плотности также могут обладать двумя основными 
пространственными длинами\cite{bib:2}, в отличие от однокомпонентных теорий. 
В \cite{bib:1,bib:2} вихревых решениях в двухкомпонентной теории были найдены 
компоненты, которые имеют немонотонное вихревое взаимодействие, с 
доминирующими частями отвечающие за дальнодействующее взаимодействие, 
отталкивающей частью близкодействия и частью отвечающею за электромагнитное 
взаимодействие. Важным обстоятельством, которое было продемонстрировано, 
является то, что эти вихри термодинамически стабильны, несмотря на 
существование притягивающего хвоста во взаимодействии.

\begin{figure}[h!]
    \center
    \includegraphics[width=.8\textwidth]{1-01}
    \caption{Сравнение фазовых диаграмм магнитных фаз чистых сверхпроводников
        первого, второго и полуторного рода при нулевой температуре. В 
        полумейсснеровском режиме макроскопическое разделение фаз в 
        двухкомпонентном состоянии Мейсснера и вихревых скоплений, где 
        один из режимов плотности подавляется внутренним перекрытием.}
    \label{fig:1}
\end{figure}

Немонотонный межвихревой потенциал взаимодействия должен привести к 
образованию вихревых скоплений в слабомагнитном поле, погруженного в 
безвихревую область, эффект упомянутый в \cite{bib:1} как 
<<полумейсснеровское состояние>>. На рисунке~\ref{fig:1} схематически 
показана фазовая диаграмма сверхпроводника 1,5-го рода.

Если вихри образуют кластеры, то нельзя использовать обычный одномерный 
аргумент относительно энергии сверхпроводника в нормальном состоянии границы 
раздела для классификации магнитного отклика системы. Прежде всего, энергия 
на вихрь, в таком случае, зависит от того, находится ли вихрь в кластере или 
нет: т.е. формирование единого изолированного вихря может быть энергетически 
невыгодным, в то время как формирование вихревых кластеров выгодно, потому что 
в кластере, где помещены вихри, создаётся минимум потенциала взаимодействия, 
энергия на квант потока меньше, чем для изолированного вихря (термодинамически 
немонотонный потенциал взаимодействия двух вихрей предусматривает, что 
наименьшей энергией, на квант потока, будет в случае равномерной сетки с шагом 
равным минимуму межвихревого потенциала двух тел).

Таким образом, помимо энергии вихря в кластере, появляется дополнительная 
энергическая характеристика, связанная с границей кластера. Другими словами, в 
этой ситуации, для определения магнитного отклика системы недостаточно 
изучения краевой задачи и задачи одиночного вихря, в отличии от системы 
отдельных составляющих. В кластере система стремится к минимуму граничной 
энергии (так же, как и в сверхпроводниках 1-го рода), в то время нарушение 
структуры одноквантового вихря внутри кластера (аналогично сверхпроводимости 
второго рода с отрицательной энергией межфазного взаимодействия). Таким 
образом, увеличение магнитного поля происходит посредством фазового перехода 
первого рода. Магнитная фаза отлична от вихря в мейсснеровском состоянии, затем 
возникает макроскопическое фазовое распределение в двухкомпонентной области 
Мейсснера и вихревых кластерах, где один из режимов подавляется основным 
перекрытием. \cite{bib:main}

\newpage
    \chapter{Теоретический анализ модели}

\section{Функционал свободной энергии}

Сверхпроводимость 1,5-го рода изучается с помощью следующего двухкомпонентного 
функционала свободной энергии Гинзбурга-Ландау:
\begin{gather}
    F = \frac{1}{2}(D\psi_1)(D\psi_1)^* + \frac{1}{2}(D\psi_2)(D\psi_2)^* - 
        \nu Re\left( (D\psi_1)(D\psi_2)^* \right) + \nonumber \\
        + \frac{1}{2}\left(\nabla\times\vec{A}\right)^2 + F_p
    \label{eq:1}
\end{gather}

Здесь \( D = \nabla + ie\vec{A} \) и \( \psi_a = |\psi_a|e^{i\theta_a} \), 
\( a = 1,2 \), представляет собой две сверхтекучих компоненты, которые в 
двухщелевом сверхпроводнике соответствуют двум сверхтекучим плотностям в 
в различных диапазонах. Слагаемое \( F_p \) содержит в текущем анализе 
произвольный набор не градиентных членов.

Особая форма двухкомпонентной модели ГЛ точно выведенная в
\cite{bib:8,bib:9,bib:10} для двухщелевых сверхпроводников представлена в виде:
\begin{gather}
    F = \frac{1}{2}(D\psi_1)(D\psi_1)^* + \frac{1}{2}(D\psi_2)(D\psi_2)^* - 
        \nu Re\left\{ (D\psi_1)(D\psi_2)^* \right\} + \nonumber \\
        + \frac{1}{2}\left(\nabla\times\vec{A}\right)^2 + 
        \alpha_1|\psi_1|^2 + \frac{1}{2}\beta_1|\psi_1|^4 + 
        \alpha_2|\psi_2|^2 + \frac{1}{2}\beta_2|\psi_2|^4 + \nonumber \\
        + \eta_1|\psi_1||\psi_2| \cos(\theta_1-\theta_2) + 
        \eta_2|\psi_1|^4|\psi_2|^2
    \label{eq:2}
\end{gather}

Первые два слагаемых представляют стандартный градиентный член 
Гинзбурга-Ландау, второе слагаемое представляет смешанные градиентные 
взаимодействия, которые появляются в двухщелевых сверхпроводниках с примесным 
рассеянием \cite{bib:8,bib:9}. Следующее член является плотностью энергии 
магнитного поля, а остальные слагаемые представляют эффективный потенциал. 
Здесь же отметим, что \( \alpha_1 \) и \( \alpha_2 \) могут менять знак 
при различных температурах. Режим, где значение \( \alpha_1 \) положительно, 
в то время \( \alpha_2 \) является отрицательным, соответствует ситуации, 
когда одина из группы не имеет собственной сверхпроводимости, но, тем не менее 
имеет некоторые сверхтекучие плотности из-за межзонного туннелирования 
Джозефсона, которая представлена 
\( \eta|\psi_1||\psi_2|\cos(\theta_1-\theta_2) \) слагаемым. Поведения 
сверхпроводника 1,5-го рода в этом режиме был исследован в \cite{bib:2}. В 
данной работе сосредоточимся в основном на ситуации когда обе зоны являются 
активными, то есть при \( \alpha_{1,2} < 0 \). Для общности добавим слагаемое 
более высокого порядка связи \( \eta_2|\psi_1|^2|\psi_2|^2 \). Также 
рассмотрим случай независимо сохраняющихся конденсатов, где третий и девятый 
члены в \eqref{eq:2} запрещены на основании симметрии, то есть 
\( \nu = \eta_1 = 0 \). Переход между текущими единицами и общепринятыми 
представлен в Приложении Б.

Точно выведенная модель ГЛ \eqref{eq:2} требует малость полей \( |\psi_a| \). 
Однако это не требует в принципе от \( \alpha_a \) менять знак при той же 
температуре. Кроме того, как и в случае однокомпонентной теории ГЛ модель 
\eqref{eq:2} даёт во многих случаях приемлемую картину в низкотемпературном 
режиме. Фактически, анализ может в некоторых случаях дать качественную картину 
для случая, когда одно из полей не обладает эффективным потенциалом ГЛ-типа, 
так как режим, где одна из зон в лондоновском приближении (т.е. она не 
обладает эффективным потенциалом ГЛ, но небольшое ядро вихря моделируется 
резкой границей отсечки) может быть восстановлена из анализа, как предельный 
случай. Из анализа представленного ниже будет понятна видна поддержка 
сверхпроводимости 1,5-го рода.

\section{Вихревая асимптотика}

Ключом к пониманию взаимодействия хорошо разделенных вихрей является анализ 
при больших \( r \) асимптотического вихревого решения. Проанализируем эту 
проблему в контексте общей двухкомпонентной модели ГЛ \eqref{eq:1}, чью 
свободную энергия можно представить в виде
\begin{equation}
    F = \frac{1}{2}\left( D_i \psi_1 \right)^{*} D_i \psi_1 + 
        \frac{1}{2}\left( D_i \psi_2 \right)^{*} D_i \psi_2 + 
        \frac{1}{2}\left( \partial_1 A_2 - \partial_2 A_1 \right)^2 + F_p
    \label{eq:3}
\end{equation}
где \( F_p \) содержит все не градиентные члены. Эта свободная энергия 
соответствует \eqref{eq:2} в случае \( \nu = 0 \). Точная форма \( F_p \) в 
данном случае не является решающим фактором для анализа. При калибровочной 
инвариантности, это может зависеть только через \( |\psi_1|, |\psi_2| \) и 
(если конденсаты не являются независимо сохраняющимися) на 
\( \theta_1 - \theta_2 \). Будем считать, что \( F_p \) принимает минимальное 
значение (которое должно быть приведено к 0), когда 
\( |\psi_1| = u_1 > 0, |\psi_2| = u_2 > 0 \) и \( \theta_1 - \theta_2 = 0 \).
Таким образом, либо нет связи фаз (\( F_p \) не зависит от 
\( \theta_1 - \theta_2 \)) и выбор \( \theta_1 - \theta_2 = 0 \) произволен, 
или связь фаз является таковой, что стимулирует её синхронизацию. 

Уравнения поля получаются из \( F \), при условии, что общая свободная энергия 
\( E = \int F dx_1 dx_2 \) является стационарной по отношению ко всем 
изменениям \( \psi_1, \psi_2 \) и \( A_i \). Обычный расчёт даёт 
\begin{gather}
    D_i D_i \psi_a = 2\pder{F_p}{\psi_a^{*}}
    \label{eq:4} \\
    \partial_i \left( \partial_i A_j - \partial_j A_i \right) = 
        e\sum\limits_{a=1}^{2}\mathrm{Im} \left( \psi_a^* D_j \psi_a \right)
    \label{eq:5}
\end{gather}
Решение этой пары связанных нелинейных дифференциальных уравнений в частных 
производных можно представить в виде
\begin{gather}
    \psi_a = f_a(r)e^{i\theta} \nonumber \\
    (A_1, A_2) = \frac{a(r)}{r}(-\sin\theta, \cos\theta)
    \label{eq:6}
\end{gather}
где \( f_1, f_2, a \) вещественные функции профиля. Примем во внимание, что 
в некоторых случаях смешанные градиентные слагаемые выступают не за 
осесимметричное решение. Рассмотрим только аксиально-симметричные вихри. 
Поля, в пределах указанного выше подхода, удовлетворяют уравнениям поля, если и 
только если функция профиля \( f_1(r), f_2(r), a(r) \) удовлетворяют взаимной 
системе дифференциальных уравнений
\begin{gather}
    f''_a + \frac{1}{r} f'_a - \frac{1}{r^2}(1+ea)^2 f_a = 
        \left. \pder{F_p}{|\psi_a|} \right|_{(u_1, u_2, 0)}
    \label{eq:7} \\
    a'' - \frac{1}{r} a' - e(1+ea)(f_1^2+f_2^2) = 0
    \label{eq:8}
\end{gather}
Потребуем чтобы вихревое решение имело поведение на границе вида 
\( f_a(r) \rightarrow u_a \), \( a(r) \rightarrow -1/e \) при
\( r \rightarrow \infty \). Так, для больших значениях \( r \) величины
\begin{equation}
    \epsilon_a(r) = f_a(r) - u_a, \quad
    \alpha(r) = a(r) + \frac{1}{e}
    \label{eq:9}
\end{equation}
малы и удовлетворяющие линеаризации \eqref{eq:7},\eqref{eq:8} относительно 
\( (u_1, u_2, -1/e) \). То есть, при больших \( r \),
\begin{gather}
    \epsilon''_a + \frac{1}{r} \epsilon'_a = \sum\limits_{b=1}^{2}
        \mathcal{H}_{ab} \epsilon_b
    \label{eq:10} \\
    \alpha'' - \frac{1}{r} \alpha' - e^2(u_1^2 + u_2^2 )\alpha = 0
    \label{eq:11}
\end{gather}
где \( \mathcal{H} \) является матрицей Гессе \( F_p(|\psi_1|, |\psi_2|, 0) \) 
и его минимум
\begin{equation}
    \mathcal{H}_{ab} = \left. \pcder{F_p}{|\psi_a|}{|\psi_b|} 
        \right|_{(u_1, u_2, 0)}
    \label{eq:12}
\end{equation}
Так \( \alpha \) асимптотически отделяется от \( \epsilon_1, \epsilon_2\) и 
сразу видно, что
\begin{equation}
    \alpha(r) = q_0 r K_1(\mu_A r), \quad
    \mu_a = e\sqrt{u_1^2 + u_2^2}
    \label{eq:13}
\end{equation}
где \( K_n \) обозначает \( n \)-ую модифицированную функцию Бесселя второго 
рода, и \( q_0 \) неизвестная действительная постоянная. Таким образом
\begin{equation}
    \vec{A} \sim \left( -\frac{1}{er} + q_0 K_1(\mu_A r) \right)
        (-\sin\theta, \cos\theta)
    \label{eq:14}
\end{equation}
Так, для всех \( n \), 
\begin{equation}
    K_n(s) \sim \sqrt{\frac{\pi}{2s}}e^{-s} \text{ as } s \rightarrow \infty
    \label{eq:15}
\end{equation}
отсюда вытекает, что магнитное поле затухает экспоненциально в зависимости от 
\( r \), с масштабной величиной (глубиной проникновения)
\begin{equation}
    \lambda \equiv \frac{1}{\mu_A} = \frac{1}{e\sqrt{u_1^2 + u_2^2}}
    \label{eq:16}
\end{equation}

С другой стороны, \eqref{eq:10} представляет, в общем, пару связанных 
обыкновенных дифференциальных уравнений для \( \epsilon_1, \epsilon_2 \). Так 
как \( (u_1, u_2, 0 ) \) является минимумом от 
\( F_p(|\psi_1|, |\psi_2|, \theta_1 - \theta_2) \), где матрица Гессе является  
симметричной и положительно определенной действительной матрицей размера 
\( 2\times2 \). Следовательно, её собственные числа \( \mu_1^2, \mu_2^2\), 
допустим вещественны и положительны, и тогда её собственные векторы 
\( v_1, v_2 \) формируют ортонормированный базис на \( \mathbb{R} \). Расширяя 
базис \( v_1, v_2 \)
\begin{equation}
    \epsilon(r) = \chi_1(r) v_1 + \chi_2(r) v_2
    \label{eq:17}
\end{equation}
видим, что \( \chi_1, \chi_2 \) удовлетворяет несвязанной паре обычных 
дифференциальных уравнений
\begin{equation}
    \chi''_a + \frac{1}{r}\chi'_a = \mu_a^2 \chi_a
    \label{eq:18}
\end{equation}
откуда
\begin{equation}
    \chi_a(r) = q_a K_0(\mu_a r)
    \label{eq:19}
\end{equation}
где \( q_1, q_2 \) некоторые неизвестные константы. Так как \( v_1, v_2 \) 
являются ортонормированными, то существует угол \( \Theta \), называемый углом 
смешивания, такой, что собственные векторы \( \mathcal{H} \) являются
\begin{equation}
    v_1 = \left( \begin{array}{c}
        \cos\Theta \\
        \sin\Theta
    \end{array} \right), \quad
    v_2 = \left( \begin{array}{c}
        -\sin\Theta \\
        \cos\Theta
    \end{array} \right)
    \label{eq:20}
\end{equation}
Так, при больших \( r \) поля плотности ведут себя как 
\begin{gather}
    \psi_1 \sim \left[ u_1 + q_1\cos\Theta K_0(\mu_1 r) - 
        q_2\sin\Theta K_0(\mu_2 r) \right]e^{i\theta} \nonumber \\
    \psi_2 \sim \left[ u_2 + q_1\sin\Theta K_0(\mu_1 r) - 
        q_2\cos\Theta K_0(\mu_2 r) \right]e^{i\theta}
    \label{eq:21}
\end{gather}
где, ещё раз, \( K_0 \) -- функции Бесселя.

Из этого анализа следует, что:
\begin{enumerate}
    \item В целом есть три фундаментальные масштабные длины в задаче (в 
        отличие от двух масштабных длин в однокомпонентной теории 
        Гинзбурга-Ландау), которые проявляются в вихревых асимптотиках, а 
        именно \( 1/\mu_A, 1/\mu_1 \) и \( 1/\mu_2 \).
    \item Они получаются из вакуумного среднего \( u_a \) в \( |\psi_a| \) 
        (в случае с \( 1/\mu_A \)) и из собственных значений 
        \( \mathcal{H} \), матрица Гессе \( F_p \) (т.е. основного состояния).
    \item \( 1/\mu_{A} \) может быть интерпретирована как лондоновская глубина 
        проникновения магнитного поля.
    \item Однако если угол смешивания \( \Theta \) не является кратным  
        \( \pi/2, 1/\mu_1 \) и тогда \( 1/\mu_2 \) не может быть истолкована 
        как длина когерентности \( \psi_1, \psi_2 \) в обычном смысле. Это 
        потому, что нормальные режимы теории поля, близкие к вакууму не 
        \( |\psi_a| - u_a \), а скорее
        \[ 
            \chi_1 = (|\psi_1| - u_1)\cos\Theta - (|\psi_2| - u_2)\sin\Theta 
        \]
        \[ 
            \chi_2 = (|\psi_1| - u_1)\sin\Theta - (|\psi_2| - u_2)\cos\Theta 
        \]
        получаемые поворотом на угол смешивания \( \Theta \), который также 
        определяется из \( \mathcal{H} \). Поэтому в целом (например, в 
        присутствии межкомпонентной джозефсоновской связи) для однопоточного
        квантово-осесимметричного вихря, восстановление обоих полей
        \( \psi_a \) в очень большом диапазоне будет происходить по тому же 
        экспоненциальному закону, который устанавливает наименьшее из масс 
        \( \mu_1, \mu_2 \); Следует использовать представление в терминах 
        \( \chi_{1,2} \), которые будут должным образом представлять два 
        пространственных масштаба, связанных с восстановлением плотности.
    \item Этот анализ говорит нам только о вихревой структуре при больших
        \( r \). Это не дает прямую информацию о ядре вихря, чтобы 
        количественно понять природу вихревых взаимодействий на переходных 
        и коротких расстояниях.
\end{enumerate}

Поскольку калибровочное поле является посредником силы отталкивания между 
вихрями, а поля конденсатов посредником силы притяжения, ясно, что можно 
считать из приведенного выше анализа условие, при котором межвихревая сила 
является притягивающей на большом расстоянии: потребуем, чтобы \( 1/\mu_A \) 
являлось не самым большим из трёх пространственных масштабных длин, или, 
более явно, чтобы (по крайней мере) одно из собственных значений 
\( \mathcal{H} \) должно быть меньше, чем \( \mu_A^2 = e^2(u_1^2 + u_2^2) \). 
Можно предсказать явную формулу для дальнодействующего двухвихревого 
потенциала взаимодействия, с помощью формализма точечного вихря \cite{bib:19} 
(краткий обзор метода приведён в Приложении Б). Это основывается на 
наблюдении, что недалеко от его ядра, поле вихря совпадают с гипотетической 
точечной частицей в линейной теории с двумя полями Клейна-Гордона 
(\( \chi_1 \) и \( \chi_2 \) указанных выше) массы и векторного поля (\( A \)) 
массы \( \mu_A \). Точечная частица несёт монопольные скалярные заряды 
\( 2\pi q_1 \) и \( 2\pi q_2 \), и магнитный дипольный момент \( 2\pi q_0 \). 
Две таких гипотетических частицы, удерживаемые на расстоянии \( r \) 
испытывают потенциал взаимодействия
\begin{equation}
    V(r) = 2\pi\left[ q_0^2 K_0(\mu_A r) - q_1^2 K_0(\mu_1 r) - 
        q_2^2 K_0(\mu_2 r) \right]
    \label{eq:22}
\end{equation}

Эта формула воспроизводит предсказанное выше объяснение: взаимодействие на 
больших расстояниях будет притягивающим, если (по крайней мере) один из 
\( \mu_1, \mu_2 \) меньше чем \( \mu_A \). 

Можно задаться вопросом: является приближённая линеаризация при малых 
параметрах \( \alpha(r), \chi_1(r), \chi_2(r) \) оправданной? Строгий анализ 
однокомпонентной модели \cite{bib:20} показывает, что если любая из масс 
скалярного режима, скажем превышает \( 2\mu_A \), то квадратичные члены в 
\( \alpha \) становятся сопоставимыми при больших \( r \) с линейными членами 
\( \chi_2 \), так что уравнение для \( \chi_2 \) должно включать в себя 
дополнительные условия. В этом случае \( \chi_2 \) убывает как 
\( K_0(\mu_A r)^2 \), а не как \( K_0(\mu_2 r) \). Следует отметить, что если 
\( \mu_1 > 2\mu_A \), то главное слагаемое в \eqref{eq:21}, спадает как 
\( K_0(\mu_1 r) \), и решение по-прежнему является верным, и это только 
главное слагаемое, которое определяет характер (притяжение или отталкивание) в 
межвихревом взаимодействии на больших расстояниях. Особый интерес представляет 
случай, когда дальнодействующая сила является притягивающей, то есть, когда 
хотя бы один из \( \mu_1, \mu_2 \) является меньше чем \( \mu_A \), так 
анализ линеаризованного уравнения, представленный выше является достаточным 
для текущих целей. \cite{bib:main}

\newpage
    \chapter{Экспериментальная часть}

\section{Минимизация конечностных разностей энергий}

Для расчёт энергии межвихревого взаимодействия мы используем минимизацию 
конечностных разностей энергии.

Функционал свободной энергии ГЛ
\begin{gather}
    F = \frac{1}{2}\sum\limits_{i=1,2}\left[ 
        \left|\left( \nabla + ie\vec{A}\right)\psi_i\right|^2 + 
        \left( 2\alpha_i + \beta_i |\psi_i|^2 \right)|\psi_i|^2 \right] + 
        \nonumber \\
        \frac{1}{2}\left( \nabla\times\vec{A} \right)^2 - 
        \eta|\psi_1||\psi_2|\cos(\theta_2-\theta_1)
    \label{eqm:1}
\end{gather}

Основные состояния вихревых систем и энергии взаимодействия между вихрями 
находятся с помощью минимизации этого функционала при условии соблюдения 
соответствующих ограничений, таких как расположение вихрей.

Для этого численно, мы дискретизации систему регулярной сеткой. Чтобы иметь 
объективный численный результат мы используем сетку адаптивно-узловую типа, с 
шагом \( h \) во всей рассматриваемой области. Дискретизация гамильтониана 
производим конечно-разностным методом

% вставить сюда определение метода LBFGS

Для того чтобы вычислить энергию межвихревого взаимодействия, нужно исправить 
положение вихрей. Фиксация позиции вихря требует особой осторожности, 
чтобы избежать ситуации, когда закрепление на расчетной сетки существенно 
влияет на вихревое решение. Фиксация положения вихря происходит следующим 
образом. В центре вихря плотность конденсата равна нулю. Затем фиксируем 
плотность только доминирующей центральной составляющей компоненты 
\( |\psi_i| \) вихря равным нулю в данной позиции расчетной сетки. Это 
эффективно предотвращает движения вихря, но не препятствует основному 
расщеплению \( |\psi_1| \) и \( |\psi_2| \) за счет магнитного давления. Этот 
метод "точки закрепления" также имеет преимущество перед "малоинвазивным", так 
как только фиксируется положение ядра особенной точки. Таким образом она 
позволяет вычислить средние и дальнодействующие силы с наибольшей точностью. 
Тем не менее, в то же время, очевидно, этот способ не работает для слишком 
малого межвихревого расстояния. Слишком малое межвихревое расстояние приводит 
к следующему легкоузнаваемому артефакту: ядро вихря из одного вихря удлиняется 
до нуля в обоих центрах закрепления, позволяющих убрать второй вихрь, в то 
время, удовлетворяющих минимизации энергии связи. Такое поведение может быть 
легко исправлено используя различные схемы закрепления, а потому, закрепления 
вихрей на малом расстоянии не имеет отношения к вопросам, изучаемых в данной 
работе, а также для обеспечения согласованности используется только одна 
процедуру фиксации.

Сходимость определяется следующим образом:
\begin{enumerate}
    \item Выбирается конкретный шаг сетки \( h_1 \) и число точек сетки 
        \( N_1 = N_{1x} \cdot N_{1y} \) даваемое размером системы
        \( L_x = h \cdot (N_{1x}-1) \), \( L_y = h \cdot (N_{1y}-1) \). 
        Тогда энергия минимизируется пока она не измениться в несколько
        тысячах иттераций. Это даёт \( E(h_1) \).
    \item Уменьшаем шаг сетки \( h \) на коэффициент 2 или 3 при сохранении 
        размера системы \( L_x, L_y \) с помощью сплайн-интерполяции. 
        Затем ещё раз перебираем энергию, пока она не измениться в 
        нескольких тысячах итераций, давая \( E(h_2) \) и так далее. 
        Затем определяем сходимость с помощью формулы
\end{enumerate}
\begin{equation}
    \frac{E(h_n) - E(h_{n+1})}{E(h_n)} = C
\end{equation}

В работе используются сетки размером до \( N \approx 10^7 \) что дает очень 
высокую точность, обычно \( C < 10^{-4} \). \cite{bib:minimization}

\section{Задание начальных условий}

Минимизация начинается с начального приближения: конфигурацию поля, несущего
\( N_v \) квантов потока, описываемого
\begin{gather}
    \Phi_a = u_a \prod\limits_{i=1}^{N_\nu} \sqrt{ 
        \frac{1}{2}\left( 1 + \tanh\left( 
            \frac{4}{\xi}\left( \mathcal{R}_i(x,y) - \xi \right)
        \right) \right)
    } e^{i\Theta_i}
    \nonumber \\
    \vec{A} = \frac{1}{e\mathcal{R}}\left( sin\Theta, -\cos\Theta \right)
    \label{eqm:6}
\end{gather}
где \( a = 1,2, u_a \) является вакуумное среднее каждого скалярного поля, 
параметр \( \xi \) даёт размер ядра, а \( \Theta \) и
\( \mathcal{R} \) определяются из
\begin{gather}
    \Theta(x,y) = \sum\limits_{i=1}^{N_v} \Theta_i(x,y), \nonumber \\
    \Theta_i(x,y) = \tan^{-1}\left(\frac{y-y_i}{x-x_i} \right), \nonumber \\
    \mathcal{R}(x,y) = \sum\limits_{i=1}^{N_v} \mathcal{R}_i(x,y), \nonumber \\
    \mathcal{R}_i(x,y) = \sqrt{(x-x_i)^2+(y-y_i)^2}.
\end{gather}
\( (x_i,y_i) \) является начальным положение данного вихря. Тогда, все степени 
свободы находятся в расслабленном состоянии одновременно без каких-либо 
ограничений для получения высокоточного решения уравнений Гинзбурга-Ландау.
\cite{bib:minimization}

\newpage
    \chapter*{Заключение}
\addcontentsline{toc}{chapter}{Заключение}

В данной работе было представлено аналитическое и численное исследование 
появлении сверхпроводимости типа 1,5 в случае двух зон с различными видами 
существенных межзонных соединений. Во всех случаях, которые были рассмотрены, в 
данной работе было продемонстрировано, что система обладает тремя основными 
масштабами длин: первая \( 1/\mu_A \) связана с Лондоновской глубиной 
проникновения магнитного поля, в то время как остальные две \( 1/\mu_{1,2} \) 
связаны с характеристической масштабов длин ответственные за изменением 
глубины проникновения поля. В пределе двух конденсатов связанных только с 
электромагнитными масштабными длинами \( 1/\mu_{1,2} \) с независимыми  
длинами когерентности двух полей. Было показано, что введение ненулевой 
Джозефсоновской связи и связи типа плотность-плотность делает напряжённости 
полей спадающими по экспоненциальному закону при очень больших расстояниях от 
ядра, в то же время система всё ещё обладает двумя основными масштабными 
длинами, которые связаны с линейной комбинации напряженности полей повернутыми 
на <<угол смешивания>>. Третья основная масштабная длина в этом режиме является 
Лондоновской глубиной проникновения, и таким образом, двухщелевая система со 
связью позволяет точно определить поведения типа 1,5. Далее была написана 
расчётная программа для моделирования структуры абрикосовских вихрей методом 
BFGS и проведён модельный эксперимент в котором было рассмотрена структура 
вихревых формы при различных параметрах ГЛ.

\newpage
    \renewcommand{\bibname}{%
    \vspace{-2em}\begin{center}
        Список используемой литературы
    \end{center}\vspace{-2em}
}

\begin{thebibliography}{10}
    \bibitem{bib:1} N. Sadovnikova, D. Parygin, E. Gnedkova, B. Sanzhapov, and N. Gidkova. 
        Evaluating the sustainability of Volgograd. In The Sustainable City VIII. WIT Press, 2013.
    \bibitem{bib:2} Нестерова А. Новая маршрутная сеть г. Томска представлена общественности 
        [Электронный ресурс] // Сетевое издание Центр дорожной информации. 2015. URL: 
        \url{http://road.perm.ru/index.php?id=1475} (дата обращения: 01.12.2015).
    % ---
    % extra
    \bibitem{bib:2.2} Комплексное обследование остановочных пунктов городского пассажирского 
        транспорта г.Оренбурга
    % ---
    \bibitem{bib:3} Nielsen G., Lange T. Network Design for Public Transport Success -- theory and 
        examples // Norwegian Ministry of Transport and Communications, Oslo. -- 2008.
    \bibitem{bib:4} Holsapple, C. Decisions and Knowledge. Handbook on Decision Support Systems 1, 
        (Cosgrove). [Электронный ресурс]. -- 2008. -- Режим доступа: 
        \url{http://www.springerlink.com/index/g182q711470w2510.pdf}.
    \bibitem{bib:5} Tennenhouse D. Proactive computing //
        Communications of the ACM. -- 2000. -- Т. 43. -- №. 5. -- С. 43-50.
    % -- ПО
    \bibitem{bib:6} PTV Visum [Электронный ресурс] // PTV Group. 2014. URL: 
        \url{http://vision-traffic.ptvgroup.com/en-uk/products/ptv-visum/} 
        (дата обращения: 16.11.2014).
    \bibitem{bib:7} INRO [Электронный ресурс] // Emme. 2014. URL: 
        \url{https://www.inrosoftware.com/en/products/emme/} (дата обращения: 11.11.2014).
    \bibitem{bib:8} Cube [Электронный ресурс] // Citilabs. 2014. URL: 
        \url{http://www.citilabs.com/software/cube/} (дата обращения: 23.11.2014).
    \bibitem{aimsun} Transport Simulation Systems [Электронный ресурс] // Aimsun. URL: 
        \url{https://www.aimsun.com/wp/} (дата обращения: 01.03.2016).
    \bibitem{transims} Transims [Электронный ресурс] // NASA. URL: 
        \url{https://code.google.com/archive/p/transims/} (дата обращения: 01.03.2016).
    % -- ПО
    \bibitem{bib:9} Ceder A. Designing public transport network and routes //
        Advanced Modeling for Transit Operations and Service Planning. -- 2003. -- Т. 3. -- С. 59-91.
    \bibitem{bib:16} Comaniciu D., Meer P. Mean shift: A robust approach toward feature space analysis //
        Pattern Analysis and Machine Intelligence, IEEE Transactions on. -- 2002. -- Т. 24. -- №. 5. -- 
        С. 603-619.
    \bibitem{bib:17} Sadovnikova N. et al. Models and Methods for the Urban Transit System Research //
        Creativity in Intelligent Technologies and Data Science. -- Springer International Publishing, 
        2015. -- С. 488-499. -- (Ser. Communications in Computer and Information Science. Vol. 535)
    \bibitem{bib:18} Ceder A. Designing public transport network and routes //
        Advanced Modeling for Transit Operations and Service Planning. -- 2003. -- Т. 3. -- С. 59-91.
    \bibitem{bib:19} Гладков Л.А., Курейчик В.В., Курейчик В.М. Генетические алгоритмы / 
        Под ред. В.М. Курейчика. -- 2-е изд., испр. и доп. -- М.: ФИЗМАТЛИТ, 2006. -- 320 с.
    \bibitem{bib:20} Golubev A, Chechetkin I, Solnushkin K.S., Sadovnikova N., Parygin D., Shcherbakov M., 
        Brebels A., Strategway: web solutions for building public transportation routes using big geodata 
        analysis // Proceedings of The 17th International Conference on Information Integration and 
        Web-based Applications \& Services (iiWAS2015) (December 11 - 13, 2015 Brussels, Belgium ) 
        ACM New York, New York pp. 665 - 668
    \bibitem{bib:21} Bast H. et al. Route planning in transportation networks //
        arXiv preprint arXiv:1504.05140. -- 2015.
    \bibitem{bib:22} Чалой Е. В., Шамрай Н. Б. Построение матрицы корреспонденций для транспортной 
        сети г. Владивостока. % дипломная работа
    \bibitem{bib:23} Гасников А. и др. (ред.). Введение в математическое моделирование транспортных 
        потоков. -- Litres, 2015.
    \bibitem{bib:24} Гасников А. В., Гасникова Е. В. О возможной динамике в модели расчета матрицы 
        корреспонденций (А. Дж. Вильсона) //ТРУДЫ МФТИ. -- 2010. -- Т. 2. -- №. 4. -- С. 45.
    \bibitem{bib:25} Werneck R. F. Public Transit Labeling //Experimental Algorithms: 
        14th International Symposium, SEA 2015, Paris, France, June 29–July 1, 2015, 
        Proceedings. -- Springer, 2015. -- Т. 9125. -- С. 273.
    % others
    \bibitem{nielsen2008network} Nielsen G., Lange T. Network Design for Public Transport Success–Theory 
        and Examples //Norwegian Ministry of Transport and Communications, Oslo. –- 2008.
    \bibitem{ceder2007} Ceder A. Public Transit Planning and Operation: Theory, Modeling and Practice. 2007.
    \bibitem{rodeheffer2013symmetric} Rodeheffer T. L. The Symmetric Shortest-Path Table Routing 
        Conjecture. -– 2013.
    \bibitem{delling2014round} Delling D., Pajor T., Werneck R. F. Round-based public transit routing //
        Transportation Science. -- 2014. -- Т. 49. -- №. 3. -- С. 591-604.
    \bibitem{delling2015customizable} Delling D. et al. Customizable route planning in road networks //
        Transportation Science. -- 2015.
    \bibitem{delling2015public} Delling D. et al. Public transit labeling //Experimental Algorithms. -- 
        Springer International Publishing, 2015. -- С. 273-285.
    \bibitem{abraham2013alternative} Abraham I. et al. Alternative routes in road networks //Journal of 
        Experimental Algorithmics (JEA). -- 2013. -- Т. 18. -- С. 1.3.
    \bibitem{wei2012constructing} Wei L. Y., Zheng Y., Peng W. C. Constructing popular routes from 
        uncertain trajectories //Proceedings of the 18th ACM SIGKDD international conference on 
        Knowledge discovery and data mining. -- ACM, 2012. -- С. 195-203.
    \bibitem{dwyer2009fast} Dwyer T., Nachmanson L. Fast edge-routing for large graphs //
        Graph Drawing. -- Springer Berlin Heidelberg, 2009. -- С. 147-158.
    % ---
    \bibitem{bib:26} Krumm J. Real time destination prediction based on efficient routes //Society of 
        Automotive Engineers (SAE) 2006 World Congress. -- 2006. -- Т. 7.
    \bibitem{bib:27} Wang Y., Zheng Y., Xue Y. Travel time estimation of a path using sparse 
        trajectories //Proceedings of the 20th ACM SIGKDD international conference on Knowledge 
        discovery and data mining. -- ACM, 2014. -- С. 25-34.
    \bibitem{bib:28} Berlingerio M. et al. AllAboard: a System for Exploring Urban Mobility and 
        Optimizing Public Transport Using Cellphone Data //Mobile Phone Data for Development, Analysis 
        of mobile phone datasets for the development of Ivory Coast, 
        viewed. -- 2014. -- Т. 9. -- С. 397-411.
    \bibitem{bib:29} Krushel E. G. et al. An Experience of Optimization Approach Application to Improve 
        the Urban Passenger Transport Structure //Knowledge-Based Software Engineering. -- Springer 
        International Publishing, 2014. -- С. 27-39.
    \bibitem{bib:30} Mees P. et al. Public transport network planning: a guide to best practice in NZ 
        cities. -- 2010. -- №. 396.
    \bibitem{bib:31} Гузенко А. В. Развитие городского пассажирского транспорта мегаполиса: проблемы 
        и перспективы //Вестник Томского государственного университета. -- 2009. -- №. 321.
    \bibitem{bib:32} Кузьмич С. И., Федина Т. О. Транспортные проблемы современных городов и 
        моделирование загрузки улично-дорожной сети //Известия Тульского государственного 
        университета. Технические науки. -- 2008. -- №. 3.
    \bibitem{bib:33} Андрианов В. Ю. Геоинформационные системы для транспорта и коммуникаций //
        T-Comm-Телекоммуникации и Транспорт. -- 2010. -- №. S2.
    \bibitem{bib:34} Шуравина Е. Н. Проблемы современной транспортной системы россии //
        Вестник Самарского государственного университета. -- 2011. -- №. 90.
    \bibitem{bib:35} Корягин М. Е. Теоретические аспекты оптимизации управления движением городского 
        транспорта //Вестник Кузбасского государственного технического университета. -- 2012. -- №. 1 (89).
    \bibitem{bib:36} Кочегурова Е. А., Мартынова Ю. А. Оптимизация составления маршрутов общественного 
        транспорта при создании автоматизированной системы поддержки принятия решений //
        Известия Томского политехнического университета. -- 2013. -- Т. 323. -- №. 5.
    \bibitem{bib:37} Агуреев И. Е., Митюгин В. А., Пышный В. А. Подготовка и обработка исходных данных 
        для математического моделирования автомобильных транспортных систем //Известия Тульского 
        государственного университета. Технические науки. -- 2014. -- №. 6.
    \bibitem{bib:38} Ефимова Е. А. Сравнительный анализ создания имитационной модели пропускной 
        способности городской транспортной сети //Известия высших учебных заведений. Поволжский регион. 
        Технические науки. -- 2009. -- №. 1.
    \bibitem{bib:39} Хегай Ю. А. Проблемы автомобильного транспорта в россии //Теория и практика 
        общественного развития. -- 2014. -- №. 8.
    \bibitem{bib:40} Синицына Е. Б., Лазарев Ю. Г. Современное состояние проблемы совершенствования 
        транспортной инфраструктуры //Технико-технологические проблемы сервиса. -- 2013. -- №. 4 (26).
    \bibitem{bib:41} Денисов М. В., Агуреев И. Е. Математическое описание динамики пассажирских 
        транспортных систем //Известия Тульского государственного университета. 
        Технические науки. -- 2010. -- №. 4-2.
    \bibitem{bib:42} Палант А. Ю. Обзор Методов Обследования Пассажиропотоков //
        Бизнес Информ. -- 2014. -- №. 11.
    \bibitem{bib:43} Чернов В. П., Кабалина Т. В. Исследование оценки качества в системе критериев 
        эффективности перевозок пассажиров //Актуальные проблемы экономики и права. -- 2010. -- №. 4 (16).
    \bibitem{bib:44} Лойко В. И., Параскевов А. В. Меры по обеспечению эффективной организации городского 
        дорожного движения //Политематический сетевой электронный научный журнал Кубанского 
        государственного аграрного университета. -- 2010. -- №. 64.
    % +
    \bibitem{bib:45} Кочетов Ю. А. Методы локального поиска для дискретных задач размещения //
        Специальность 05.13. 18 математическое моделирование, численные методы и 
        комплексы программ. -- 2011.
    % +
    \bibitem{bib:46} Блох И. И., Дураков А. В. Алгоритмы построения маршрута на карте по параметрам.
    % +
    \bibitem{bib:47} Дасгупта С., Пападимитриу Х., Вазирани У. Алгоритмы //М.: МНЦМО. -- 2014.
    
    %% + (нужно оформить по ГОСТ)
    \bibitem{bib:48} Шербина О. А. Метаэвристические алгоритмы для задач комбинаторной оптимизации 
    %% http://tvim.info/files/56_72_Shcherbina.pdf
    
    % +
    \bibitem{bib:50} Ипатов А. В. Модифицированный метод имитации отжига в задаче маршрутизации 
        транспорта //Труды Института математики и механики УрО РАН. -- 2011. -- Т. 17. -- №. 4. -- С. 121-125.
    % +
    \bibitem{bib:51} Ипатов А. В. Решение задачи маршрутизации транспорта методом имитации отжига //
        Проблемы теорет. и прикл. математики: тр. -- С. 290-294.
    % +
    \bibitem{bib:52} Карпенко А. П. Современные алгоритмы поисковой оптимизации //Алгоритмы, 
        вдохновленные природой: учеб. пособие. М.: Изд-во МГТУ им. НЭ Баумана. -- 2014.
    \bibitem{bib:53} Ковалев М. Я. Теория алгоритмов. Курс лекций: в 2 ч //Минск: БГУ. -- 2003.
    % +
    \bibitem{bib:54} Кочетов Ю. А. Вероятностные методы локального поиска для задач дискретной 
        оптимизации //Дискретная математика и ее приложения. Сборник лекций молодежных и научных школ 
        по дискретной математике и ее приложениям. М: МГУ. -- 2001. -- С. 87-117.
    % +
    \bibitem{bib:55} Кулаков Ю. А., Воротников В. В. Формирование оптимальных маршрутов в мобильных сетях 
        на основе модифицированного алгоритма Дейкстры //Вестник НТУУ <<КПИ>>: Информатика, управление 
        и вычислительная техника. -- 2012. -- Т. 2012. -- №. 56.
    % +
    \bibitem{bib:56} Романовский И. В. Алгоритмы решения экстремальных задач. -- Наука, 1977.
    % +
    \bibitem{bib:57} Кочетов Ю. А., Младенович Н., Хансен П. Локальный поиск с чередующимися окрестностями //
        Дискретный анализ и исследование операций. -- 2003. -- Т. 10. -- №. 1. -- С. 11-43.

    % прочие источники
    % https://mipt.ru/education/chair/computational_mathematics/upload/22b/Book-arpglktefbb.pdf
    % http://zoneos.com/traffic/
    % http://www.mou.mipt.ru/gasnikov1129.pdf
    % Швецов http://www.isa.ru/transnet/TrafficReview.pdf
    % http://spkurdyumov.ru/uploads/2013/08/Semenov.pdf
\end{thebibliography}
% \endgroup
    \addcontentsline{toc}{chapter}{Приложение А Результаты экспериментов}
\begin{center}
    Приложение А\\
    Результаты экспериментов
\end{center}
\newpage
\addtocounter{page}{2}
\addcontentsline{toc}{chapter}{Приложение Б Код программы}
\begin{center}
    Приложение Б\\
    Код программы
\end{center}
\lstinputlisting[language=c++]{../code/calculation/vortex.cpp}
\end{document}
