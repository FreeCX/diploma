\chapter*{Заключение}
\addcontentsline{toc}{chapter}{Заключение}

В данной работе было представлено аналитическое и численное исследование 
появлении сверхпроводимости типа 1,5 в случае двух зон с различными видами 
существенных межзонных соединений. Во всех случаях, которые были рассмотрены, в 
данной работе было продемонстрировано, что система обладает тремя основными 
масштабами длин: первая \( 1/\mu_A \) связана с Лондоновской глубиной 
проникновения магнитного поля, в то время как остальные две \( 1/\mu_{1,2} \) 
связаны с характеристической масштабов длин ответственные за изменением 
глубины проникновения поля. В пределе двух конденсатов связанных только с 
электромагнитными масштабными длинами \( 1/\mu_{1,2} \) с независимыми  
длинами когерентности двух полей. Было показано, что введение ненулевой 
Джозефсоновской связи и связи типа плотность-плотность делает напряжённости 
полей спадающими по экспоненциальному закону при очень больших расстояниях от 
ядра, в то же время система всё ещё обладает двумя основными масштабными 
длинами, которые связаны с линейной комбинации напряжённости полей повернутыми 
на <<угол смешивания>>. Третья основная масштабная длина в этом режиме является 
Лондоновской глубиной проникновения, и таким образом, двущелевая система со 
связью позволяет точно определить поведения типа 1,5. Далее была написана 
расчётная программа для моделирования структуры абрикосовских вихрей методом 
LBFGS и проведён модельный эксперимент в котором было рассмотрена структура 
вихревых формы при различных параметрах ГЛ.

\newpage