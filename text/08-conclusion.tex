\part{Заключение}
В данной работе было рассмотрено общее состояние существующей проблемы в транспортной инфраструктуре. Описаны 
существующие программные продукты частично решающие данную проблему в полуавтоматическом режиме, были 
рассмотрены алгоритмы, которые могут быть использованы для построения маршрутной сети с описанием их плюсов и 
минусов, а также предложенные и рассмотрены несколько новых. Был проанализирован и протестирован алгоритм для 
построения сети маршрутов основанного на обработке геораспределенных данных, а также представлены и объяснены 
три его реализации и произведена оценка производительности данного алгоритма.

Согласно результатам полученным в работе можно сделать следующий вывод -- алгоритм может быть использован для 
создания предварительной сети маршрутов общественного транспорта или как промежуточное звено для создания 
начальной сети с последующей её модификацией.