\chapter{Экспериментальная часть}

\section{Минимизация конечностных разностей энергий}

Для расчёт энергии межвихревого взаимодействия мы используем минимизацию 
конечностных разностей энергии.

Функционал свободной энергии ГЛ
\begin{gather}
    F = \frac{1}{2}\sum\limits_{i=1,2}\left[ 
        \left|\left( \nabla + ie\vec{A}\right)\psi_i\right|^2 + 
        \left( 2\alpha_i + \beta_i |\psi_i|^2 \right)|\psi_i|^2 \right] + 
        \nonumber \\
        \frac{1}{2}\left( \nabla\times\vec{A} \right)^2 - 
        \eta|\psi_1||\psi_2|\cos(\theta_2-\theta_1)
    \label{eqm:1}
\end{gather}

Основные состояния вихревых систем и энергии взаимодействия между вихрями 
находятся с помощью минимизации этого функционала при условии соблюдения 
соответствующих ограничений, таких как расположение вихрей.

Для этого численно, мы дискретизации систему регулярной сеткой. Чтобы иметь 
объективный численный результат мы используем сетку адаптивно-узловую типа, с 
шагом \( h \) во всей рассматриваемой области. Дискретизация гамильтониана 
производим конечно-разностным методом

% вставить сюда определение метода LBFGS

Для того чтобы вычислить энергию межвихревого взаимодействия, нужно исправить 
положение вихрей. Фиксация позиции вихря требует особой осторожности, 
чтобы избежать ситуации, когда закрепление на расчетной сетки существенно 
влияет на вихревое решение. Фиксация положения вихря происходит следующим 
образом. В центре вихря плотность конденсата равна нулю. Затем фиксируем 
плотность только доминирующей центральной составляющей компоненты 
\( |\psi_i| \) вихря равным нулю в данной позиции расчетной сетки. Это 
эффективно предотвращает движения вихря, но не препятствует основному 
расщеплению \( |\psi_1| \) и \( |\psi_2| \) за счет магнитного давления. Этот 
метод "точки закрепления" также имеет преимущество перед "малоинвазивным", так 
как только фиксируется положение ядра особенной точки. Таким образом она 
позволяет вычислить средние и дальнодействующие силы с наибольшей точностью. 
Тем не менее, в то же время, очевидно, этот способ не работает для слишком 
малого межвихревого расстояния. Слишком малое межвихревое расстояние приводит 
к следующему легкоузнаваемому артефакту: ядро вихря из одного вихря удлиняется 
до нуля в обоих центрах закрепления, позволяющих убрать второй вихрь, в то 
время, удовлетворяющих минимизации энергии связи. Такое поведение может быть 
легко исправлено используя различные схемы закрепления, а потому, закрепления 
вихрей на малом расстоянии не имеет отношения к вопросам, изучаемых в данной 
работе, а также для обеспечения согласованности используется только одна 
процедуру фиксации.

Сходимость определяется следующим образом:
\begin{enumerate}
    \item Выбирается конкретный шаг сетки \( h_1 \) и число точек сетки 
        \( N_1 = N_{1x} \cdot N_{1y} \) даваемое размером системы
        \( L_x = h \cdot (N_{1x}-1) \), \( L_y = h \cdot (N_{1y}-1) \). 
        Тогда энергия минимизируется пока она не измениться в несколько
        тысячах иттераций. Это даёт \( E(h_1) \).
    \item Уменьшаем шаг сетки \( h \) на коэффициент 2 или 3 при сохранении 
        размера системы \( L_x, L_y \) с помощью сплайн-интерполяции. 
        Затем ещё раз перебираем энергию, пока она не измениться в 
        нескольких тысячах итераций, давая \( E(h_2) \) и так далее. 
        Затем определяем сходимость с помощью формулы
\end{enumerate}
\begin{equation}
    \frac{E(h_n) - E(h_{n+1})}{E(h_n)} = C
\end{equation}

В работе используются сетки размером до \( N \approx 10^7 \) что дает очень 
высокую точность, обычно \( C < 10^{-4} \). \cite{bib:minimization}

\section{Задание начальных условий}

Минимизация начинается с начального приближения: конфигурацию поля, несущего
\( N_v \) квантов потока, описываемого
\begin{gather}
    \Phi_a = u_a \prod\limits_{i=1}^{N_\nu} \sqrt{ 
        \frac{1}{2}\left( 1 + \tanh\left( 
            \frac{4}{\xi}\left( \mathcal{R}_i(x,y) - \xi \right)
        \right) \right)
    } e^{i\Theta_i}
    \nonumber \\
    \vec{A} = \frac{1}{e\mathcal{R}}\left( sin\Theta, -\cos\Theta \right)
    \label{eqm:6}
\end{gather}
где \( a = 1,2, u_a \) является вакуумное среднее каждого скалярного поля, 
параметр \( \xi \) даёт размер ядра, а \( \Theta \) и
\( \mathcal{R} \) определяются из
\begin{gather}
    \Theta(x,y) = \sum\limits_{i=1}^{N_v} \Theta_i(x,y), \nonumber \\
    \Theta_i(x,y) = \tan^{-1}\left(\frac{y-y_i}{x-x_i} \right), \nonumber \\
    \mathcal{R}(x,y) = \sum\limits_{i=1}^{N_v} \mathcal{R}_i(x,y), \nonumber \\
    \mathcal{R}_i(x,y) = \sqrt{(x-x_i)^2+(y-y_i)^2}.
\end{gather}
\( (x_i,y_i) \) является начальным положение данного вихря. Тогда, все степени 
свободы находятся в расслабленном состоянии одновременно без каких-либо 
ограничений для получения высокоточного решения уравнений Гинзбурга-Ландау.
\cite{bib:minimization}

\newpage